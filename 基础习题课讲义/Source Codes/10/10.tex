\documentclass[12pt,UTF8]{ctexart}
\usepackage{ctex,amsmath,amssymb,geometry,fancyhdr,bm,amsfonts
,mathtools,extarrows,graphicx,url,enumerate,color,float,multicol} 
% 加入中文支持
\newcommand\Set[2]{%
\left\{#1\ \middle\vert\ #2 \right\}}
\geometry{a4paper,scale=0.80}
\pagestyle{fancy}
\rhead{习题7.1\&7.2\&7.3\&7.4}
\lhead{基础习题课讲义}
\chead{微积分B(1)}
\begin{document}
\setcounter{section}{9}
\section{定积分(1)}
\noindent
\subsection{知识结构}
\noindent第7章定积分
	\begin{enumerate}
		\item[7.1] 积分概念和积分存在的条件
			\begin{enumerate}
				\item[7.1.1] 曲边梯形的面积
				\item[7.1.2] 黎曼积分的定义
				\item[7.1.3] 积分存在的条件
			\end{enumerate}
		\item[7.2] 定积分的性质
			\begin{enumerate}
				\item方向性
				\item线性性质
				\item区间可加性
				\item特殊函数定积分的性质(奇偶性、周期性)
				\item比较定理(保号性)
					\begin{itemize}
						\item估值定理
					\end{itemize}
				\item绝对值函数的可积性
				\item积分中值定理
					\begin{itemize}
						\item第一积分中值定理
					\end{itemize}
			\end{enumerate}
		\item[7.3]变上限积分与牛顿-莱布尼茨公式
			\begin{enumerate}
				\item[7.3.1]变上限积分
				\item[7.3.2]牛顿-莱布尼茨公式
			\end{enumerate}
		\item[7.4]定积分的换元积分法和分部积分法
			\begin{enumerate}
				\item[7.4.1]定积分的换元积分法
				\item[7.4.2]定积分的分部积分法
			\end{enumerate}
%		\item[7.5]定积分的几何应用
%			\begin{enumerate}
%				\item[7.5.1]平面图形的面积
%				\item[7.5.2]曲线长度的计算
%				\item[7.5.3]平面曲线的曲率
%				\item[7.5.4]旋转体的体积与旋转曲面的面积
%			\end{enumerate}
	\end{enumerate}
\subsection{习题7.1解答}
\begin{enumerate}
\item用积分的几何意义计算下列定积分:
\newline
(1)$\int_1^3(1+2x)\mathrm dx$;
\newline
(2)$\int_{-3}^0\sqrt{9-x^2}\mathrm dx$.

解:(1)积分$\int_1^3(1+2x)\mathrm dx$表示直线$y=1+2x,x=1,x=3$和$x$轴围成的梯形面积,故$\int_1^3(1+2x)\mathrm dx=(3+7)\times2/2=10$.

(2)积分$\int_{-3}^0\sqrt{9-x^2}\mathrm dx$表示圆$x^2+y^2=9$在第二象限部分的面积,则$\int_{-3}^0\sqrt{9-x^2}\mathrm dx=\frac14\pi\times9=\frac94\pi$.
\item利用定理7.1.1证明狄利克雷函数在区间$[0,1]$上不可积.

证明:对于狄利克雷函数$D(x)=\begin{cases}
1,&x\in\mathbb Q\\
0,&x\notin\mathbb Q
\end{cases}$在区间$[0,1]$上的任意一个分割$T:a=x_0<x_1<\cdots<x_n=b$,其中每一个区间$[x_{i-1},x_i]$上的振幅$\omega_i$均为$1$,则$\sum_{i=1}^n\omega_i\Delta x_i=\sum_{i=1}^n\Delta x_i=1$,故$\lim\limits_{\lambda\rightarrow0}\sum_{i=1}^n\omega_i\Delta x_i=1\neq0$,故狄利克雷函数在区间$[0,1]$上不可积.
\item利用定理7.1.1证明:若$f\in R[a,b]$,则$|f|\in R[a,b],f^2\in R[a,b]$.

证明:$\because f\in R[a,b]$

$\therefore$对于区间$[a,b]$上的任意一个分割$T:a=x_0<x_1<\cdots<x_n=b$,有$\lim\limits_{\lambda\rightarrow0}\sum_{i=1}^n\omega_i\Delta x_i=0$

对于同一个分割$T$,记$|f|$的振幅为$\omega_i^*$,易知$\omega_i^*\leq\omega_i$

$\therefore0\leq\sum_{i=1}^n\omega_i^*\Delta x_i\leq\sum_{i=1}^n\omega_i\Delta x_i$

$\therefore\lim\limits_{\lambda\rightarrow0}\sum_{i=1}^n\omega_i^*\Delta x_i=0$

$\therefore|f|\in R[a,b]$

对于上述分割$T$,记$f^2$的振幅为$\omega_i^{**}$,对于区间$[x_{i-1},x_i]$,设$|f|$在其中的最大值和最小值分别为$M_i,m_i$,则$\omega_i^{**}=M_i^2-m_i^2=(M_i-m_i)(M_i+m_i)=(M_i+m_i)\omega_i^{**}$

由$f\in R[a,b]$知$f$有界,从而$|f|$有界,故$\exists A>0,s.t.M_i+m_i<A$

$\therefore\omega_i^{**}\leq A\omega_i^{**}$

$\therefore0\leq\sum_{i=1}^n\omega_i^{**}\Delta x_i\leq\sum_{i=1}^nA\omega_i^{*}\Delta x_i=A\sum_{i=1}^n\omega_i^{*}\Delta x_i$

$\therefore\lim\limits_{\lambda\rightarrow0}\sum_{i=1}^n\omega_i^{**}\Delta x_i=0$

$\therefore f^2\in R[a,b]$.
\item举例说明:由$|f|\in R[a,b]$一般不能推出$f\in R[a,b]$.

解:比如函数$f(x)=\begin{cases}
1,&x\in\mathbb Q\\
-1,&x\notin\mathbb Q
\end{cases}$,$|f(x)|=1,x\in\mathbb R$,$|f|\in C[0,1]$故$|f|\in R[0,1]$,但对区间$[0,1]$上的任意一个分割$T:a=x_0<x_1<\cdots<x_n=b$,$f$在其中每一个区间$[x_{i-1},x_i]$上的振幅$\omega_i$均为$2$,则$\sum_{i=1}^n\omega_i\Delta x_i=\sum_{i=1}^n\Delta x_i=2$,故$\lim\limits_{\lambda\rightarrow0}\sum_{i=1}^n\omega_i\Delta x_i=2\neq0$,故$f\notin R[a,b]$.
\end{enumerate}
\subsection{习题7.2解答}
\begin{enumerate}
\item比较下列每组中两个积分的大小:
\newline
(1)$\int_0^1\mathrm e^x\mathrm dx,\int_0^1\mathrm e^{x^2}\mathrm dx$.
\newline
(2)$\int_0^{\frac\pi2}\sin x\mathrm dx,\int_0^1\sin(\sin x)\mathrm dx$.

解:(1)当$x\in(0,1)$时,$x-x^2=x(1-x)>0$,故$\mathrm e^x-\mathrm e^{x^2}=\mathrm e^{x^2}(\mathrm e^{x-x^2}-1)>0$

$\therefore\int_0^1\mathrm e^x\mathrm dx>\int_0^1\mathrm e^{x^2}\mathrm dx$

(2)当$x\in(0,\frac\pi2)$时,$0<\sin x<x<\frac\pi2$,故$\sin x>\sin(\sin x)$

$\therefore\int_0^{\frac\pi2}\sin x\mathrm dx>\int_0^{\frac\pi2}\sin(\sin x)\mathrm dx>\int_0^1\sin(\sin x)\mathrm dx$.
\item证明下列不等式
\newline
(1)$\frac2{\sqrt[4]{\mathrm e}}<\int_0^2\mathrm e^{x^2-x}\mathrm dx<2\mathrm e^2$;
\newline
(2)$\int_0^{2\pi}|a\sin x+b\cos x|\mathrm dx\leq2\pi\sqrt{a^2+b^2}$.

证明:(1)当$x\in[0,2]$时,$-\frac14\leq x^2-x\leq2$,故$-\frac1{\sqrt[4]{\mathrm e}}\leq\mathrm e^{x^2-x}\leq\mathrm e^2$

$\therefore\int_0^2\frac1{\sqrt[4]{\mathrm e}}\mathrm dx<\int_0^2\mathrm e^{x^2-x}\mathrm dx<\int_0^2\mathrm e^2\mathrm dx$

即$\frac2{\sqrt[4]{\mathrm e}}<\int_0^2\mathrm e^{x^2-x}\mathrm dx<2\mathrm e^2$.

(2)$|a\sin x+b\cos x|=\sqrt{a^2+b^2}|\frac a{\sqrt{a^2+b^2}}\sin x+\frac b{\sqrt{a^2+b^2}}\cos x|=\sqrt{a^2+b^2}\sin(x+\phi)\leq\sqrt{a^2+b^2}$

$\therefore\int_0^{2\pi}|a\sin x+b\cos x|\mathrm dx\leq\int_0^{2\pi}\pi\sqrt{a^2+b^2}\mathrm dx=2\pi\sqrt{a^2+b^2}$.
\item证明下列等式
\newline
(1)$\lim\limits_{A\rightarrow+\infty}\int_A^{A+1}\frac{\cos x}x\mathrm dx=0$.
\newline
(2)$\lim\limits_{p\rightarrow+\infty}\int_0^{\frac\pi2}\sin^px\mathrm dx=0$.

证明:(1)$0\leq|\int_A^{A+1}\frac{\cos x}x|\mathrm dx\leq\int_A^{A+1}|\frac{\cos x}x|\mathrm dx\leq\int_A^{A+1}\frac1x\mathrm dx=\frac1\xi(A+1-A)=\frac1\xi,A<\xi<A+1$

$\because\lim\limits_{A\rightarrow+\infty}\frac1\xi=\lim\limits_{\xi\rightarrow+\infty}\frac1\xi=0$

$\therefore\lim\limits_{A\rightarrow+\infty}|\int_A^{A+1}\frac{\cos x}x|=0$

$\therefore\lim\limits_{A\rightarrow+\infty}\int_A^{A+1}\frac{\cos x}x=0$.

(2)$\forall\varepsilon>0(\text{不妨设}\varepsilon<\frac\pi2),\int_0^{\frac\pi2}\sin^px\mathrm dx=\int_0^{\frac\pi2-\varepsilon}\sin^px\mathrm dx+\int_{\frac\pi2-\varepsilon}^{\frac\pi2}\sin^px\mathrm dx$

$\because\int_0^{\frac\pi2-\varepsilon}\sin^px\mathrm dx\leq\int_0^{\frac\pi2-\varepsilon}\sin^p(\frac\pi2-\varepsilon)\mathrm dx=(\frac\pi2-\varepsilon)\sin^p(\frac\pi2-\varepsilon)$

$\because0<\sin x<1$

$\therefore$对于该$\varepsilon,\exists P>0,s.t.\int_0^{\frac\pi2-\varepsilon}\sin^px\mathrm dx\leq(\frac\pi2-\varepsilon)\sin^p(\frac\pi2-\varepsilon)<\varepsilon(p>P)$

$\because\int_{\frac\pi2-\varepsilon}^{\frac\pi2}\sin^px\mathrm dx\leq\int_{\frac\pi2-\varepsilon}^{\frac\pi2}1\mathrm dx=\varepsilon$

$\therefore|\int_0^{\frac\pi2}\sin^px\mathrm dx-0|=\int_0^{\frac\pi2}\sin^px\mathrm dx=\int_0^{\frac\pi2-\varepsilon}\sin^px\mathrm dx+\int_{\frac\pi2-\varepsilon}^{\frac\pi2}\sin^px\mathrm dx<2\varepsilon$

$\therefore\lim\limits_{p\rightarrow+\infty}\int_0^{\frac\pi2}\sin^px\mathrm dx=0$.

\item证明下列不等式:
\newline
(1)$\int_1^n\ln x\mathrm dx<\ln n!$;
\newline
(2)$f\in C[0,1],f(0)=0,f(1)=1,f''(x)>0$则$\int_0^1f(x)\mathrm dx<\frac12$.

证明:(1)$\int_1^n\ln x\mathrm dx=\int_1^2\ln x\mathrm dx+\int_2^3\ln x\mathrm dx+\cdots+\int_{n-1}^n\ln x\mathrm dx<\int_1^2\ln 1\mathrm dx+\int_2^3\ln 2\mathrm dx+\cdots+\int_{n-1}^n\ln n\mathrm dx=\ln 2+\ln 3+\cdots+\ln n=\ln n!$.

(2)$\because f''(x)>0,x\in[0,1]$

$\therefore f(x)$在区间$[0,1]$上严格下凸

又$\because f(0)=0,f(1)=1$

$\therefore f(x)$在直线$y=x$的下方,即$f(x)<x,x\in(0,1)$

$\therefore\int_0^1f(x)\mathrm dx<\int_0^1x\mathrm dx=\frac12$.
\end{enumerate}
\subsection{习题7.3解答}
\begin{enumerate}
\item求下列变限积分的导数:
\newline
(1)$\frac{\mathrm d}{\mathrm dx}\int_0^x\sqrt{1+t}\mathrm dt$;
\newline
(2)$\frac{\mathrm d}{\mathrm dx}\int_x^{x^2}\frac{\mathrm dt}{\sqrt{1+t}}$;
\newline
(3)$\frac{\mathrm d}{\mathrm dx}\int_0^x\sin x\cos t^2\mathrm dt$;
\newline
(4)$\frac{\mathrm d}{\mathrm dx}\int_0^{x^2}\sqrt{1+t}\mathrm dt$.

解:(1)$\frac{\mathrm d}{\mathrm dx}\int_0^x\sqrt{1+t}\mathrm dt=\sqrt{1+x}$.

(2)$\frac{\mathrm d}{\mathrm dx}\int_x^{x^2}\frac{\mathrm dt}{\sqrt{1+t}}=\frac{\mathrm d}{\mathrm dx}(\int_x^0\frac{\mathrm dt}{\sqrt{1+t}}+\int_0^{x^2}\frac{\mathrm dt}{\sqrt{1+t}})=-\frac1{\sqrt{1+x}}+\frac{2x}{\sqrt{1+x^2}}$.

(3)$\frac{\mathrm d}{\mathrm dx}\int_0^x\sin x\cos t^2\mathrm dt=\cos x\int_0^x\cos t^2\mathrm dt+\sin x\cos x^2$.

(4)$\frac{\mathrm d}{\mathrm dx}\int_0^{x^2}\sqrt{1+t}\mathrm dt=\sqrt{1+x^2}\cdot2x=2x\sqrt{1+x^2}$.

\item求下列极限:
\newline
(1)$\lim\limits_{x\rightarrow0}\frac{\int_0^x\cos t^2\mathrm dt}{\ln(1+x)}$;
\newline
(2)$\lim\limits_{x\rightarrow0}\frac{(\int_0^x\sin t\mathrm dt)^2}{\int_0^x\sin t^2\mathrm dt}$.

解:(1)$\lim\limits_{x\rightarrow0}\frac{\int_0^x\cos t^2\mathrm dt}{\ln(1+x)}=\lim\limits_{x\rightarrow0}\frac{\int_0^x\cos t^2\mathrm dt}x=\lim\limits_{x\rightarrow0}\frac{\cos x^2}1=1$.

(2)$\lim\limits_{x\rightarrow0}\frac{(\int_0^x\sin t\mathrm dt)^2}{\int_0^x\sin t^2\mathrm dt}=\lim\limits_{x\rightarrow0}\frac{2(\int_0^x\sin t\mathrm dt)\sin x}{\sin x^2}=\lim\limits_{x\rightarrow0}\frac{2(\int_0^x\sin t\mathrm dt)x}{x^2}=\lim\limits_{x\rightarrow0}\frac{2\int_0^x\sin t\mathrm dt}x=\lim\limits_{x\rightarrow0}\frac{2\sin t}1=0$.

\item用牛顿-莱布尼茨公式计算下列积分($m,k$是整数):
\newline
\begin{tabular}{ll}
(1)$\int_0^1x(1-2x^2)^8\mathrm dx$;&(2)$\int_0^\pi(a\cos x+b\sin x)\mathrm dx$;\\
(3)$\int_{\mathrm e}^{\mathrm e^2}\frac{\mathrm dx}{x\ln x}$;&(4)$\int_{-1}^0(x+1)\sqrt{1-x-\frac12x^2}\mathrm dx$;\\
(5)$\int_{-\pi}^\pi\sin mx\sin kx\mathrm dx$;&(6)$\int_{-\pi}^\pi\cos mx\cos kx\mathrm dx$;\\
(7)$\int_{-\pi}^\pi\sin mx\cos kx\mathrm dx$;&(8)$\int_{-\pi}^\pi\sqrt{1-\cos^2x}\mathrm dx$.
\end{tabular}

解:(1)$\int x(1-2x^2)^8\mathrm dx=-\frac14\int(1-2x^2)^8\mathrm d(1-2x^2)=-\frac14\frac19(1-2x^2)^9+C=-\frac1{36}(1-2x^2)^9$

$\int_0^1x(1-2x^2)^8\mathrm dx=[-\frac1{36}(1-2x^2)^9]|_{x=1}-[-\frac1{36}(1-2x^2)^9]|_{x=0}=\frac1{18}$.

(2)$\int_0^\pi(a\cos x+b\sin x)\mathrm dx=(a\sin x-b\cos x)|_{x=\pi}-(a\sin x-b\cos x)|_{x=0}=2b$.

(3)$\int_{\mathrm e}^{\mathrm e^2}\frac{\mathrm dx}{x\ln x}=\ln\ln x|_{x=\mathrm e^2}-\ln\ln x|_{x=\mathrm e}=\ln2$.

(4)$\int(x+1)\sqrt{1-x-\frac12x^2}\mathrm dx=-\int\sqrt{\frac32-\frac12(x+1)^2}\mathrm d[\frac32-\frac12(1+x)^2]\\
=-\frac23[\frac32-\frac12(1+x)^2]^{\frac32}+C$

$\int_{-1}^0(x+1)\sqrt{1-x-\frac12x^2}\mathrm dx=-\frac23[\frac32-\frac12(1+x)^2]^{\frac32}|_{x=0}+\frac23[\frac32-\frac12(1+x)^2]^{\frac32}|_{x=-1}=-\frac23[\frac32-\frac12(1+0)^2]^{\frac32}+\frac23[\frac32-\frac12(1-1)^2]^{\frac32}=-\frac23+\frac23(\frac32)^{\frac32}=\sqrt{\frac32}-\frac23$.

(5)$\int\sin mx\sin kx\mathrm dx=\frac12\int[\cos(m-k)x-\cos(m+k)x]\mathrm dx\\
=\begin{cases}
\frac12(x-\frac1{2m}\sin2mx)+C,&m=k,\\
\frac1{2(m-k)}\sin(m-k)x-\frac1{2(m+k)}\sin(m+k)x+C,&m\neq k,
\end{cases}m,n\in\mathbb Z^+$

$\int_{-\pi}^\pi\sin mx\sin kx\mathrm dx=\begin{cases}
\pi,&m=k\neq0,\\
0,&m\neq k,
\end{cases}m,n\in\mathbb Z^+$.

(6)$\int\cos mx\cos kx\mathrm dx=\frac12\int[\cos(m-k)x+\cos(m+k)x]\mathrm dx\\
=\begin{cases}
\frac12(x+\frac1{2m}\sin2mx)+C,&m=k,\\
\frac12[\frac1{m-k}\sin(m-k)x+\frac1{m+k}\sin(m+k)x]+C,&m\neq k,
\end{cases}m,n\in\mathbb Z^+$

$\int_{-\pi}^\pi\cos mx\cos kx\mathrm dx=\begin{cases}
\pi,&m=k\neq0,\\
0,&m\neq k,
\end{cases}m,n\in\mathbb Z^+$.

(7)$\int\sin mx\cos kx\mathrm dx=\frac12\int[\sin(m+k)x+\sin(m-k)x]\mathrm dx\\
=\begin{cases}
-\frac12\frac1{2m}\cos2mx+C,&m=k,\\
-\frac12[\frac1{m+k}\cos(m+k)x+\frac1{m-k}\cos(m-k)x]+C,&m\neq k,
\end{cases}m,n\in\mathbb Z^+$

$\int_{-\pi}^\pi\sin mx\cos kx\mathrm dx=0$.

(8)$\int_{-\pi}^\pi\sqrt{1-\cos^2x}\mathrm dx=\int_{-\pi}^\pi|\sin x|\mathrm dx=-\int_{-\pi}^0\sin x\mathrm dx+\int_0^\pi\sin x\mathrm dx\\
=\cos x|_{x=0}-\cos x|_{x=-\pi}+(-\cos x)|_{x=\pi}-(-\cos x)|_{x=0}=4$.

\item计算$\int_{-1}^2\max\{x,x^2\}\mathrm dx$.

解:$\int_{-1}^2\max\{x,x^2\}\mathrm dx=\int_{-1}^0x^2\mathrm dx+\int_0^1x\mathrm dx+\int_1^2x^2\mathrm dx\\
=\frac13x^3|_{x=0}-\frac13x^3|_{x=-1}+\frac12x^2|_{x=1}-\frac12x^2|_{x=0}+\frac13x^3|_{x=2}-\frac13x^3|_{x=1}=\frac13+\frac12+\frac83-\frac13=\frac{19}6$.

\item用定积分求下列极限:
\newline
(1)$\lim\limits_{n\rightarrow\infty}\frac{1^p+2^p+\cdots+n^p}{n^{p+1}},p>0$;
\newline
(2)$\lim\limits_{n\rightarrow\infty}(\frac1{n+1}+\frac1{n+2}+\cdots+\frac1{2n})$;
\newline
(3)$\lim\limits_{n\rightarrow\infty}\frac1n(\sin\frac\pi n+\sin\frac{2\pi}n+\cdots+\sin\frac{(n-1)\pi}n)$;
\newline
(4)$\lim\limits_{n\rightarrow\infty}\frac{\sqrt[n]{(2n)!}}{n\sqrt[n]{n!}}$.

解:(1)$\lim\limits_{n\rightarrow\infty}\frac{1^p+2^p+\cdots+n^p}{n^{p+1}}=\lim\limits_{n\rightarrow\infty}\frac1n\sum_{k=1}^n(\frac kn)^p=\lim\limits_{n\rightarrow\infty}\sum_{k=1}^n(\frac kn)^p\cdot\frac1n=\int_0^1x^p\mathrm dx=\frac1{p+1}$.

(2)$\lim\limits_{n\rightarrow\infty}(\frac1{n+1}+\frac1{n+2}+\cdots+\frac1{2n})=\lim\limits_{n\rightarrow\infty}\frac1n\sum_{k=1}^n\frac1{1+\frac kn}=\lim\limits_{n\rightarrow\infty}\sum_{k=1}^n\frac1{1+\frac kn}\cdot\frac1n=\int_1^2\frac1x\mathrm dx=\ln2$.

(3)$\lim\limits_{n\rightarrow\infty}\frac1n(\sin\frac\pi n+\sin\frac{2\pi}n+\cdots+\sin\frac{(n-1)\pi}n)=\lim\limits_{n\rightarrow\infty}\frac1n(\sin\frac\pi n+\sin\frac{2\pi}n+\cdots+\sin\frac{(n-1)\pi}n+\sin\frac{n\pi}n)\\
=\lim\limits_{n\rightarrow\infty}\frac1n\sum_{k=1}^n\sin\frac{k\pi}n=\lim\limits_{n\rightarrow\infty}\sum_{k=1}^n\sin\frac{k\pi}n\cdot\frac1n=\int_0^\pi\sin \pi x\mathrm dx=\frac2\pi$.

(4)$\lim\limits_{n\rightarrow\infty}\frac{\sqrt[n]{(2n)!}}{n\sqrt[n]{n!}}=\lim\limits_{n\rightarrow\infty}\sqrt[n]{\frac{2n\cdot(2n-1)\cdot(2n-2)\cdot\cdots\cdot(n+1)}{n^n}}\\
=\lim\limits_{n\rightarrow\infty}\sqrt[n]{(1+\frac nn)\cdot(1+\frac{n-1}n)\cdot(1+\frac{n-2}n)\cdot\cdots\cdot(1+\frac1n)}=\lim\limits_{n\rightarrow\infty}\mathrm e^{\frac1n\sum_{k=1}^n\ln(1+\frac kn)}\\
=\lim\limits_{n\rightarrow\infty}\mathrm e^{\sum_{k=1}^n\ln(1+\frac kn)\cdot\frac1n}=\lim\limits_{n\rightarrow\infty}\mathrm e^{\int_1^2\ln x\mathrm dx}=\mathrm e^{(x\ln x-x)|_{x=2}-(x\ln x-x)|_{x=1}}=\mathrm e^{2\ln2-1}=\frac4{\mathrm e}$.

\item假设$f(x)$连续、单调增加. 求证:$\int_{-\pi}^\pi f(x)\sin x\mathrm dx>0$.

证明:$\int_{-\pi}^\pi f(x)\sin x\mathrm dx=\int_{-\pi}^0 f(x)\sin x\mathrm dx+\int_0^\pi f(x)\sin x\mathrm dx\\
=\int_0^\pi f(t-\pi)\sin (t-\pi)\mathrm dt+\int_0^\pi f(x)\sin x\mathrm dx\\
=-\int_0^\pi f(t-\pi)\sin t\mathrm dt+\int_0^\pi f(x)\sin x\mathrm dx\\
=\int_0^\pi[f(x)-f(x-\pi)]\sin x\mathrm dx$

当$0<x<\pi$时,$x-\pi<x,\sin x>0$,故$f(x-\pi)-f(x)<0$,$[f(x)-f(x-\pi)]\sin x>0$

$\therefore\int_{-\pi}^\pi f(x)\sin x\mathrm dx=\int_0^\pi[f(x)-f(x-\pi)]\sin x\mathrm dx>0$.
\end{enumerate}
\subsection{习题7.4解答}
\begin{enumerate}
\item求下列定积分:
\newline
\begin{tabular}{ll}
(1)$\int_0^3\frac x{1+\sqrt{1+x}}\mathrm dx$;&(2)$\int_1^{\mathrm e}\frac{1+\ln x}x\mathrm dx$;\\
(3)$\int_{\frac1\pi}^{\frac2\pi}\frac{\sin\frac1x}{x^2}\mathrm dx$;&(4)$\int_{\frac12}^{\frac34}\frac{\arctan\sqrt x}{\sqrt{x(1-x)}}\mathrm dx$;\\
(5)$\int_1^2\frac{\sqrt{4-x^2}}{x^2}\mathrm dx$;&(6)$\int_0^2\sqrt{(4-x^2)^3}\mathrm dx$;\\
(7)$\int_0^4\frac{\sqrt x}{1+x\sqrt x}\mathrm dx$;&(8)$\int_0^{\frac\pi4}\frac{\mathrm dx}{1+\cos^2x}$;\\
(9)$\int_0^{\ln 2}\sqrt{\mathrm e^x-1}\mathrm dx$;&(10)$\int_{\sqrt2}^2\frac{\mathrm dx}{x\sqrt{x^2-1}}$;\\
(11)$\int_0^1\ln(1+x^2)\mathrm dx$;&(12)$\int_0^{\mathrm e}x(\ln x)^2\mathrm dx$;\\
(13)$\int_0^4\cos(\sqrt x-1)\mathrm dx$;&(14)$\int_0^1x\arctan x\mathrm dx$;\\
(15)$\int_0^1\arcsin x\mathrm dx$;&(16)$\int_0^1\mathrm e^{\sqrt x}\mathrm dx$;\\
(17)$\int_0^{\sqrt{\ln2}}x^3\mathrm e^{-x^2}\mathrm dx$;&(18)$\int_0^1\frac{x\mathrm e^x}{(1+x)^2}\mathrm dx$;\\
(19)$\int_1^{\mathrm e}\cos(\ln x)\mathrm dx$;&(20)$\int_1^2\frac{x\mathrm e^x}{(\mathrm e^x-1)^2}\mathrm dx$.
\end{tabular}

解:(1)$\int_0^3\frac x{1+\sqrt{1+x}}\mathrm dx\xlongequal{t=\sqrt{1+x}}\int_1^2\frac{t^2-1}{1+t}2t\mathrm dt=2\int_1^2(t^2-t)\mathrm dt=2(\frac13t^3-\frac12t^2)\Big|_1^2=\frac53$.

(2)$\int_1^{\mathrm e}\frac{1+\ln x}x\mathrm dx=\int_1^{\mathrm e}(1+\ln x)\mathrm d\ln x=[\ln x+\frac12(\ln x)^2]_1^{\mathrm e}=\frac32$.

(3)$\int_{\frac1\pi}^{\frac2\pi}\frac{\sin\frac1x}{x^2}\mathrm dx=-\int_{\frac1\pi}^{\frac2\pi}\sin\frac1x\mathrm d(\frac1x)=\cos\frac1x\Big|_{\frac1\pi}^{\frac2\pi}=1$.

(4)$\int_{\frac12}^{\frac34}\frac{\arcsin\sqrt x}{\sqrt{x(1-x)}}\mathrm dx\xlongequal{t=\arcsin\sqrt x}\int_{\frac\pi4}^{\frac\pi3}\frac t{\sqrt{\sin^2t(1-\sin^2t)}}2\sin t\cos t\mathrm dt=\int_{\frac\pi4}^{\frac\pi3}2t\mathrm dt=t^2\Big|_{\frac\pi4}^{\frac\pi3}=\frac{7\pi^2}{144}$.

(5)$\int_1^2\frac{\sqrt{4-x^2}}{x^2}\mathrm dx\xlongequal{x=2\sin t}\int_{\frac\pi6}^{\frac\pi2}\frac{2\cos t}{4\sin^2t}2\cos t\mathrm dt=\int_{\frac\pi6}^{\frac\pi2}(\csc^2t-1)\mathrm dt=-\cot t-t\Big|_{\frac\pi6}^{\frac\pi2}=\sqrt3-\frac\pi3=3\pi$.

(6)$\int_0^2\sqrt{(4-x^2)^3}\mathrm dx\xlongequal{x=2\sin t}\int_0^{\frac\pi2}\sqrt{(4-4\sin^2t)^3}2\cos t\mathrm dt=16\int_0^{\frac\pi2}\cos^4t\mathrm dt=16(\frac{3\cdot1}{4\cdot2}\frac\pi2)\\
=3\pi$.

(7)$\int_0^4\frac{\sqrt x}{1+x\sqrt x}\mathrm dx\xlongequal{t=\sqrt x}\int_0^2\frac t{1+t^3}2t\mathrm dt=2\int_0^2\frac{t^2}{1+t^3}\mathrm dt=\frac23\int_0^2\frac{\mathrm dt^3}{1+t^3}=\frac23\ln|1+t^3|\Big|_0^2=\frac43\ln3$.

(8)$\int_0^{\frac\pi4}\frac{\mathrm dx}{1+\cos^2x}=\int_0^{\frac\pi4}\frac{\mathrm dx}{\sin^2x+2\cos^2x}=\int_0^{\frac\pi4}\frac{\sec^2x\mathrm dx}{\tan^2x+2}=\int_0^{\frac\pi4}\frac{\mathrm d\tan x}{\tan^2x+2}=\frac1{\sqrt2}\arctan(\frac1{\sqrt2}\tan x)\Big|_0^{\frac\pi4}\\
=\frac{\sqrt2}2\arctan\frac{\sqrt2}2$.

(9)$\int_0^{\ln 2}\sqrt{\mathrm e^x-1}\mathrm dx\xlongequal{t=\sqrt{\mathrm e^x-1}}\int_0^1t\frac{2t}{t^2+1}\mathrm dt=2\int_0^1(1-\frac1{t^2+1})\mathrm dt=2(t-\arctan t)\Big|_0^1=2-\frac\pi2$.

(10)$\int_{\sqrt2}^2\frac{\mathrm dx}{x\sqrt{x^2-1}}\xlongequal{x=\sec t}\int_{\frac\pi4}^{\frac\pi3}\frac{\tan t\sec t\mathrm dt}{\sec t\tan t}=t\Big|_{\frac\pi4}^{\frac\pi3}=\frac\pi{12}$.

(11)$\int_0^1\ln(1+x^2)\mathrm dx=x\ln(1+x^2)\Big|_0^1-\int_0^1x\mathrm d\ln(1+x^2)=\ln2-\int_0^1\frac{2x^2}{1+x^2}\mathrm dx\\
=\ln2-2\int_0^1(1-\frac1{1+x^2})\mathrm dx=\ln2-2(x-\arctan x)\Big|_0^1=\ln2-2(1-\frac\pi4)=\ln2-2+\frac\pi2$.

(12)$\int_1^{\mathrm e}x(\ln x)^2\mathrm dx=\frac12\int_1^{\mathrm e}(\ln x)^2\mathrm d(x^2)=\frac12[x^2(\ln x)^2\Big|_1^{\mathrm e}-\int_1^{\mathrm e}x^2\mathrm d(\ln x)^2]=\frac12[\mathrm e^2-\int_1^{\mathrm e}x2\ln x\mathrm dx]\\
=\frac12\mathrm e^2-\frac12\int_1^{\mathrm e}\ln x\mathrm dx^2=\frac12\mathrm e^2-\frac12(x^2\ln x\Big|_1^{\mathrm e}-\int_1^{\mathrm e}x^2\frac1x\mathrm dx)=\frac12\int_1^{\mathrm e}x\mathrm dx=\frac14x^2\Big|_1^{\mathrm e}=\frac14(\mathrm e^2-1)$.

(13)$\int_0^4\cos(\sqrt x-1)\mathrm dx\xlongequal{t=\sqrt x-1}\int_{-1}^1(\cos t)2(t+1)\mathrm dt=2\int_{-1}^1(t+1)\mathrm d\sin t\\
=2[(t+1)\sin t\Big|_{-1}^1-\int_{-1}^1\sin t\mathrm dt]=4\sin1+\cos t\Big|_{-1}^1=4\sin1$.

(14)$\int_0^1x\arctan x\mathrm dx=\frac12\int_0^1\arctan x\mathrm dx^2=\frac12x^2\arctan x\Big|_0^1-\frac12\int_0^1x^2\frac1{1+x^2}\mathrm dx\\
=\frac\pi8-\frac12\int_0^1(1-\frac1{1+x^2})\mathrm dx=\frac\pi8-\frac12(x-\arctan x)\Big|_0^1=\frac\pi4-\frac12$.

(15)$\int_0^1\arcsin x\mathrm dx=x\arcsin x\Big|_0^1-\int_0^1x\mathrm d\arcsin x=\frac\pi2-\int_0^1x\frac1{\sqrt{1-x^2}}\mathrm dx=\frac\pi2+\frac12\int_0^1\frac{\mathrm d(1-x^2)}{\sqrt{1-x^2}}\\
=\frac\pi2+\frac122\sqrt{1-x^2}\Big|_0^1=\frac\pi2-1$.

(16)$\int_0^1\mathrm e^{\sqrt x}\mathrm dx\xlongequal{t=\sqrt x}\int_0^1\mathrm e^t2t\mathrm dt=2\int_0^1t\mathrm d\mathrm e^t=2(t\mathrm e^t\Big|_0^1-\int_0^1\mathrm e^t\mathrm dt)=2\mathrm e-2\mathrm e^t\Big|_0^1=2$.

(17)$\int_0^{\sqrt{\ln2}}x^3\mathrm e^{-x^2}\mathrm dx=\frac12\int_0^{\sqrt{\ln2}}x^2\mathrm e^{-x^2}\mathrm dx^2\xlongequal{t=x^2}\frac12\int_0^{\ln2}t\mathrm e^{-t}\mathrm dt=\frac12(-t\mathrm e^{-t}\Big|_0^{\ln2}+\int_0^{\ln2}\mathrm e^{-t}\mathrm dt)\\
=-\frac14\ln2+\frac12(-\mathrm e^{-t})\Big|_0^{\ln2}=-\frac14\ln2+\frac14$.

(18)$\int_0^1\frac{x\mathrm e^x}{(1+x)^2}\mathrm dx=\int_0^1x\mathrm e^x\mathrm d(-\frac1{1+x})=-\frac{x\mathrm e^x}{1+x}\Big|_0^1+\int_0^1\frac1{1+x}(\mathrm e^xx+\mathrm e^x)\mathrm dx=-\frac{\mathrm e}2+\mathrm e^x\Big|_0^1=\frac{\mathrm e}2-1$.

(19)$\int_1^{\mathrm e}\cos(\ln x)\mathrm dx=x\cos(\ln x)\Big|_1^{\mathrm e}+\int_1^{\mathrm e}\sin(\ln x)\mathrm dx=\mathrm e\cos1-1+[x\sin(\ln x)\Big|_1^{\mathrm e}-\int_1^{\mathrm e}\cos(\ln x)\mathrm dx]\\
=\mathrm e\cos1-1+\mathrm e\sin1-\int_1^{\mathrm e}\cos(\ln x)\mathrm dx=\frac12(\mathrm e\cos1+\mathrm e\sin1-1)$.

(20)$\int_1^2\frac{x\mathrm e^x}{(\mathrm e^x-1)^2}\mathrm dx=\int_1^2\frac{x}{(\mathrm e^x-1)^2}\mathrm d(\mathrm e^x-1)=\int_1^2x\mathrm d(-\frac1{\mathrm e^x-1})=-\frac x{\mathrm e^x-1}\Big|_1^2+\int_1^2\frac1{\mathrm e^x-1}\mathrm dx\\
=-\frac2{\mathrm e^2-1}+\frac1{\mathrm e-1}+\int_1^2(-1+\frac{\mathrm e^x}{\mathrm e^x-1})\mathrm dx=-\frac2{\mathrm e^2-1}+\frac1{\mathrm e-1}+[-x+\ln(\mathrm e^x-1)]\Big|_1^2\\
=\frac1{\mathrm e+1}-1+\ln(\mathrm e^2-1)-\ln(\mathrm e-1)=\frac1{\mathrm e+1}-1+\ln(\mathrm e+1)$.

\item设$f(x)$在$[0,1]$上连续,证明:
\newline
(1)$\int_0^{\frac\pi2}f(\sin x)\mathrm dx=\int_0^{\frac\pi2}f(\cos x)\mathrm dx$;
\newline
(2)$\int_0^\pi xf(\sin x)\mathrm dx=\frac\pi2\int_0^\pi f(\sin x)\mathrm dx$.

证明:(1)$\int_0^{\frac\pi2}f(\sin x)\mathrm dx\xlongequal{x=\frac\pi2-t}\int_{\frac\pi2}^0f(\sin(\frac\pi2-t))\mathrm d(\frac\pi2-t)=-\int_{\frac\pi2}^0f(\cos t)\mathrm dt=\int_0^{\frac\pi2}f(\cos x)\mathrm dx$.

(2)$\int_0^\pi xf(\sin x)\mathrm dx\xlongequal{x=\pi-t}\int_\pi^0(\pi-t)f(\sin(\pi-t))\mathrm d(\pi-t)=-\int_\pi^0\pi f(\sin t)\mathrm dt+\int_\pi^0tf(\sin t)\mathrm dt\\
=\pi\int_0^\pi f(\sin x)\mathrm dx-\int_0^\pi xf(\sin x)\mathrm dx$

$\therefore\int_0^\pi xf(\sin x)\mathrm dx=\frac\pi2\int_0^\pi f(\sin t)\mathrm dt$.

\item证明:
\newline
(1)连续奇函数的一切原函数都是偶函数;
\newline
(2)连续偶函数的原函数中有一个是奇函数.

证明:(1)设$f(x)$为一个连续奇函数,则$f(x)$的任意原函数可表示为$F(x)=\int_0^x f(t)\mathrm dt+C$

$F(-x)=\int_0^{-x}f(t)\mathrm dt+C\xlongequal{u=-t}-\int_0^{x}f(-u)\mathrm du+C=\int_0^xf(u)\mathrm du+C=F(x)$

故$f(x)$的任意原函数$F(x)$均为偶函数.

(2)设$f(x)$为一个连续偶函数,则$f(x)$的任意原函数可表示为$F(x)=\int_0^x f(t)\mathrm dt+C$

$F(-x)=\int_0^{-x}f(t)\mathrm dt+C\xlongequal{u=-t}-\int_0^{x}f(-u)\mathrm du+C=-\int_0^xf(u)\mathrm du+C=-F(x)+2C$

当且仅当$C=0$时$F(x)$是奇函数.

\item设$f(x)$连续,证明:
\[
\int_0^Rx^3f(x^2)\mathrm dx=\frac12\int_0^{R^2}xf(x)\mathrm dx.
\]

证明:$\int_0^Rx^3f(x^2)\mathrm dx=\int_0^Rx^2f(x^2)\mathrm d(\frac12x^2)\xlongequal{t=x^2}\frac12\int_0^{R^2}tf(t)\mathrm dt=\frac12\int_0^{R^2}xf(x)\mathrm dx$.
\end{enumerate}
%\subsection{习题7.5解答}
%\begin{enumerate}
%\item求下列曲线所围图形的面积:
%\newline
%(1)$y=x^2$与$y=2-x^2$;
%\newline
%(2)$y=|\ln x|,y=0,x=\frac1{10}$及$x=10$;
%\newline
%(3)蚌线$\rho=a\cos\theta+b,b\geq a>0$;
%\newline
%(4)$\frac{x^2}{a^2}+\frac{y^2}{b^2}=1,a>0,b>0$;
%\newline
%(5)$\sqrt{\frac xa}+\sqrt{\frac yb}=1,a,b>0,x=0,y=0$.
%
%解:(1)$S=\int_{-1}^1(2-x^2-x^2)\mathrm dx=\int_{-1}^1(2-2x^2)\mathrm dx=2x-\frac23x^3\Big|_{-1}^1=4-\frac43=\frac83$.
%
%(2)$S=\int_{\frac1{10}}^{10}|\ln x|\mathrm dx=-\int_{\frac1{10}}^1\ln x\mathrm dx+\int_1^{10}\ln x\mathrm dx=-(x\ln x-x)\Big|_{\frac1{10}}^1+(x\ln x-x)\Big|_1^{10}\\
%=1+(\frac1{10}\ln\frac1{10}-\frac1{10})+10\ln10-10-(-1)=-\frac{81}{10}+\frac{99}{10}\ln10$.
%
%(3)$S=\int_0^{2\pi}\frac12\rho^2\mathrm d\theta=\int_0^{2\pi}\frac12(a\cos\theta+b)^2\mathrm d\theta=\frac12\int_0^{2\pi}(a^2\cos^2\theta+2ab\cos\theta+b^2)\mathrm d\theta\\
%=\frac12\int_0^{2\pi}[\frac{a^2}2(1+\cos2\theta)+2ab\cos\theta+b^2]\mathrm d\theta=\frac12[\frac12a^2(\theta+\frac12\sin2\theta)+2ab\sin\theta+b^2\theta]_0^{2\pi}\\
%=\frac12[\frac12a^2(2\pi+0)+b^22\pi]=\frac\pi2a^2+\pi b^2$.
%
%(4)$S=4\int_0^ab\sqrt{1-\frac{x^2}{a^2}}\mathrm dx=4\frac ba\int_0^a\sqrt{a^2-x^2}\mathrm dx=4\frac ba\int_0^{\frac\pi2}(a\cos t)(a\cos t)\mathrm dt\\
%=4\frac ba\int_0^{\frac\pi2}a^2\frac{1+\cos2t}2\mathrm dt=2ab\int_0^{\frac\pi2}(1+\cos2t)\mathrm dt=2ab(t+\frac12\sin2t)\Big|_0^\frac\pi2=\pi ab$.
%
%(5)$S=$
%\end{enumerate}
\end{document}