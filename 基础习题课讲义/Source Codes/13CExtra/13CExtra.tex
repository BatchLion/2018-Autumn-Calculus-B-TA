\documentclass[12pt,UTF8]{ctexart}
\usepackage{ctex,amsmath,amssymb,geometry,fancyhdr,bm,amsfonts
,mathtools,extarrows,graphicx,url,enumerate,color,float,multicol,,subfig} 
\allowdisplaybreaks[4]
% 加入中文支持
\newcommand\Set[2]{\left\{#1\ \middle\vert\ #2 \right\}}
\newcommand\Lim[0]{\lim\limits_{n\rightarrow\infty}}
\newcommand\LIM[2]{\lim\limits_{#1\rightarrow#2}}
\newcommand\Ser[1]{\sum_{n=#1}^\infty}
\newcommand{\SER}[2]{\sum_{#1=#2}^\infty}
\newcommand{\varSer}[3]{\sum_{#1=#2}^{#3}}
\newcommand{\me}[0]{\mathrm e}
\geometry{a4paper,scale=0.80}
\pagestyle{fancy}
\rhead{第8章补充题}
\lhead{基础习题课讲义}
\chead{微积分B(1)}
\begin{document}
\def\thesection{13C}
\section{第8章补充题}
\def\thesubsection{\thesection.\arabic{subsection}}
\subsection{第8章补充题解答}
\begin{enumerate}
\item讨论下列级数的收敛性.

\begin{tabular}{ll}
(1)$\Ser{1}\frac{n!}{n^n}a^n(a>0)$;& (2)$\Ser{2}\frac2{2^{\ln n}}$;\\
(3)$\Ser{2}\frac1{(\ln n)^{\ln n}}$;& (4)$\Ser{1}\frac{\sin^2(\pi\sqrt{n^2+n})}n$;\\
\multicolumn{2}{l}{(5)$\Ser{2}\ln(1+\frac{(-1)^n}{n^p})(p>0)$;}\\
\multicolumn{2}{l}{(6)$\Ser{1}(-1)^n(\mathrm e^{\frac1{\sqrt n}}-1-\frac1{\sqrt n})$;}\\
\multicolumn{2}{l}{(7)$\Ser{2}\sin(n\pi+\frac1{\ln n})$.}
\end{tabular}

解:(1)$\Lim\frac{u_{n+1}}{u_n}=\Lim\frac{(n+1)!a^{n+1}}{(n+1)^{n+1}}\frac{n^n}{n!a^n}=\Lim a\frac1{(1+\frac1n)^n}=\frac a{\mathrm e}$

$\therefore$当$a<\mathrm e$时级数收敛,当$a>\mathrm e$时级数发散;

当$a=\mathrm e$时,由函数$f(x)=(1+\frac1x)^x$在$x\geq1$时单调增加(见教材P160例5.1.3)且$\Lim(1+\frac1x)^x=\mathrm e$可知$(1+\frac1n)^n<\mathrm e$

$\therefore\frac{u_{n+1}}{u_n}=\mathrm e\frac1{(1+\frac1n)^n}>1$

$\therefore u_{n+1}>u_n>\cdots>u_2>u_1=\mathrm e,\Lim u_n=\Lim\frac{n!a^n}{n^n}\neq0$,级数发散.

(2)$\Ser{2}\frac2{2^{\ln n}}=\Ser{2}\frac2{\me^{\ln2\ln n}}=\Ser{2}\frac2{n^{\ln2}}$

$\because0<\ln2<1$

$\therefore$级数发散.

(3)$\Ser{2}\frac1{(\ln n)^{\ln n}}=\Ser{2}\frac1{\me^{\ln(\ln n)\ln n}}=\Ser{2}\frac1{n^{\ln(\ln n)}}$

$\because\Lim n^2\cdot\frac1{n^{\ln(\ln n)}}=\Lim\frac1{n^{\ln(\ln n)-2}}=0$,

$\therefore$该级数收敛.

(4)$\because\sin(\pi\sqrt{n^2+n})=\sin[(\pi\sqrt{n^2+n}-\pi n)+\pi n]=(-1)^n\sin(\pi\sqrt{n^2+n}-\pi n)\\
=(-1)^n\sin\frac{n\pi}{\sqrt{n^2+n}+n}=(-1)^n\sin\frac{\pi}{\sqrt{1+\frac1n}+1}$

$\therefore\sin^2(\pi\sqrt{n^2+n})=\sin^2\frac{\pi}{\sqrt{1+\frac1n}+1}$

$\because2<\sqrt{1+\frac1n}+1\leq1+\sqrt2$

$\therefore1>\sin^2\frac{\pi}{\sqrt{1+\frac1n}+1}\geq\sin^2\frac{\pi}{1+\sqrt2}$

$\therefore\frac{\sin^2(\pi\sqrt{n^2+n})}n=\frac{\sin^2\frac{\pi}{\sqrt{1+\frac1n}+1}}n\geq\frac{\sin^2\frac{\pi}{1+\sqrt2}}n$

$\because\Ser1\frac{\sin^2\frac{\pi}{1+\sqrt2}}n$发散

$\therefore\frac{\sin^2(\pi\sqrt{n^2+n})}n$发散.

(5)$\because\Lim n^p|\ln(1+\frac{(-1)^n}{n^p})|=\Lim|n^p\frac{(-1)^n}{n^p}|=1$

$\therefore$当$p>1$时,该级数绝对收敛

$\ln(1+\frac{(-1)^n}{n^p})=\frac{(-1)^n}{n^p}-\frac1{2n^{2p}}+o(\frac1{n^{2p}})$

当$\frac12<p\leq1$时,级数$\Ser2\frac{(-1)^n}{n^p}$条件收敛

$\because\Lim n^{2p}|-\frac1{2n^{2p}}+o(\frac1{n^{2p}})|=\frac12$

$\therefore$级数$\Ser2[-\frac1{2n^{2p}}+o(\frac1{n^{2p}})]$绝对收敛

$\because|\frac{(-1)^n}{n^p}-\frac1{2n^{2p}}+o(\frac1{n^{2p}})|\geq\frac1{n^p}-|\frac1{n^{2p}}+o(\frac1{n^{2p}})|$

$\because\Ser2[\frac1{n^p}-|\frac1{n^{2p}}+o(\frac1{n^{2p}})|]=+\infty$

$\therefore\Ser2|\ln(1+\frac{(-1)^n}{n^p})|=\Ser2|\frac{(-1)^n}{n^p}-\frac1{2n^{2p}}+o(\frac1{n^{2p}})|$发散

$\therefore$级数$\Ser2\ln(1+\frac{(-1)^n}{n^p})$条件收敛

当$0<p\leq\frac12$时,级数$\Ser2\frac{(-1)^n}{n^p}$条件收敛

$\because\Lim n^{2p}|-\frac1{2n^{2p}}+o(\frac1{n^{2p}})|=\frac12$

$\therefore$级数$\Ser2[-\frac1{2n^{2p}}+o(\frac1{n^{2p}})]$发散

$\therefore$级数$\Ser2\ln(1+\frac{(-1)^n}{n^p})=\Ser2[\frac{(-1)^n}{n^p}-\frac1{2n^{2p}}+o(\frac1{n^{2p}})]$发散

综上所述,当$p>1$时原级数绝对收敛,当$\frac12<p\leq1$时原级数条件收敛,当$0<p\leq\frac12$时原级数发散.
%当$0<p\leq1$时,
%
%$s_{2n+1}=\varSer k2{2n+1}\ln(1+\frac{(-1)^n}{n^p})=\varSer k1n\ln(1+\frac1{(2k)^p})+\varSer k1n\ln(1-\frac1{(2k+1)^p})$
%
%$\because$级数$\varSer k1\infty\ln(1+\frac1{(2k)^p})$和$\varSer k1\infty\ln(1-\frac1{(2k+1)^p})$均收敛
%
%$\therefore\{s_{2n+1}\}$收敛
%
%$\because\Lim\ln(1+\frac{(-1)^{2n+2}}{(2n+2)^p})=0$
%
%$\therefore\{s_{2n+2}\}$也收敛,且$\Lim s_{2n+2}=\Lim s_{2n+1}$
%
%$\therefore$级数$\Ser{2}\ln(1+\frac{(-1)^n}{n^p})$收敛


%由
%\[\begin{aligned}
%&|\ln[1+\frac1{(2n)^p}]|-|\ln[1-\frac1{(2n+1)^p}]|\\
%=&\ln[1+\frac1{(2n)^p}]+\ln[1-\frac1{(2n+1)^p}]\\
%=&\ln[1+\frac1{(2n)^p}-\frac1{(2n+1)^p}-\frac1{(2n)^p(2n+1)^p}]\\
%=&\ln[1+\frac{(2n+1)^p-(2n)^p}{(2n)^p(2n+1)^p}-\frac1{(2n)^p(2n+1)^p}],n\geq2
%\end{aligned}\]
%$\because$
%\[(2n+1)^p-(2n)^p=p\xi^{p-1}(2n+1-2n)=\frac p{\xi^{1-p}}<\frac p{(2n)^{1-p}}\leq\frac p{2^{1-p}}\leq p<1,\ 2n<\xi<2n+1\]
%
%$\therefore$
%\[
%|\ln[1+\frac1{(2n)^p}]|-|\ln[1-\frac1{(2n+1)^p}]|=\ln[1+\frac{(2n+1)^p-(2n)^p}{(2n)^p(2n+1)^p}-\frac1{(2n)^p(2n+1)^p}]<0
%\]
%即
%\[
%|\ln[1+\frac1{(2n)^p}]<|\ln[1-\frac1{(2n+1)^p}]|
%\]
%$\because$
%\[\begin{aligned}
%&|\ln[1-\frac1{(2n+1)^p}]|-|\ln[1+\frac1{(2n+2)^p}]|\\
%=&-\ln[1-\frac1{(2n+1)^p}]-\ln[1+\frac1{(2n+2)^p}]\\
%=&-\ln[1-\frac1{(2n+1)^p}+\frac1{(2n+2)^p}-\frac1{(2n+1)^p(2n+2)^p}]\\
%=&-\ln[1-\frac{(2n+2)^p-(2n+1)^p}{(2n+1)^p(2n+2)^p}-\frac1{(2n+1)^p(2n+2)^p}]>0
%\end{aligned}\]
%$\therefore$
%\[|\ln[1-\frac1{(2n+1)^p}]|>|\ln[1+\frac1{(2n+2)^p}]|\]

(6)令$f(x)=\me^x-1-x$, 则当$0<x\leq1$时$f'(x)=\me^x-1>0$

$\therefore|u_{n+1}|=f(\frac1{\sqrt{n+1}})<f(\frac1{\sqrt n})=|u_n|$

又$\because\Lim(\mathrm e^{\frac1{\sqrt n}}-1-\frac1{\sqrt n})=1-1-0=0$

$\therefore\Ser{1}(-1)^n(\mathrm e^{\frac1{\sqrt n}}-1-\frac1{\sqrt n})$是莱布尼茨交错级数

$\therefore\Ser{1}(-1)^n(\mathrm e^{\frac1{\sqrt n}}-1-\frac1{\sqrt n})$收敛

$\because\Lim n\cdot|u_n|=\Lim n\cdot|\me^{\frac1{\sqrt n}}-1-\frac1{\sqrt n}|=\Lim n\cdot|1+\frac1{\sqrt n}+\frac1{2n}+o(\frac1n)-1-\frac1{\sqrt n}|\\
=\Lim n\cdot|1+\frac1{\sqrt n}+\frac1{2n}+o(\frac1n)-1-\frac1{\sqrt n}|=\frac12$

$\therefore\Ser1|u_n|$发散

$\therefore$级数$\Ser{1}(-1)^n(\mathrm e^{\frac1{\sqrt n}}-1-\frac1{\sqrt n})$条件收敛.

(7)$\sin(n\pi+\frac1{\ln n})=\sin n\pi\sin\frac1{\ln n}+\cos n\pi\sin\frac1{\ln}=(-1)^n\sin\frac1{\ln n}$

$\because\frac1{\ln 2}=1.4427<\frac\pi2$

$\therefore$当$n>2$时$\sin\frac1{\ln n}>\sin\frac1{\ln(n+1)}$

又$\because\Lim\sin\frac1{\ln n}=0$

$\therefore$级数$\Ser{2}\sin(n\pi+\frac1{\ln n})$是莱布尼茨交错级数,故收敛

$\because\Lim n\cdot|\sin(n\pi+\frac1{\ln n})|=\Lim n\cdot|\sin\frac1{\ln n}|=\Lim n\cdot\frac1{\ln n}=+\infty$

$\therefore\Ser{2}|u_n|$发散

$\therefore$级数$\Ser{2}\sin(n\pi+\frac1{\ln n})$条件收敛.
\item设$a_n\geq0$且$\Ser{1}a_n$发散,讨论下列级数的收敛性:

\begin{tabular}{lll}
(1)$\Ser{1}\frac{a_n}{1+a_n}$;& (2)$\Ser{1}\frac{a_n}{1+n^2a_n}$;& (3)$\Ser{1}\frac{a_n}{1+a_n^2}$.
\end{tabular}

解:(1)假设$\Ser{1}\frac{a_n}{1+a_n}$收敛,则$\Lim\frac{a_n}{1+a_n}=\Lim\frac1{\frac1{a_n}+1}=0$

$\therefore\Lim a_n=0$

$\therefore\Lim\frac1{a_n}\cdot\frac{a_n}{1+a_n}=1$

$\because$级数$\Ser{1}a_n$发散

$\therefore$级数$\Ser{1}\frac{a_n}{1+a_n}$发散,与假设矛盾

$\therefore$级数$\Ser{1}\frac{a_n}{1+a_n}$发散.

(2)$\because\Lim n^2\cdot\frac{a_n}{1+n^2a_n}=\Lim\frac{a_n}{\frac1{n^2}+a_n}=1$

$\therefore$级数$\Ser{1}\frac{a_n}{1+n^2a_n}$收敛.

(3)当$a_n=n$时,$\Ser{1}\frac{a_n}{1+a_n^2}=\Ser{1}\frac n{1+n^2}$,由$\Lim n\cdot\frac n{1+n^2}=1$知该级数发散

当$a_n=n^2$时,$\Ser{1}\frac{a_n}{1+a_n^2}=\Ser{1}\frac{n^2}{1+n^4}$,由$\Lim n^2\cdot\frac{n^2}{1+n^4}=1$知该级数收敛

$\therefore$该级数可能收敛也可能发散.
\item设$a>0,b_n=\frac{a^{\frac{n(n+1)}2}}{[(1+a)(1+a^2)\cdots(1+a^n)]}$,讨论级数$\Ser{1}b_n$的收敛性.

解:$b_n=\frac{a^{1+2+\cdots+n}}{(1+a)(1+a^2)\cdots(1+a^n)}=\frac a{1+a}\cdot\frac{a^2}{1+a^2}\cdot\cdots\cdot\frac{a^n}{1+a^n}$

当$a=1$时$b_n=\frac1{2^n}$, 级数$\Ser{1}b_n$收敛

$\because f(x)=\frac x{1+x}=1-\frac1{1+x}$单调增加

$\therefore$当$0<a<1$时$b_n=\frac a{1+a}\cdot\frac{a^2}{1+a^2}\cdot\cdots\cdot\frac{a^n}{1+a^n}<\frac a{1+a}\cdot\frac a{1+a}\cdot\cdots\cdot\frac a{1+a}=(\frac a{1+a})^n$

$\because$级数$\Ser1(\frac a{1+a})^n$收敛

【或者:$b_n=\frac{a^{1+2+\cdots+n}}{(1+a)(1+a^2)\cdots(1+a^n)}<a^{1+2+\cdots+n}=a^1\cdot a^2\cdot\cdots\cdot a^n<a^n$, 级数$\Ser1a^n$收敛】

$\therefore$级数$\Ser{1}b_n$收敛

当$a>1$时$b_n=\frac a{1+a}\cdot\frac{a^2}{1+a^2}\cdot\cdots\cdot\frac{a^n}{1+a^n}=\frac1{(1+\frac1a)(1+\frac1{a^2})\cdots(1+\frac1{a^n})}=\frac1{\me^{\ln[(1+\frac1a)(1+\frac1{a^2})\cdots(1+\frac1{a^n})]}}\\
=\frac1{\me^{\ln(1+\frac1a)+\ln(1+\frac1{a^2})+\cdots+\ln(1+\frac1{a^n})}}>\frac1{\me^{\frac1a+\frac1{a^2}+\cdots+\frac1{a^n}}}=\frac1{\me^{\frac{\frac1a(1-\frac1{a^n})}{1-\frac1a}}}>\frac1{\me^{\frac1{a-1}}}$

$\therefore\Lim b_n\neq0$, 故级数$\Ser{1}b_n$发散

综上所述,当$0<a\leq1$时,级数$\Ser{1}b_n$收敛;当$a>1$时,级数$\Ser{1}b_n$发散.

\item设$a_n>0$且$\Lim\frac{\ln\frac1{a_n}}{\ln n}=l>1$,求证$\Ser{1}a_n$收敛.

证明:$\because\Lim\frac{\ln\frac1{a_n}}{\ln n}=\Lim\frac{\ln a_n}{\ln\frac1n}=l>1$

$\therefore$根据数列极限的保号性知$\exists\varepsilon>0$使得$|\frac{\ln a_n}{\ln\frac1n}-l|<\varepsilon$即$\frac{\ln a_n}{\ln\frac1n}>l-\varepsilon$, $\varepsilon$应满足$l-\varepsilon>1$

$\therefore$当$n\geq2$时$\ln a_n<(l-\varepsilon)\ln\frac1n=\ln\frac1{n^{l-\varepsilon}}$

$\therefore a_n<\frac1{n^{l-\varepsilon}}(n\geq2)$

$\because$当$l-\varepsilon>1$时$\Ser2\frac1{n^{l-\varepsilon}}$收敛

$\therefore$$\Ser{1}a_n$收敛.
\item已知$\Lim(n^{2n\sin\frac1n}\cdot a_n)=1$,试讨论$\Ser{1}a_n$的收敛性.

解:$\because\Lim(n^{2n\sin\frac1n}\cdot a_n)=\Lim(n^{2n\sin\frac1n-1.5}n^{1.5}\cdot a_n)=\Lim[\me^{(2n\sin\frac1n-1.5)\ln n}n^{1.5}\cdot a_n]\\
=\Lim\{\me^{[2n(\frac1n+o(\frac1{n^2}))-1.5]\ln n}n^{1.5}\cdot a_n\}=\Lim\{\me^{(2+o(\frac1{n})-1.5)\ln n}n^{1.5}\cdot a_n\}\\
=\Lim\{\me^{(0.5+o(\frac1{n}))\ln n}n^{1.5}\cdot a_n\}=1$

$\because\Lim\me^{(0.5+o(\frac1{n}))\ln n}=+\infty$

$\therefore\Lim n^{1.5}\cdot a_n=\Lim\frac1{\me^{(0.5+o(\frac1{n}))\ln n}}=0$

$\therefore$级数$\Ser1{a_n}$收敛.

\item设$p>0,\Lim[n^p(\mathrm e^{\frac1n}-1)a_n]=1$,试讨论$\Ser{1}a_n$的收敛性.

解:$\because\Lim n^p(\mathrm e^{\frac1n}-1)a_n=\Lim n^{p-1}\frac{\mathrm e^{\frac1n}-1}{\frac1n}a_n=\Lim n^{p-1}a_n=1$

$\therefore$当$p-1>1$即$p>2$时,级数收敛;

当$0<p-1\leq1$即$1<p\leq2$时,级数发散;

当$p=1$时$p-1=0,\ \Lim a_n=1$,级数发散;

当$0<p<1$时$p-1<0,\ \Lim a_n=+\infty$,级数发散.

\item设$a_n>0,\Ser{1}a_n$发散,令$S_k=a_1+a_2+\cdots+a_k$,试证$\Ser{1}\frac{a_n}{S_n}$也发散.

证明:$\because a_n>0,\Ser{1}a_n$发散

$\therefore\Lim S_n=+\infty$

$\therefore\varSer k{n+1}{n+p}\frac{a_k}{S_k}\geq\frac{a_{n+1}+a_{n+2}+\cdots+a_{n+p}}{S_{n+p}}=\frac{S_{n+p}-S_{n+1}}{S_{n+p}}=1-\frac{S_{n+1}}{S_{n+p}}\rightarrow1,p\rightarrow\infty$

$\therefore\exists P>0,s.t.$当$p>P$时$\varSer k{n+1}{n+p}\frac{a_k}{S_k}\geq\frac{a_{n+1}+a_{n+2}+\cdots+a_{n+p}}{S_{n+p}}>1-\frac12=\frac12$

$\therefore$对于$\varepsilon_0=\frac12, \forall N>0$,如取$n=N+1,p=P+1$,则
\[|\varSer k{n+1}{n+p}\frac{a_k}{S_k}|\geq\frac{a_{n+1}+a_{n+2}+\cdots+a_{n+p}}{S_{n+p}}>\frac12=\varepsilon_0\]

$\therefore$级数$\Ser{1}\frac{a_n}{S_n}$发散.

\item设$\varphi(x)$在$(-\infty,+\infty)$上连续,周期为$1$,且$\int_0^1\varphi(x)\mathrm dx=0,f(x)$在$[0,1]$上连续可导,令$a_n=\int_0^1f(x)\varphi(nx)\mathrm dx$,求证级数$\Ser{1}a_n^2$收敛.

证明:令$G(x)=\int_0^x\varphi(nx)\mathrm dx$, 则$G(1)=0$

$\therefore$
\[
a_n=\int_0^1f(x)\varphi(nx)\mathrm dx=f(x)G(x)\big|_0^1-\int_0^1G(x)f'(x)\mathrm dx=-\int_0^1G(x)f'(x)\mathrm dx
\]
$\because f(x)$在$[0,1]$上连续可导,$\varphi(x)$在$(-\infty,+\infty)$上连续

$\therefore\exists M_1=\max|f'(x)|,\exists M_2=\max|\varphi(x)|$

$\therefore$
\[
|a_n|\leq\int_0^1|G(x)||f(x)|\mathrm dx\leq M_1\int_0^1|G(x)|\mathrm dx
\]

$\because\int_0^1\varphi(x)\mathrm dx=0$

$\therefore$
\[
|G(x)|=|\int_0^x\varphi(nt)\mathrm dt|=\frac1n|\int_0^{nx}\varphi(u)\mathrm du|=\frac1n|\int_{[nx]}^{nx}\varphi(u)\mathrm du|\leq\frac1nM_2
\]
$\therefore a_n\leq\frac1nM_1M_2,\ a_n^2\leq\frac1{n^2}(M_1M_2)^2$

$\therefore$级数$\Ser{1}a_n^2$收敛.
\item确定下列函数级数的收敛域:

\begin{tabular}{ll}
(1)$\Ser{1}\frac{x^{n^2}}{2^n}$;& (2)$\Ser{1}\frac n{x^n}$;\\
(3)$\Ser{1}n!(\frac xn)^n$;& (4)$\Ser{1}(1-\cos\frac xn)$;\\
(5)$\Ser{1}\sin\frac1{n^{\frac{x+1}x}}$;& (6)$\Ser{1}\frac{x^n}{1+x^{2n}}$.
\end{tabular}

解:(1)$\Lim\sqrt[n]{|\frac{x^{n^2}}{2^n}|}=\Lim\frac{|x|^n}2=\begin{cases}
0,&|x|<1\\
\frac12,&x=1\\
\frac12,&x=-1\\
\infty,&|x|>1
\end{cases}$

$\therefore$收敛域为$[-1,1]$.

(2)$\Lim\sqrt[n]{|\frac n{x^n}|}=\Lim\frac{\sqrt[n]n}{|x|}=\frac1{|x|}$

$\therefore$当$|x|>1$时级数绝对收敛;当$|x|<1$时级数发散;当$|x|=1$时,$\Lim\frac n{x^2}\neq0$, 故发散

$\therefore$级数的收敛域为$(-\infty,-1)\cup(1,+\infty)$.

(3)$\Lim\frac{|u_{n+1}|}{|u_n|}=\Lim\frac{(n+1)!(|\frac x{n+1}|)^{n+1}}{n!(|\frac xn|)^n}=\Lim(\frac n{n+1})^n|x|=\Lim\frac 1{(1+\frac1n)^n}|x|=\frac{|x|}\me$

$\therefore$当$|x|<\me$时级数绝对收敛;当$|x|>\me$时级数发散

当$|x|=\me$时,$\frac{|u_{n+1}|}{|u_n|}=\frac\me{(1+\frac1n)^n}$

由函数$f(x)=(1+\frac1x)^x$在$x\geq1$时单调增加(见教材P160例5.1.3)且$\Lim(1+\frac1x)^x=\mathrm e$

$\therefore\frac{|u_{n+1}|}{|u_n|}>1$, $|u_{n+1}|>|u_n|>\cdots>|u_1|$

$\therefore\Lim{|u_n|}\neq0$, 级数发散

综上所述,该级数的收敛域为$(-\me,\me)$.

(4)$1-\cos\frac xn=2\sin^2\frac x{2n}$

$\because\Lim n^2\cdot(1-\cos\frac xn)=\Lim n^2\cdot2\sin^2\frac x{2n}=\Lim2(\frac{\sin\frac x{2n}}{\frac1n})^2=2(\frac x2)^2=\frac{x^2}2$

$\therefore$该级数的收敛域为$(-\infty,+\infty)$.

(5)$\Lim n^{\frac{1+1+\frac1x}2}\cdot\sin\frac1{n^{\frac{x+1}x}}=\Lim n^{1+\frac1{2x}}\cdot\frac1{n^{\frac{x+1}x}}=\Lim n^{-\frac1{2x}}=\begin{cases}
0,&x>0\\
+\infty,&x<0
\end{cases}$

$\because$当$x>0$时,级数$\Ser1\frac1{n^{\frac{1+1+\frac1x}2}}$收敛;当$x<0$时,级数$\Ser1\frac1{n^{\frac{1+1+\frac1x}2}}$发散

$\therefore$原级数的收敛域为$(0,+\infty)$.

(6)$\because\Lim\sqrt[n]{|\frac{x^n}{1+x^{2n}}|}=\Lim\frac{|x|}{\sqrt[n]{1+x^{2n}}}=\begin{cases}
|x|,&|x|<1\\
1,&|x|=1\\
0,&|x|>1
\end{cases}$

$\therefore$当$x\neq1$时,该级数收敛

当$x=1$时$\Ser{1}\frac{x^n}{1+x^{2n}}=\Ser1\frac12$,该级数发散

$\therefore$该级数的收敛域为$(-\infty,1)\cup(1+\infty)$.
\item讨论下列函数级数在指定区间上的一致收敛性:
\\
(1)$\Ser{1}n\mathrm e^{-nx},(0,+\infty)$与$[\delta,+\infty),\delta>0$为常数;\\
(2)$\Ser{1}\frac{x^2}{(1+x^2)^n},(0,+\infty)$与$[\delta,+\infty),\delta>0$;\\
(3)$\Ser{1}(-1)^n\frac1{x+n},[0,+\infty)$.

解:(1)在区间$(0,+\infty)$上:

$\forall N>0$, 取$n_0=N+1,p_0\geq1,0<x_0<\frac{\ln[p_0(n_0+1)]}{n_0+p_0}$, 则
\[
|\sum_{k=n_0+1}^{n_0+p_0}k\me^{-kx_0}|\geq|\sum_{k=n_0+1}^{n_0+p_0}(n_0+1)\me^{-(n_0+p_0)x_0}|=p_0(n_0+1)\me^{-(n_0+p_0)x_0}=1
\]
故级数$\Ser{1}n\mathrm e^{-nx}$在区间$(0,+\infty)$上不一致收敛.

在区间$[\delta,+\infty),\delta>0$上:

$\because\Lim n^2\cdot n\me^{-nx}=0$

$\therefore$级数$\Ser{1}n\mathrm e^{-nx}$收敛

$\because n\me^{-nx}<n\me^{-n\delta}$

又$\because\Lim n^2\cdot n\me^{-n\delta}=0$, 故级数$\Ser{1}n\mathrm e^{-n\delta}$在区间$[\delta,+\infty)$上收敛

$\therefore$级数$\Ser{1}n\mathrm e^{-nx}$在区间$[\delta,+\infty)$上一致收敛.

(2)该级数的部分和序列$S_n(x)=\varSer k1n\frac{x^2}{(1+x^2)^k}=\frac{x^2}{1+x^2}\frac{1-\frac1{(1+x^2)^n}}{1-\frac1{1+x^2}}=1-\frac1{(1+x^2)^n}$

和函数$S(x)=\Ser{1}\frac{x^2}{(1+x^2)^n}=\frac{x^2}{1+x^2}\frac1{1-\frac1{1+x^2}}=1$

当$x\in(0,+\infty)$时:

$\forall N>0$, 取$n_0=N+1,0<x_{n_0}<\sqrt{2^{\frac1{n_0}}-1}$, 则
\[
|S_{n_0}(x)-S(x)|=\frac1{(1+x_{n_0}^2)^{n_0}}>\frac12
\]

故该级数不一致收敛.

当$x\in[\delta,+\infty)$时:

$\forall\varepsilon>0$, 取$N=\max\{[\frac{\ln\frac1{\varepsilon}}{\ln(1+\delta^2)}]+1,1\}$, 当$n>N$时$n>\frac{\ln\frac1{\varepsilon}}{\ln(1+\delta^2)}$, 故
\[
|S_{n_0}(x)-S(x)|=\frac1{(1+x_n^2)^n}<\frac1{(1+\delta^2)^n}<\frac1{(1+\delta^2)^{\frac{\ln\frac1{\varepsilon}}{\ln(1+\delta^2)}}}<\frac1{\me^{\ln(1+\delta^2)\frac{\ln\frac1{\varepsilon}}{\ln(1+\delta^2)}}}<\varepsilon
\]
$\therefore$该级数一致收敛.

(3)$\forall x\in[0,+\infty)$, 级数$\Ser{1}(-1)^n\frac1{x+n}$是莱布尼茨交错级数,故收敛

$\because\forall\varepsilon>0$, 取$N=\max\{[\frac1\varepsilon-x-1]+1,1\}$, 则当$n>N$时,$\forall x\in[0,+\infty),\ \forall p\in\mathbb Z^+$,
\[|\varSer k{n+1}{n+p}(-1)^n\frac1{x+n}|\leq\frac1{x+n+1}<\varepsilon\]

$\therefore$该级数一致收敛.
\item设函数$f(x)$在$(-\infty,+\infty)$上有任意阶导数,记$f_n(x)=f^{(n)}(x)(n=1,2,\cdots)$,设函数序列$\{f_n(x)\}$在任意有限区间上一致收敛于某个函数$\varphi(x)$,求证:存在常数$c$,使$\varphi(x)=c\mathrm e^x$.

证明:$\because f(x)$在$(-\infty,+\infty)$上有任意阶导数

$\therefore f^{(n)}(x)$在任意区间$[a,b]$上连续可微

$\because\{f^{(n)}\}$在$[a,b]$内一致收敛于$\varphi(x)$

$\therefore$其导函数序列$\{f^{(n)}(x)\}$在$[a,b]$内也一致收敛于$\varphi(x)$

$\therefore$函数$\varphi(x)$在$[a,b]$上可导,且
\[
\varphi'(x)=[\Lim f^{(n)}(x)]'=\Lim f^{(n+1)(x)}=\varphi(x)
\]
即
\[\frac{\mathrm d\varphi(x)}{\mathrm dx}=\varphi(x)\]

$\therefore$
\[\frac{\mathrm d\varphi(x)}{\varphi(x)}=\mathrm dx\]

$\therefore\ln\varphi(x)=x+C$, 即$\varphi(x)=\me^C\me^x=c\me^x$, 其中$c=\me^C$.

\item已知$\{a_n\}$是一单增有界的正数列,试证级数$\Ser{1}(1-\frac{a_n}{a_{n+1}})$收敛.

证明:$\because\{a_n\}$是一单增有界的正数列

$\therefore\Lim a_n$存在,不妨设为$A$

$\because\varSer k1n(1-\frac{a_k}{a_{k+1}})=\varSer k1n\frac{a_{k+1}-a_k}{a_{k+1}}\leq\varSer k1n\frac{a_{k+1}-a_k}{a_2}=\frac1{a_2}(a_{n+1}-a_1)\rightarrow\frac1{a_2}(A-a_1),n\rightarrow\infty$

$\therefore\varSer k1n(1-\frac{a_k}{a_{k+1}})$有界

又$\because1-\frac{a_n}{a_{n+1}}>0$

$\therefore$根据单调有界收敛定理,级数$\Ser{1}(1-\frac{a_n}{a_{n+1}})$收敛.
\item设$a_n$是方程$\tan\sqrt x=x$的正根$(n=1,2,\cdots)$. 研究$\Ser{1}\frac1{a_n}$是否收敛.

解:令$y=\sqrt x$, 则$\tan\sqrt x=x\Leftrightarrow\tan y=y^2$

$\because a_n$是方程$\tan\sqrt x=x$的正根$(n=1,2,\cdots)$

$\therefore$由$\tan y$和$y^2$的图像可知:
\[\pi<\sqrt{a_1}<\frac32\pi,\ 2\pi<\sqrt{a_2}<\frac52\pi,\cdots,n\pi<\sqrt{a_n}<n\pi+\frac\pi2,\cdots\]
$\therefore a_n>(n\pi)^2,n=1,2,\cdots$

$\therefore\frac1{a_n}<\frac1{n^2\pi^2},n=1,2,\cdots$

$\because$级数$\Ser1\frac1{n^2}$收敛

$\therefore$级数$\Ser{1}\frac1{a_n}$收敛.

\item判定级数$\Ser{1}(\ln n+\ln\sin\frac1n)$的收敛性.

解:$\because\sin\frac1n<\frac1n(n\geq1)$

$\therefore\ln n+\ln\sin\frac1n=\ln\sin\frac1n-\ln\frac1n<0$

$\because\ln n+\ln\sin\frac1n=\ln\sin\frac1n-\ln\frac1n=\ln\frac{\sin\frac1n}{\frac1n}=\ln\frac{\frac1n-\frac1{3!n^3}+o(\frac1{n^3})}{\frac1n}=\ln[1-\frac1{6n^2}+o(\frac1{n^2})]$

$\therefore\Lim n^2[-(\ln n+\ln\sin\frac1n)]=-\Lim n^2\ln[1-\frac1{6n^2}+o(\frac1{n^2})]=-\Lim n^2[-\frac1{6n^2}+o(\frac1{n^2})]=\frac16$

$\therefore$正项级数$\Ser{1}-(\ln n+\ln\sin\frac1n)$收敛,故级数$\Ser{1}(\ln n+\ln\sin\frac1n)$收敛.

\item设正项数列$\{a_n\}$单调减少且$\Ser{1}(-1)^na_n$发散,试问级数$\Ser{1}(\frac1{1+a_n})^n$是否收敛,并说明理由.

解:$\because\{a_n\}$是正项数列且单调减少

$\therefore\{a_n\}$的极限$\Lim a_n=A$存在且$A\geq0$

$\because\Ser{1}(-1)^na_n$发散

$\therefore A\neq0$, 即$A>0$

$\because\Lim\sqrt[n]{(\frac1{1+a_n})^n}=\Lim\frac1{1+a_n}=\frac1{1+A}<1$

$\therefore$级数$\Ser{1}(\frac1{1+a_n})^n$收敛.

\item试证函数级数$\Ser{1}\frac{nx}{1+n^5x^2}$在其收敛域内一致收敛.

证明:$\because|\frac{nx}{1+n^5x^2}|=\frac n{\frac1{|x|}+n^5|x|}\leq\frac n{2\sqrt{n^5}}=\frac1{2n^{\frac32}},\ x\neq0$

又$\because$级数$\Ser1\frac1{2n^{\frac32}}$收敛

$\therefore$级数$\Ser{1}\frac{nx}{1+n^5x^2}$在其收敛域内一致收敛.

\item设$u_n>0,v_n>0,\frac{u_{n+1}}{u_n}\leq\frac{v_{n+1}}{v_n}(n=1,2,\cdots)$. 证明由$\Ser{1}v_n$收敛可以推出$\Ser{1}u_n$收敛.

解:$\because\frac{u_{n+1}}{u_n}\leq\frac{v_{n+1}}{v_n}$

$\therefore\frac{u_{n+1}}{v_{n+1}}\leq\frac{u_n}{v_n}\leq\frac{u_{n-1}}{v_{n-1}}\leq\cdots\leq\frac{u_1}{v_1}$

$\therefore u_n\leq\frac{u_1}{v_1}v_n$

$\because\Ser{1}v_n$收敛

$\therefore\Ser{1}u_n$收敛.

\item设$\Lim a_n>1$. 求证$\Ser{1}\frac1{n^{a_n}}$收敛.


证明:$\because\Lim a_n=A>1$

$\therefore\exists N>0,s.t.a_n>\frac{1+A}2=q>1(n>N)$

$\therefore$当$n>N$时$0<\frac1{n^{a_n}}<\frac1{n^q}$

$\because\Ser{N}\frac1{n^q}$收敛,故$\Ser{N}\frac1{n^{a_n}}$收敛,故$\Ser{1}\frac1{n^{a_n}}$收敛.

\item研究下列级数的收敛性:

\begin{tabular}{ll}
(1)$\Ser{1}\int_0^{n^{-p}}\ln(1+x^2)\mathrm dx(p>0)$;& (2)$\Ser{1}\int_0^{\frac1n}(\mathrm e^{\sqrt x}-1)\mathrm dx$;\\
(3)$\Ser{1}\int_0^{\frac1{\sqrt n}}(\mathrm e^{\sqrt x}-1)\mathrm dx$;& (4)$\Ser{1}\int_n^{n+1}\mathrm e^{\frac1x}\mathrm dx$.
\end{tabular}

解:(1)$\int_0^{n^{-p}}\ln(1+x^2)\mathrm dx=x\ln(1+x^2)\big|_0^{n^{-p}}-\int_0^{n^{-p}}x\frac{2x}{1+x^2}\mathrm dx\\
=n^{-p}\ln(1+n^{-2p})-2\int_0^{n^{-p}}(1-\frac1{1+x^2})\mathrm dx=n^{-p}\ln(1+n^{-2p})-2n^{-p}+2\arctan(n^{-p})\\
=n^{-p}[n^{-2p}+o(n^{-2p})]-2n^{-p}+2[n^{-p}-\frac16n^{-3p}+o(n^{-3p})]=\frac23n^{-3p}+o(n^{-3p})$

$\therefore\Lim n^{3p}\int_0^{n^{-p}}\ln(1+x^2)\mathrm dx=\Lim n^{3p}[\frac23n^{-3p}+o(n^{-3p})]=\frac23$

$\therefore$当$3p>1$即$p>\frac13$时级数收敛,当$0<3p\leq1$即$p\leq\frac13$时级数发散.

(2)$\because\int_0^{\frac1n}(\mathrm e^{\sqrt x}-1)\mathrm dx\leq\int_0^{\frac1n}(\mathrm e^{\frac1{\sqrt n}}-1)\mathrm dx=\frac1n(\mathrm e^{\frac1{\sqrt n}}-1)$

$\therefore\Lim n^{\frac32}\cdot\frac1n(\mathrm e^{\frac1{\sqrt n}}-1)=\Lim\sqrt n(\frac1{\sqrt n})=1$

$\therefore$级数$\Ser1\frac1n(\mathrm e^{\frac1{\sqrt n}}-1)$收敛

$\therefore$级数$\Ser{1}\int_0^{\frac1n}(\mathrm e^{\sqrt x}-1)\mathrm dx$收敛.

(3)$\me^{\sqrt x}-1=1+\sqrt x+\frac12(\sqrt x)^2+\frac16(\sqrt x)^3+\cdots-1=\sqrt x+\frac12(\sqrt x)^2+\frac16(\sqrt x)^3+\cdots$

$\therefore\int_0^{\frac1{\sqrt n}}(\me^{\sqrt x}-1)\mathrm dx=(\frac1{1+\frac12}x^{\frac12+1}+\frac12\frac1{1+1}x^{1+1}+\frac16\frac1{1+\frac32}x^{\frac32+1}+\cdots)\big|_0^{\frac1{\sqrt n}}=\frac23\frac1{n^{\frac34}}+\frac14\frac1n+\frac1{15}\frac1{n^{\frac54}}+\cdots$

$\therefore\Lim n\frac34\cdot\int_0^{\frac1{\sqrt n}}(\me^{\sqrt x}-1)\mathrm dx=\Lim n\frac34\cdot(\frac23\frac1{n^{\frac34}}+\frac14\frac1n+\frac1{15}\frac1{n^{\frac54}}+\cdots)=\frac23$

$\therefore$级数$\Ser{1}\int_0^{\frac1{\sqrt n}}(\mathrm e^{\sqrt x}-1)\mathrm dx$发散.

(4)$\Ser{1}\int_n^{n+1}\mathrm e^{\frac1x}\mathrm dx=\int_0^{+\infty}\me^{\frac1x}\mathrm dx$

$\because\Lim\sqrt x\cdot\me^{\frac1x}=+\infty$

$\therefore$无穷积分$\int_0^{+\infty}\me^{\frac1x}\mathrm dx$发散

$\therefore$级数$\Ser{1}\int_n^{n+1}\mathrm e^{\frac1x}\mathrm dx$发散.

\item求函数级数$\Ser{1}x^{1+\frac12+\cdots+\frac1n}$的收敛域.

解:【该题可用级数收敛的广义比值判定准则直接得到结果,即:对于正项级数$\Ser1a_n$, 若$\Lim(\frac{a_{n+1}}{a_n})^n=q$, 则当$0\leq q<\frac1\me$时,级数$\Ser1a_n$收敛;当$q>\frac1\me$时,级数$\Ser1a_n$发散.
下面的证明过程相当于是广义比值判定准则的证明过程。】

$\because\Lim(\frac{a_{n+1}}{a_n})^n=\Lim(\frac{x^{1+\frac12+\cdots+\frac1n+\frac1{n+1}}}{x^{1+\frac12+\cdots+\frac1n}})^n=\Lim x^{\frac n{n+1}}=x$

(1)当$0\leq x<\frac1\me$时$\Lim[(\frac{a_{n+1}}{a_n})^n-(\frac n{n+1})^n]=x-\frac1\me<0$

根据数列极限的保号性知$\exists N>0$, 当$n>N$时$(\frac{a_{n+1}}{a_n})^n-(\frac n{n+1})^n<0$即$(\frac{a_{n+1}}{a_n})^n<(\frac n{n+1})^n$

$\because(\frac n{n+1})^n<1$

$\therefore\exists p>1,\ s.t.(\frac{a_{n+1}}{a_n})^n\leq[(\frac n{n+1})^n]^p<(\frac n{n+1})^n$

$\therefore\frac{a_{n+1}}{\frac1{(n+1)^p}}\leq\frac{a_n}{\frac1{n^p}}\leq\cdots\leq\frac{a_1}{\frac1{1^p}}$

$\therefore a_{n}\leq a_1\frac1{n^p}$

$\therefore\Ser1a_n$收敛;

(2)当$x\geq\frac1\me$时$\Lim[(\frac{a_{n+1}}{a_n})^n-(\frac n{n+1})^n]=x-\frac1\me\geq0$

根据数列极限的保号性知$\exists N>0$, 当$n>N$时$(\frac{a_{n+1}}{a_n})^n-(\frac n{n+1})^n\geq0$即$(\frac{a_{n+1}}{a_n})^n\geq(\frac n{n+1})^n$, 即$(\frac{a_{n+1}}{a_n})^n\geq(\frac n{n+1})^n$

$\therefore\frac{a_{n+1}}{\frac1{n+1}}\geq\frac{a_n}{\frac1n}\geq\cdots\geq\frac{a_1}{\frac11}$

$\therefore a_n\geq a_1\frac1n$

$\therefore\Ser1a_n$发散;

综上所述,级数$\Ser1{a_n}$的收敛域为$[0,\frac1\me)$.

\item设$p>0$. 求证当且仅当$p>1$时,曲线$y=x^p\cos\frac\pi x(0<x\leq1)$具有有限的长度.

证明:曲线长度可表示为
\[
\int_0^1\sqrt{1+[y'(x)]^2}\mathrm dx=\int_0^1\sqrt{1+[px^{p-1}\cos\frac\pi x+\pi x^{p-2}\sin\frac\pi x]^2}\mathrm dx,
\]
(1)当$p>2$时该积分的被积函数有界,为一定积分,故原曲线有有限的长度;

(2)当$1<p\leq2$时该积分为以$0$为瑕点的反常积分,$\frac12\leq\frac1p<1,\ p+\frac1p>2$,
\[\begin{aligned}
&\LIM x0x^{\frac1p}\cdot\sqrt{1+[px^{p-1}\cos\frac\pi x+\pi x^{p-2}\sin\frac\pi x]^2}\\
=&\LIM x0\sqrt{x^{\frac2p}+[px^{p+\frac1p-1}\cos\frac\pi x+\pi x^{p+\frac1p-2}\sin\frac\pi x]^2}\\
=&0,
\end{aligned}\]
故该反常积分收敛,原曲线有有限的长度;

(3)当$0<p\leq1$时
\[\begin{aligned}
\int_0^1\sqrt{1+[y'(x)]^2}\mathrm dx&\geq\int_0^1|y'(x)|\mathrm dx=\int_0^1|[x^p\cos\frac\pi x]'|\mathrm dx=\Ser1\int_{\frac1{n+1}}^{\frac1n}|[x^p\cos\frac\pi x]'|\mathrm dx\\
&\geq\Ser1|\int_{\frac1{n+1}}^{\frac1n}[x^p\cos\frac\pi x]'\mathrm dx|=\Ser1|x^p\cos\frac\pi x\big|_{\frac1{n+1}}^{\frac1n}|\\
&=\Ser1|\frac1{n^p}(-1)^n-\frac1{(n+1)^p}(-1)^{n+1}|\\
&=\Ser1[\frac1{n^p}+\frac1{(n+1)^p}]\\
&=+\infty,
\end{aligned}\]
故该反常积分发散,原曲线为无限长.

综上所述,当且仅当$p>1$时,曲线$y=x^p\cos\frac\pi x(0<x\leq1)$具有有限的长度.
\end{enumerate}
\end{document}