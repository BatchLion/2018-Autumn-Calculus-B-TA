\documentclass[12pt,UTF8]{ctexart}
\usepackage{ctex,amsmath,amssymb,geometry,fancyhdr,bm,amsfonts
,mathtools,extarrows,graphicx,url,enumerate,color} 
% 加入中文支持
\newcommand\Set[2]{%
\left\{#1\ \middle\vert\ #2 \right\}}
\geometry{a4paper,scale=0.80}
\pagestyle{fancy}
\rhead{习题2.1\&2.2}
\lhead{基础习题课讲义}
\chead{微积分B(1)}
\begin{document}
\setcounter{section}{1}
\section{数列极限}
\noindent
\subsection{知识结构}
\noindent第二章实数与函数
	\begin{enumerate}
		\item[2.1] 数列极限的概念和性质
			\begin{enumerate}
				\item[2.1.1]数列极限的概念
				\item[2.1.2]数列极限的性质
					\begin{itemize}
						\item唯一性
						\item有界性
						\item保号性
					\end{itemize}
				\item[2.1.3]数列的子列
				\item[2.1.4]无穷大量与无界变量
			\end{enumerate}
		\item[2.2] 数列极限存在的条件
			\begin{enumerate}
				\item[2.2.1]夹逼原理
				\item[2.2.2]单调收敛定理
				\item[2.2.3]柯西收敛准则
				\item[2.2.4]数列极限的四则运算
			\end{enumerate}
\end{enumerate}
\subsection{习题2.1解答}
\begin{enumerate}
\item 用数列极限定义证明以下各题:
\newline
$(1)\lim\limits_{n\rightarrow\infty}\frac{5n^3}{1+n^3}=5;$
\newline
$(2)\lim\limits_{n\rightarrow\infty}\frac{\sin n^2}{n}=0.$
\newline
证明:(1){\bf(多项式的标准过程.)}

$\because|\frac{5n^3}{1+n^3}-5|=\frac{5}{1+n^3}$

$\therefore\forall\varepsilon>0,\ \text{取} N>\sqrt[3]{\frac5\varepsilon-1},\ \text{则当}n>N\text{时},\ |\frac{5n^3}{1+n^3}-5|<\varepsilon$

故$\lim\limits_{n\rightarrow\infty}\frac{5n^3}{1+n^3}=5$.
\newline
(2){\bf(常用$\sin x\leq1$.)}

$\because|\frac{\sin n^2}{n}-0|\leq\frac1{n}$

$\therefore\forall\varepsilon>0,\ \text{取}N>\frac1\varepsilon,\ \text{则当}n>N\text{时},\ |\frac{\sin n^2}{n}-0|<\varepsilon$

故$\lim\limits_{n\rightarrow\infty}\frac{\sin n^2}{n}=0$.
\item用极限定义证明以下各题:
\begin{enumerate}[(1)]
	\item若$\lim\limits_{n\rightarrow\infty}a_n=A$,则$\lim\limits_{n\rightarrow\infty}|a_n|=|A|$;
	\item若$\lim\limits_{n\rightarrow\infty}a_n=A>0$,则$\lim\limits_{n\rightarrow\infty}\sqrt{a_n}=\sqrt{A}$;
	\item若$\lim\limits_{n\rightarrow\infty}a_n=A$,则$\lim\limits_{n\rightarrow\infty}a_n^2=A^2$;
	\item若$\lim\limits_{n\rightarrow\infty}a_n=A$,则$\lim\limits_{n\rightarrow\infty}\frac{a_n}n=0$.
\end{enumerate}
{\bf(常用结论的证明.)}
\newline
证明:(1)$||a_n|-|A||\leq|a_n-A|$

$\because\lim\limits_{n\rightarrow\infty}a_n=A$

$\therefore\forall\varepsilon>0,\exists N>0,\ \text{使}n>N$时, $||a_n|-|A||\leq|a_n-A|<\varepsilon$

则$\lim\limits_{n\rightarrow\infty}|a_n|=|A|$.


(2)$\because\lim\limits_{n\rightarrow\infty}a_n=A>0$

$\therefore\exists N_1>0$,使$n>N_1$时,$a_n>0$

$\therefore n>N_1$时,$|\sqrt{a_n}-\sqrt A|=\frac{|a_n-A|}{\sqrt{a_n}+\sqrt A}<\frac{|a_n-A|}{\sqrt A}$

$\because\lim\limits_{n\rightarrow\infty}a_n=A$

$\therefore\forall\varepsilon>0$,$\exists N_2>0$,使$n>N_2$时,$|a_n-A|<\varepsilon$

取$N=\text{max}\{N_1,N_2\}$,当$n>N$时,$|\sqrt{a_n}-\sqrt A|=\frac{|a_n-A|}{\sqrt{a_n}+\sqrt A}<\frac{|a_n-A|}{\sqrt A}<\frac{\varepsilon}{\sqrt A}$

故$\lim\limits_{n\rightarrow\infty}\sqrt{a_n}=\sqrt{A}$.


(3)$|a_n^2-A^2|=|a_n-A||a_n+A|$,

$\because\lim\limits_{n\rightarrow\infty}a_n=A$

$\therefore\exists M>0$,使$|a_n|<M(n>0)$

$\therefore|a_n+A|<|a_n|+|A|<M+|A|$

$\therefore\forall\varepsilon>0,\exists N>0$,当$n>N$时,$|a_n-A|<\varepsilon$

$\therefore|a_n^2-A^2|=|a_n-A||a_n+A|<\varepsilon(M+|A|)$

则$\lim\limits_{n\rightarrow\infty}a_n^2=A^2$.

(4)$\because\lim\limits_{n\rightarrow\infty}a_n=A$

$\therefore\exists M>0$,使$|a_n|<M$

$\forall\varepsilon>0$,取$N>\frac M\varepsilon$,则当$n>N$时,$|\frac{a_n}n-0|<\frac Mn<\varepsilon$.

故$\lim\limits_{n\rightarrow\infty}\frac{a_n}n=0$.
\item设$\lim\limits_{n\rightarrow\infty}a_n=A,\lim\limits_{n\rightarrow\infty}b_n=B$,且$A<B$,则存在正整数$N$,使得当$n>N$时,恒有$a_n<b_n$.
\newline
证明:先证明$\lim\limits_{n\rightarrow\infty}a_n-b_n=A-B$:

$\because\lim\limits_{n\rightarrow\infty}a_n=A,\lim\limits_{n\rightarrow\infty}b_n=B$

$\therefore\forall\varepsilon>0,\exists N_1>0$,使得$n>N_1$时$|a_n-A|<\frac12\varepsilon$,当$n>N_2$时$|b_n-B|<\varepsilon$,当$n>N_2$时$|b_n-B|<\frac12\varepsilon$

取$N=$max$\{N_1,N_2\}$,则当$n>N$时$|a_n-b_n-(A-B)|<|a_n-A|+|b_n-B|<\varepsilon$

$\therefore\lim\limits_{n\rightarrow\infty}a_n-b_n=A-B$.

$\because A<B$

$\therefore A-B<0$

根据数列极限的保号性知存在正整数$N$,使得当$n>N$时,恒有$a_n<b_n$.
\end{enumerate}
\subsection{习题2.2}
\begin{enumerate}
\item用夹逼原理求下列极限
	\begin{enumerate}[(1)]
		\item$\lim\limits_{n\rightarrow\infty}(2+\frac1n)^{\frac1n}$;
		\item$\lim\limits_{n\rightarrow\infty}n^{\frac1n}$;
		\item$\lim\limits_{n\rightarrow\infty}(\frac1{n^2+1}+\frac2{n^2+2}+\cdots+\frac n{n^2+n})$;
		\item$\lim\limits_{n\rightarrow\infty}(\frac1{n^2+1}+\frac1{n^2+2}+\cdots+\frac 1{n^2+n})$.
	\end{enumerate}
证明:(1)$(2+0)^{\frac1n}<(2+\frac1n)^{\frac1n}\leq(2+1)^{\frac1n}$,即$2^{\frac1n}<(2+\frac1n)^{\frac1n}\leq3^{\frac1n}$

$\because\lim\limits_{n\rightarrow\infty}2^{\frac1n}=1,\lim\limits_{n\rightarrow\infty}3^{\frac1n}=1$

$\therefore\lim\limits_{n\rightarrow\infty}(2+\frac1n)^{\frac1n}=1$.

(2)$\because n^{\frac1n}\geq1$

$\therefore a_n=n^{\frac1n}-1\geq0$

$\therefore n=(1+a_n)^n=1+na_n+\frac{n(n+1)}{2!}a_n^2+\cdots>\frac{n(n+1)}{2!}a_n^2(n\geq2)$

$\therefore0<a_n<\sqrt{\frac2{n+1}}(n\geq2)$

$\because\lim\limits_{n\rightarrow\infty}\sqrt{\frac2{n+1}}(n\geq2)=0$

$\therefore\lim\limits_{n\rightarrow\infty}a_n=0$

$\therefore\lim\limits_{n\rightarrow\infty}n^{\frac1n}=1$.

(3)$\frac1{n^2+n}+\frac2{n^2+n}+\cdots+\frac n{n^2+n}<\frac1{n^2+1}+\frac2{n^2+2}+\cdots+\frac n{n^2+n}<\frac1{n^2}+\frac2{n^2}+\cdots+\frac n{n^2}$

即$\frac{\frac{n(n+1)}2}{n^2+n}=\frac12<\frac1{n^2+1}+\frac2{n^2+2}+\cdots+\frac n{n^2+n}<\frac{\frac{n(n+1)}2}{n^2}=\frac{n^2+n}{2n^2}$

$\because\lim\limits_{n\rightarrow\infty}\frac{n^2+n}{2n^2}=\lim\limits_{n\rightarrow\infty}\frac{1+\frac1n}{2}=\frac12$

$\therefore\lim\limits_{n\rightarrow\infty}(\frac1{n^2+1}+\frac2{n^2+2}+\cdots+\frac n{n^2+n})=\frac12$.

(4)$\frac n{\sqrt{n^2+n}}<\frac1{\sqrt{n^2+1}}+\frac1{\sqrt{n^2+2}}+\cdots+\frac 1{\sqrt{n^2+n}}<\frac n{\sqrt{n^2}}=1$

$\because\lim\limits_{n\rightarrow\infty}\frac n{\sqrt{n^2+n}}=\lim\limits_{n\rightarrow\infty}\frac 1{\sqrt{1+\frac1n}}=1$

$\therefore\lim\limits_{n\rightarrow\infty}(\frac1{\sqrt{n^2+1}}+\frac1{\sqrt{n^2+2}}+\cdots+\frac 1{\sqrt{n^2+n}})=1$.
\item用单调收敛定理求下列极限:
	\begin{enumerate}[(1)]
		\item设$x\neq0$,令$a_1=\sin x,a_n=\sin a_{n-1}(n=2,3,\cdots)$,求$\lim\limits_{n\rightarrow\infty}a_n$.
		\item设$a>0,k>0,a_1=\frac12(a+\frac ka),a_n=\frac12(a_{n-1}+\frac k{a_{n-1}})(n=2,3,\cdots)$,求证:$\lim\limits_{n\rightarrow\infty}a_n=\sqrt k$.
		\item设$x_1=a>0,y_1=b>0,x_{n+1}=\sqrt{x_ny_n},y_{n+1}=\frac12(x_n+y_n)(n=1,2,\dots)$. 求证:${x_n}$和${y_n}$收敛于同一个实数.
	\end{enumerate}
(1)解:$a_n-a_{n-1}=\sin a_{n-1}-a_{n-1}$,可分以下两种情况讨论:

(i)当$1\geq a_1=\sin x\geq0$时,$1\geq a_2=\sin a_1\geq0,1\geq a_3=\sin a_2\geq0,\cdots,1\geq a_n=\sin a_{n-1}\geq0,\cdots$

$\therefore a_n-a_{n-1}=\sin a_{n-1}-a_{n-1}\leq0(n=2,3,\cdots)$

$\therefore\{a_n\}$单调非增有下界,故收敛,记$\lim\limits_{n\rightarrow\infty}a_n=A$.

将$a_n=\sin a_{n-1}$两边取极限得$A=\sin A$,即$\lim\limits_{n\rightarrow\infty}a_n=A=0$.

(ii)当$-1\leq a_1=\sin x<0$时,$-1\leq a_2=\sin a_1<0,-1\leq a_3=\sin a_2<0,\cdots,-1\leq a_n=\sin a_{n-1}<0,\cdots$

$\therefore a_n-a_{n-1}=\sin a_{n-1}-a_{n-1}\geq0(n=2,3,\cdots)$

$\therefore\{a_n\}$单调非减有上界,故收敛,记$\lim\limits_{n\rightarrow\infty}a_n=A$.

将$a_n=\sin a_{n-1}$两边取极限得$A=\sin A$,即$\lim\limits_{n\rightarrow\infty}a_n=A=0$.

(2)证明:$a_n-a_{n-1}=\frac12(a_{n-1}+\frac k{a_{n-1}})-a_{n-1}=\frac12(\frac k{a_{n-1}}-a_{n-1})$

$\because a>0$

$\therefore a_1=\frac12(a+\frac ka)>\sqrt k$,$a_2=\frac12(a_1+\frac k{a_1})>\sqrt k,a_3=\frac12(a_2+\frac k{a_2})>\sqrt k,\cdots,a_n=\frac12(a_{n-1}+\frac k{a_{n-1}})>\sqrt k,\cdots$

$\therefore a_n-a_{n-1}=\frac12(\frac k{a_{n-1}}-a_{n-1})<0$

$\therefore\{a_n\}$单调非增有下界,故收敛,记$\lim\limits_{n\rightarrow\infty}a_n=A$.

将$a_n=\frac12(a_{n-1}+\frac k{a_{n-1}})$两边取极限得$A=\frac12(A+\frac kA)$,即$\lim\limits_{n\rightarrow\infty}a_n=A=\sqrt k$.

(3)证明:$y_{n+1}-x_{n+1}=\frac12(x_n+y_n)-\sqrt{x_ny_n}$

$\because a>0,b>0$

$\therefore x_2=\sqrt{ab}>0,y_2=\frac12{a+b}>0,x_3=\sqrt{x_2y_2},y_3=\frac12{x_2+y_2},\cdots,x_n=\sqrt{x_{n-1}y_{n-1}}>0,y_n=\frac12(x_{n-1}+y_{n-1})>0,\cdots$

$\therefore y_{n+1}-x_{n+1}=\frac12(x_n+y_n)-\sqrt{x_ny_n}\geq0(n=1,2,\cdots)$

$\therefore x_n=\sqrt{x_{n-1}y_{n-1}}\geq\sqrt{x_{n-1}x_{n-1}}=x_{n-1},y_n=\frac12(x_{n-1}+y_{n-1})\leq y_{n-1}(n=3,4,\cdots)$

$\therefore x_2=\sqrt{ab}\leq x_3\leq x_4\leq\cdots\leq x_n\leq\cdots,y_2=\frac12(a+b)\geq y_3\geq y_4\geq\cdots\geq y_n\geq\cdots$

且$x_n\leq y_n(n=2,3,\cdots)$

$\therefore\{x_n\},\{y_n\}$均单调有界,故收敛, 记$\lim\limits_{n\rightarrow\infty}x_n=A,\lim\limits_{n\rightarrow\infty}y_n=B$,则$A,B>0$.

将$x_{n+1}=\sqrt{x_ny_n}$和$y_{n+1}=\frac12(x_n+y_n)$两边取极限得$A=\sqrt{AB}$和$B=\frac12(A+B)$,即$A=B$.
\item设数列$\{a_n\}$具有这样的性质:$\forall p\in\mathbb Z^+$,有$\lim\limits_{n\rightarrow\infty}|a_{n+p}-a_n|=0$. 问$\{a_n\}$是不是柯西数列?研究下列数列是否满足上述条件?是否收敛?
\begin{enumerate}[(1)]
	\item$a_n=\sqrt n(n\in\mathbb Z^+)$;
	\item$a_n=\sum_{k=1}^{n}\frac1k$.
\end{enumerate}
解:$\{a_n\}$不一定是柯西数列,根据柯西数列的定义可知柯西数列满足该条件,但满足该条件的数列不一定是柯西数列.如: 

(1)$a_n=\sqrt n(n\in\mathbb Z^+)$,$\forall p\in\mathbb Z^+$,有$\lim\limits_{n\rightarrow\infty}|a_{n+p}-a_n|=\lim\limits_{n\rightarrow\infty}|\sqrt{n+p}-\sqrt n|=\lim\limits_{n\rightarrow\infty}\frac p{\sqrt{n+p}+\sqrt n}=\lim\limits_{n\rightarrow\infty}\frac {\frac p{\sqrt n}}{\sqrt{1+\frac p{n}}+1}=0$,但显然$\{a_n\}$不收敛(因为无界),故不是柯西数列.

(2)$a_n=\sum_{k=1}^{n}\frac1k$,$\forall p\in\mathbb Z^+$,有$\lim\limits_{n\rightarrow\infty}|a_{n+p}-a_n|=\lim\limits_{n\rightarrow\infty}\sum_{k=n+1}^{n+p}\frac1k=\lim\limits_{n\rightarrow\infty}(\frac1{n+1}+\frac1{n+1}+\cdots+\frac1{n+p})=0$,但可证明$\{a_n\}$不收敛,故不是柯西数列.

证明如下:
$\because\ln(1+\frac1k)<\frac1k$

$\therefore a_n=\sum_{k=1}^{n}\frac1k>\sum_{k=1}^{n}\ln(1+\frac1k)=\sum_{k=1}^{n}\ln(1+k)-\ln k=\ln(n+1)$

$\because\ln(n+1)$无上界

$\therefore\{a_n\}$无上界

易知$\{a_n\}$单调增加,故$\{a_n\}$发散.
\item用柯西收敛准则证明下列级数收敛:
\begin{enumerate}[(1)]
	\item$a_n=\sum_{k=1}^n\frac{\sin k}{2^k}(n\in\mathbb Z^+)$;
	\item$a_n=\sum_{k=1}^{n}\frac1{k(k+1)}$
\end{enumerate}
证明:(1)$|a_{n+p}-a_n|=\sum_{k=n+1}^{n+p}\frac{\sin k}{2^k}<\sum_{k=n+1}^{n+p}\frac1{2^k}=\frac1{2^{n+1}}\frac{1-\frac1{2^p}}{1-\frac12}<\frac1{2^n}$

$\therefore\forall\varepsilon>0$,取$N>\log_2\frac1\varepsilon$,当$n>N$时,$\forall p\in\mathbb Z^+,|a_{n+p}-a_n|<\frac1{2^n}<\varepsilon$

$\therefore\{a_n\}$是柯西数列,故收敛.

(2)$|a_{n+p}-a_n|=\sum_{k=n+1}^{n+p}\frac1{k(k+1)}=\frac1{n+1}-\frac1{n+p}<\frac1n$

$\therefore\forall\varepsilon>0$,取$N>\frac1\varepsilon$,当$n>N$时,$\forall p\in\mathbb Z^+,|a_{n+p}-a_n|<\frac1{n}<\varepsilon$

$\therefore\{a_n\}$是柯西数列,故收敛.
\item利用四则运算法则求下列极限:
\begin{enumerate}[(1)]
	\item$\lim\limits_{n\rightarrow\infty}(\frac{1+2+\cdots+n}{n+2}-\frac n2)$;
	\item$\lim\limits_{n\rightarrow\infty}(\sqrt{n^2+n}-n)$;
	\item$\lim\limits_{n\rightarrow\infty}(\sqrt[n]1+\sqrt[n]2+\cdots+\sqrt[n]100)$.
\end{enumerate}
解:(1)$\lim\limits_{n\rightarrow\infty}(\frac{1+2+\cdots+n}{n+2}-\frac n2)=\lim\limits_{n\rightarrow\infty}[\frac{\frac{n(n+1)}2}{n+2}-\frac n2]=\lim\limits_{n\rightarrow\infty}[\frac{n(n+1)}{2(n+2)}-\frac n2]=\lim\limits_{n\rightarrow\infty}[\frac{-n}{2(n+2)}]=\lim\limits_{n\rightarrow\infty}[\frac{-1}{2(1+\frac2n)}]=-\frac12$.

(2)$\lim\limits_{n\rightarrow\infty}(\sqrt{n^2+n}-n)=\lim\limits_{n\rightarrow\infty}\frac{n^2+n-n^2}{\sqrt{n^2+n}+n}=\lim\limits_{n\rightarrow\infty}\frac{1}{\sqrt{1+\frac1n}+1}=\frac12$.

(3)$\lim\limits_{n\rightarrow\infty}(\sqrt[n]1+\sqrt[n]2+\cdots+\sqrt[n]{100})=\lim\limits_{n\rightarrow\infty}\sqrt[n]1+\lim\limits_{n\rightarrow\infty}\sqrt[n]2+\cdots+\lim\limits_{n\rightarrow\infty}\sqrt[n]{100}=1+1+\cdots+1=100$.

\item设$a_n\neq0(n\in\mathbb Z^+),\lim\limits_{n\rightarrow\infty}|\frac{a_{n+1}}{a_n}|=q<1$,求证:$\lim\limits_{n\rightarrow\infty}{a_n}=0$.

解:$\because \lim\limits_{n\rightarrow\infty}|\frac{a_{n+1}}{a_n}|=q$

$\therefore \forall \varepsilon>0,\ \exists N>0,\ \text{使得}n>N\text{时},\ ||\frac{a_{n+1}}{a_n}|-q|<\varepsilon$

$\therefore  |\frac{a_{n+1}}{a_n}|<q+\varepsilon$

$\text{可知当}\varepsilon\text{足够小时}, |\frac{a_{n+1}}{a_n}|<q+\varepsilon<1(n>N), \text{即在}n>N\text{时, 数列}\{|a_n|\}\text{单调减少}$

$\because a_n\neq0$

$\therefore \lim\limits_{n\rightarrow\infty}|a_n|\text{存在}$

$\therefore\lim\limits_{n\rightarrow\infty}|a_{n+1}|=\lim\limits_{n\rightarrow\infty}|a_{n}|$

$\therefore \lim\limits_{n\rightarrow\infty}|a_{n+1}|=q\lim\limits_{n\rightarrow\infty}|a_{n}|=\lim\limits_{n\rightarrow\infty}|a_n|,\ \text{即}\lim\limits_{n\rightarrow\infty}|a_n|(1-q)=0$

$\because q<1$

$\therefore 1-q\neq0$

$\therefore\lim\limits_{n\rightarrow\infty}|a_n|=0$

$\therefore \forall \varepsilon>0,\ \exists N>0,\ \text{使得}n>N\text{时},\ |a_n|-0<\varepsilon,\ \text{即}|a_n-0|<\varepsilon$

$ \therefore\lim\limits_{n\rightarrow\infty}a_n=0$.

\item利用上题结论证明下列结论:
\begin{enumerate}[(1)]
	\item$\lim\limits_{n\rightarrow\infty}\frac{a^n}{n!}=0(a>0)$;
	\item$\lim\limits_{n\rightarrow\infty}\frac{n^2}{a^n}=0(a>1)$;
	\item$\lim\limits_{n\rightarrow\infty}\frac{a^n}{(n!)^2}=0(a>0)$.
\end{enumerate}
证明:(1)$a_n=\frac{a^n}{n!},\lim\limits_{n\rightarrow\infty}|\frac{a_{n+1}}{a_n}|=\lim\limits_{n\rightarrow\infty}\frac a{n+1}=0<1(a>0)$,故$\lim\limits_{n\rightarrow\infty}\frac{a^n}{n!}=0$.

(2)$a_n=\frac{n^2}{a^n},\lim\limits_{n\rightarrow\infty}|\frac{a_{n+1}}{a_n}|=\lim\limits_{n\rightarrow\infty}\frac1a\frac{(n+1)^2}{n^2}=\frac1a<1(a>1)$,故$\lim\limits_{n\rightarrow\infty}\frac{n^2}{a^n}=0$.

(3)$a_n=\frac{a^n}{(n!)^2},\lim\limits_{n\rightarrow\infty}|\frac{a_{n+1}}{a_n}|=\lim\limits_{n\rightarrow\infty}\frac a{(n+1)^2}=0<1(a>0)$,故$\lim\limits_{n\rightarrow\infty}\frac{a^n}{(n!)^2}=0$.
\item求极限:
\begin{enumerate}[(1)]
	\item$\lim\limits_{n\rightarrow\infty}\sin^2(\pi\sqrt{n^2+1})$;
	\item$\lim\limits_{n\rightarrow\infty}\sin^2(\pi\sqrt{n^2+n})$.
\end{enumerate}
解:(1)$\lim\limits_{n\rightarrow\infty}\sin^2(\pi\sqrt{n^2+1})=\lim\limits_{n\rightarrow\infty}\sin^2(\pi\sqrt{n^2+1}-\pi n)=\lim\limits_{n\rightarrow\infty}\sin^2\frac{\pi}{\sqrt{n^2+1}+n}=\sin^20=0$.

(2)$\lim\limits_{n\rightarrow\infty}\sin^2(\pi\sqrt{n^2+n})=\lim\limits_{n\rightarrow\infty}\sin^2(\pi\sqrt{n^2+n}-\pi n)=\lim\limits_{n\rightarrow\infty}\sin^2\frac{\pi n}{\sqrt{n^2+n}+n}=\lim\limits_{n\rightarrow\infty}\sin^2\frac{\pi}{\sqrt{1+\frac1n}+1}=\sin^2\frac\pi2=1$.


{\bf(这里用到了结论:若$\lim\limits_{n\rightarrow\infty}a_n=A$,则$\lim\limits_{n\rightarrow\infty}\sin^2a_n=\sin^2A$.)}
\end{enumerate}
\end{document}