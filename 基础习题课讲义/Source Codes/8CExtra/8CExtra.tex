\documentclass[12pt,UTF8]{ctexart}
\usepackage{ctex,amsmath,amssymb,geometry,fancyhdr,bm,amsfonts
,mathtools,extarrows,graphicx,url,enumerate,color,float,multicol} 
% 加入中文支持
\newcommand\Set[2]{%
\left\{#1\ \middle\vert\ #2 \right\}}
\geometry{a4paper,scale=0.80}
\pagestyle{fancy}
\rhead{第5章补充题}
\lhead{基础习题课讲义}
\chead{微积分B(1)}
\begin{document}
\def\thesection{8C}
\section{第5章补充题}
\def\thesubsection{\thesection.\arabic{subsection}}
\subsection{第5章补充题解答}
\begin{enumerate}
\item求证$n$次拉盖尔多项式
\[
\mathrm L_n(x)=\mathrm e^x\frac{\mathrm d^n}{\mathrm dx^n}(x^n\mathrm e^{-x})
\]在$(0,+\infty)$上有$n$个相异实根.

证明:首先证明:若$f\in C[a,+\infty),f(a)=0,\lim\limits_{x\rightarrow+\infty}f(x)=0$,且$f(x)$不恒等于$0$,则$\exists\eta\in[a,+\infty),s.t.f'(\eta)=0$.

若存在一点$x_0\in[a,+\infty),s.t.f(x_0)>0$,由于$\lim\limits_{x\rightarrow+\infty}f(x)=0$,所以$\exists X>\mathrm{max}\{a,x_0\},\\
s.t.f(x)<f(x_0)$. 在区间$[a,X]$上,对连续函数$f(x)$应用最大最小值定理可知:$\exists\eta\in[a,X],s.t.f(\eta)=\mathrm{max}\{f(x)|0\leq x\leq X\}$,则当$x>X$时,$f(x)<f(x_0)\leq f(\eta)$,所以$f(\eta)$是$f(x)$在$[a,+\infty)$上的最大值,则$f'(\eta)=0$.

同理,若存在一点$x_0\in[a,+\infty),s.t.f(x_0)<0$,则$\exists\eta',s.t.f(\eta')=\mathrm{min}\{f(x)|a\leq x<+\infty\},f'(\eta')=0$.

记$f(x)=x^n\mathrm e^{-x}$,则$f(x)$的$1$到$n-1$阶导数$f'(x),f''(x),\cdots,f^{(n-1)}(x)$都以点$x=0$为零点,且$\lim\limits_{x\rightarrow+\infty}f^{(k)}(x)=0,k=0,1,\cdots,n-1$.

根据上面证明的结论,$\exists\xi_1^{[1]},s.t.\frac{\mathrm d}{\mathrm dx}f(\xi_1^{[1]})=0$,此时$x=0,x=\xi_1^{[1]}$都是$\frac{\mathrm d}{\mathrm dx}f(x)$的零点,且仍有$\lim\limits_{x\rightarrow+\infty}\frac{\mathrm d}{\mathrm dx}f(x)=0$,根据罗尔定理和上面证明的结论$\frac{\mathrm d^2}{\mathrm dx^2}f(x)$在$(0,+\infty)$上存在两个不同的零点$x=\xi_2^{[1]},x=\xi_2^{[2]}$,此时$x=0,x=\xi_2^{[1]},x=\xi_2^{[2]}$都是$\frac{\mathrm d^2}{\mathrm dx^2}f(x)$的零点,且仍有$\lim\limits_{x\rightarrow+\infty}\frac{\mathrm d^2}{\mathrm dx^2}f(x)=0$,故$\frac{\mathrm d^3}{\mathrm dx^3}f(x)$在$(0,+\infty)$上存在$3$个不同的零点,以此类推,可知$\frac{\mathrm d^n}{\mathrm dx^n}f(x)$在$(0,+\infty)$上存在$n$个不同的零点.

因为$\mathrm L_n(x)$是一个$n$次多项式,故最多有$n$个实零点,因此$n$次拉盖尔多项式在$(0,+\infty)$上有$n$个相异实零点.

\item设$f$在$[a,b]$上可导,且$f'(a)f'(b)<0$,试证存在$\xi\in(a,b)$,使得$f'(\xi)=0$.

证明:$\because f'(a)f'(b)<0$不妨设$f'(a)>0,f'(b)<0$

$\because f'(a)>0$

$\therefore\exists x_1\in(a,b),s.t.f(x_1)>f(a)$

$\because f'(b)<0$

$\therefore\exists x_2\in(x_1,b),s.t.f(x_2)>f(b)$

$\therefore\exists\xi\in(a,b),s.t.f(\xi)=\mathrm{max}\{f(x)|a\leq x\leq b\}$

$\therefore f'(\xi)=0$.

\item设$f$在$[a,b]$上可导,且$f'(a)\neq f'(b)$,试证对于介于$f'(a)$和$f'(b)$之间的每一个实数$\mu$都存在$\xi\in(a,b)$,使$f'(\xi)=\mu$.

证明:令$F(x)=f(x)-\mu x$

$\because F'(a)F'(b)=[f'(a)-\mu][f'(b)-\mu]<0$

$\therefore$根据上题的结论,$\exists\xi\in(a,b),s.t.F'(\xi)=f'(\xi)-\mu=0$,即$f'(\xi)=\mu$.

\item设$f$在$(-\infty,+\infty)$上可导,并且满足$\frac{f(x)}{|x|}\rightarrow+\infty(x\rightarrow\infty)$,试证$\forall a\in\mathbb R,\exists\xi\in(-\infty,+\infty)$,使得$f'(\xi)=a$.

证明:方法1:$\because\frac{f(x)}{|x|}\rightarrow+\infty(x\rightarrow\infty)$

$\therefore\frac{f(x)}{x}\rightarrow+\infty(x\rightarrow+\infty),\frac{f(x)}{x}\rightarrow-\infty(x\rightarrow-\infty)$

设$x=x_0$是$f(x)$上的任意点

$\because f$在$(-\infty,+\infty)$上可导

$\therefore$根据拉格朗日中值定理,$\frac{f(x)-f(x_0)}{x-x_0}=f'(\eta),\eta$介于$x_0$和$x$之间

$\therefore\lim\limits_{\eta\rightarrow+\infty}f'(\eta)=\lim\limits_{x\rightarrow+\infty}\frac{f(x)-f(x_0)}{x-x_0}=\lim\limits_{x\rightarrow+\infty}\frac{\frac{f(x)}x-\frac{f(x_0)}x}{1-\frac{x_0}x}=+\infty,\lim\limits_{\eta\rightarrow-\infty}f'(\eta)=\lim\limits_{x\rightarrow-\infty}\frac{f(x)-f(x_0)}{x-x_0}\\
=\lim\limits_{x\rightarrow-\infty}\frac{\frac{f(x)}x-\frac{f(x_0)}x}{1-\frac{x_0}x}=-\infty$

$\therefore\forall a\in\mathbb R,\exists\eta_1>x_0,s.t.f'(\eta_1)>a,\exists\eta_2<x_0,s.t.f'(\eta_2)<f'(a)$

$\therefore\exists\xi\in(\eta_1,\eta_2),s.t.f'(\xi)=a$.

方法2:设$g(x)=f(x)-f(0)$

$\because\frac{f(x)}{|x|}\rightarrow+\infty(x\rightarrow\infty)$

$\therefore\forall a\in\mathbb R$,取$M\geq|a|$,则$\exists N>0,s.t.g(N)-g(0)=f(N)-f(0)=N[\frac{f(N)}N-\frac{f(0)}N]>NM\geq N|a|,g(-N)-g(0)=f(-N)-f(0)=N[\frac{f(-N)}N-\frac{f(0)}N]>NM\geq N|a|$

$\because g$在$(-\infty,+\infty)$上可导

$\therefore$根据拉格朗日中值定理$\exists\eta_1\in(0,N),s.t.g(N)-g(0)=g'(\eta_1)N\geq N|a|\geq Na,\exists\eta_2\in(-N,0),s.t.g(-N)-g(0)=-g'(\eta_2)N\geq N|a|\geq-Na$

$\therefore g'(\eta_2)\leq a\leq g'(\eta_1)$

$\therefore\exists\xi\in(\eta_1,\eta_2),s.t.g'(\xi)=f'(\xi)=a$.

方法3:令$g(x)=f(x)-ax$

$\because\frac{f(x)}{|x|}\rightarrow+\infty(x\rightarrow\infty)$

$\therefore\lim\limits_{x\rightarrow-\infty}g(x)=\lim\limits_{x\rightarrow-\infty}x(\frac{f(x)}x-a)=+\infty,\lim\limits_{x\rightarrow+\infty}g(x)=\lim\limits_{x\rightarrow+\infty}x(\frac{f(x)}x-a)=+\infty$

$\therefore\exists x_1>0,x_2<0,s.t.g(x_1)>0,g(x_2)<0$

$\therefore$对于$g(x_1)>0,g(x_2)>0,\exists N_1>x_2,N_2>-x_2,s.t.g(x)>g(x_1)(x>N_1),g(x)>g(x_2),(x<-N_2)$

$\because f$在$(-\infty,+\infty)$上可导,则$g(x)$在$(-\infty,+\infty)$上可导,故连续

$\therefore\exists\xi\in[N_1,N_2],s.t.g(\xi)=\min\{g(x)|N_1<x<N_2\}$且$g(\xi)<g(x_1),g(\xi)<g(x_2)$

$\therefore g(\xi)\leq g(x),x\in(-\infty,+\infty)$,即$g(\xi)=\min\{g(x)|-\infty<x<+\infty\}$

$\therefore g'(\xi)=f'(\xi)-a=0$,即$f'(\xi)=a$.

\item设$f$在$[a,b]$上可导,在$(a,b)$内二阶可导,如果$f'(a)f'(b)>0$,且$f(a)=f(b)$,试证$\exists\xi\in(a,b)$使得$f''(\xi)=0$.

证明:$\because f'(a)f'(b)>0$,不妨设$f'(a)>0,f'(b)>0$

$\therefore\exists x_1\in(a,b),s.t.f(x_1)>f(a),\exists x_2\in(x_1,b),s.t.f(x_2)<f(b)=f(a)$

$\therefore\exists\eta\in(x_1,x_2),s.t.f(\eta)=f(a)=f(b)$

$\therefore\exists\xi_1\in(a,\eta),\xi_2\in(\eta,b),s.t.f'(\xi_1)=f'(\xi_2)=0$

$\therefore\exists\xi\in(\xi_1,\xi_2)\subset(a,b),s.t.f''(\xi)=0$.

\item若$f$在$(a,b)$可导,则其导函数$f'(x)$没有第一类间断点.

证明:方法1:假设$f'(x)$存在第一类间断点$x_0\in(a,b)$,则$\lim\limits_{x\rightarrow x_0^+}f'(x)$和$\lim\limits_{x\rightarrow x_0^-}f'(x)$都存在

$\because f$在$(a,b)$可导

$\therefore f'_+(x_0)=f'(x_0)=f'_-(x_0)$

又$\because f'_+(x_0)=\lim\limits_{x\rightarrow x_0^+}\frac{f(x)-f(x_0)}{x-x_0}=\lim\limits_{\xi\rightarrow x_0^+}\frac{f'(\xi)(x-x_0)}{x-x_0}=\lim\limits_{x\rightarrow x_0^+}f'(x),f'_-(x_0)=\lim\limits_{x\rightarrow x_0^-}\frac{f(x)-f(x_0)}{x-x_0}\\
=\lim\limits_{\xi\rightarrow x_0^-}\frac{f'(\xi)(x-x_0)}{x-x_0}=\lim\limits_{x\rightarrow x_0^-}f'(x)$

$\therefore\lim\limits_{x\rightarrow x_0^+}f'(x)=f'(x_0)=\lim\limits_{x\rightarrow x_0^-}f'(x)$

$\therefore f'(x)$在$x_0$处连续,假设不成立

$\therefore f'(x)$没有第一类间断点.

方法2:假设$x_0\in(a,b)$是$f'(x)$的第一类间断点,不妨设$\lim\limits_{x\rightarrow x_0^+}f'(x)=A,A\neq f'(x_0)$(这里不妨设$A>f'(x_0)$)

根据极限的保号性,$\exists\delta>0,s.t.f'(x)>\frac{A+f'(x_0)}2>f'(x_0),x\in(x_0,x_0+\delta)$

取$x_1\in(x_0,x_0+\delta),\mu=\frac{A+3f'(x_0)}4$,则$f'(x_1)>\frac{A+f'(x_0)}2>\mu>f'(x_0)$

此时不存在$\xi\in(x_0,x_0+\delta),s.t.f'(\xi)=\mu$,这与上述第3题的结论(Darboux定理)矛盾

故$f'(x)$没有第一类间断点.

\item试举出一个函数$f$,它在$(-\infty,+\infty)$上处处可导,其导函数$f'(x)$在$x=0$处有第二类间断点.

解:$f(x)=\begin{cases}
x^2\sin\frac1x,&x\neq0\\
0,&x=0
\end{cases}$,$f'(x)=\begin{cases}
2x\sin\frac1x-\cos\frac1x,&x\neq0\\
0,&x=0
\end{cases}$

\item设$f(x)$在$[0,a]$二阶可导,$|f''(x)|\leq M,0\leq x\leq a$. 又设$f(x)$在$(0,a)$取得极大值. 求证$|f'(0)|+|f'(a)|\leq Ma$.

证明:$\because f(x)$在$(0,a)$取得极大值

$\therefore\exists\xi\in(0,a),s.t.f'(\xi)=0$

$\therefore\exists\eta_1\in(0,\xi),s.t.f'(\xi)-f'(0)=f''(\eta_1)\xi,\exists\eta_2\in(\xi,a),s.t.f'(a)-f'(\xi)=f''(\eta_2)(a-\xi)$

$\therefore|f'(0)|+|f'(a)|=|f'(\xi)-f'(0)|+|f'(a)-f'(\xi)|=|f''(\eta_1)|\xi+|f''(\eta_2)|(a-\xi)\leq M\xi+M(a-\xi)=Ma$.

\item设$f(x)$在$[0,1]$处处可导,$f(0)=0,f(1)=1$且$f(x)\not\equiv x$. 求证$\exists\xi\in(0,1)$使$f'(\xi)>1$.

证明:方法1:假设$\forall x\in(0,1),s.t.f'(x)\leq1$,令$F(x)=f(x)-x$,则$F'(x)=f'(x)-1\leq0$

$\therefore F(x)$单调非增

$\because F(0)=0=F(1)$

$\therefore\forall x\in(0,1),0=F(0)\geq F(x)\geq F(1)=0$

$\therefore F(x)\equiv0$,矛盾

故$\exists\xi\in(0,1)$使$f'(\xi)>1$.

方法2:$\because f(x)\not\equiv x$

$\therefore\exists x_0\in(0,1),s.t.f(x_0)\neq x_0$

若$f(x_0)>x_0$,则$\exists\eta\in(0,x_0),s.t.f(x_0)-f(0)=f'(\eta)x_0>x_0-f(0)=x_0$

$\therefore[f'(\eta)-1]x_0>0$,即$f'(\eta)>1$.

\item选择$a$与$b$,使得$x-(a+b\cos x)\sin x$为5阶无穷小$(x\rightarrow0)$.

解:\[\begin{split}
&x-(a+b\cos x)\sin x\\
=&x-\{a+b[\sum_{k=0}^n\frac{(-1)^k}{(2k)!}x^{2k}+o(x^{2n+1}]\}[\sum_{k=0}^n\frac{(-1)^k}{(2k+1)!}x^{2k+1}+o(x^{2n+2})]\\
=&x-\{a+b[1-\frac12x^2+\frac1{4!}x^4+o(x^5)]\}[x-\frac1{3!}x^3+\frac1{5!}x^5+o(x^6)]\\
=&(1-a-b)x+(\frac b2+\frac{a+b}{3!})x^3+(\frac b{4!}-\frac b{2\cdot3!}-\frac{a+b}{5!})x^5+o(x^5)
\end{split}\]

要使$x-(a+b\cos x)\sin x$为5阶无穷小$(x\rightarrow0)$

则\[\begin{cases}
1-a-b=0\\
\frac b2+\frac{a+b}{3!}=0
\end{cases}\]

则$a=\frac43,b=-\frac13$.

\item利用泰勒公式求下列极限:
\newline
(1)$\lim\limits_{x\rightarrow0}\frac{\sin(\sin x)-\tan(\tan x)}{\sin x-\tan x}$;
\newline
(2)$\lim\limits_{x\rightarrow0^+}\frac{\mathrm e^x-1-x}{\sqrt{1-x}-\cos\sqrt x}$;
\newline
(3)$\lim\limits_{x\rightarrow0}\frac1{x^4}[\ln(1+\sin^2x)-6(\sqrt[3]{2-\cos x}-1)]$.

解:(1)\[\begin{split}
&\lim\limits_{x\rightarrow0}\frac{\sin(\sin x)-\tan(\tan x)}{\sin x-\tan x}\\
=&\lim\limits_{x\rightarrow0}\frac{[\sin x-\frac16\sin^3x+o(\sin^4x)]-[\tan x+\frac13\tan^3x+o(\tan^4x)]}{[x-\frac16x^3+o(x^4)]-[x+\frac13x^3+o(x^4)]}\\
=&\lim\limits_{x\rightarrow0}\frac{[x-\frac16x^3+o(x^4)-\frac16x^3+o(x^3)+o(x^4)]-[x+\frac13x^3+o(x^4)+\frac13x^3+o(x^3)+o(x^4)]}{-\frac12x^3+o(x^4)}\\
=&2.
\end{split}\]

(2)\[\begin{split}
&\lim\limits_{x\rightarrow0^+}\frac{\mathrm e^x-1-x}{\sqrt{1-x}-\cos\sqrt x}\\
=&\lim\limits_{x\rightarrow0^+}\frac{1+x+\frac{x^2}{2!}+o(x^2)-1-x}{1-\frac12x+\frac{\frac12(\frac12-1)}{2!}x^2+o(x^2)-[1-\frac12x+\frac1{4!}x^2+o(x^2)]}\\
=&-3.
\end{split}\]
(3)\[\begin{split}
&\lim\limits_{x\rightarrow0}\frac1{x^4}[\ln(1+\sin^2x)-6(\sqrt[3]{2-\cos x}-1)]\\
=&\lim\limits_{x\rightarrow0}\frac1{x^4}(\sin^2x-\frac12\sin^4x+o(\sin^4x)-6\{1+\frac13(1-\cos x)+\frac{\frac13(\frac13-1)}{2!}(1-\cos x)^2\\
&+o[(1-\cos x)^2]-1\})\\
=&\lim\limits_{x\rightarrow0}\frac1{x^4}([x-\frac16x^3+o(x^3)]^2-\frac12x^4+o(x^4)+o(x^4)-6\{\frac13[\frac12x^2-\frac1{4!}x^4+o(x^4)]\\
&+\frac{\frac13(\frac13-1)}{2!}[\frac12x^2+o(x^2)]^2+o[(1-\cos x)^2]\})\\
=&\lim\limits_{x\rightarrow0}\frac1{x^4}\{x^2-\frac13x^4+o(x^4)-\frac12x^4+o(x^4)-6[\frac16x^2-\frac1{3\cdot4!}x^4+o(x^4)-\frac19\frac14x^4\\
&+o(x^4)+o(x^4)]\}\\
&=\lim\limits_{x\rightarrow0}\frac1{x^4}[-\frac13x^4-\frac12x^4-6(-\frac1{3\cdot4!}x^4-\frac19\frac14x^4)+o(x^4)]\\
=&-\frac7{12}.
\end{split}\]
\item设$f(x)$在$[a,b]$上二阶可导,证明:$\exists x_0\in(a,b)$,使得
\[
f(b)-2f(\frac{a+b}2)+f(a)=\frac{(b-a)^2}4f''(x_0).
\]

证明:$\because f(x)$在$[a,b]$上二阶可导

$\therefore f(x)=f(\frac{a+b}2)+f'(\frac{a+b}2)(x-\frac{a+b}2)+\frac{f''(\xi)}{2!}(x-\frac{a+b}2)^2,\xi$介于$x$和$\frac{a+b}2$之间

$\therefore f(a)=f(\frac{a+b}2)+f'(\frac{a+b}2)(a-\frac{a+b}2)+\frac{f''(\xi_1)}{2!}(a-\frac{a+b}2)^2\\
=f(\frac{a+b}2)+f'(\frac{a+b}2)\frac{a-b}2+\frac{f''(\xi_1)}{2!}(\frac{a-b}2)^2,\xi_1\in(a,\frac{a+b}2)$

$f(b)=f(\frac{a+b}2)+f'(\frac{a+b}2)(b-\frac{a+b}2)+\frac{f''(\xi_2)}{2!}(b-\frac{a+b}2)^2\\
=f(\frac{a+b}2)+f'(\frac{a+b}2)\frac{b-a}2+\frac{f''(\xi_2)}{2!}(\frac{b-a}2)^2,\xi_2\in(\frac{a+b}2,b)$

以上两式相加得$f(a)+f(b)=2f(\frac{a+b}2)+\frac12[f''(\xi_1)+f''(\xi_2)]\frac{(a-b)^2}4$

$\because\min\{f''(\xi_1),f''(\xi_2)\}\leq\frac12[f''(\xi_1)+f''(\xi_2)]\leq\max\{f''(\xi_1),f''(\xi_2)\}$

$\therefore\exists x_0\in[\xi_1,\xi_2]\subset(a,b),s.t.f''(x_0)=\frac12[f''(\xi_1)+f''(\xi_2)]$

即$f(b)-2f(\frac{a+b}2)+f(a)=\frac{(b-a)^2}4f''(x_0)$.

\item设$f(x)$在区间$[a,b]$上一阶可导,在$(a,b)$内二阶可导,且$f'(a)=f'(b)=0$,试证$\exists x_0\in(a,b)$,使得
\[
|f''(x_0)|\geq\frac4{(b-a)^2}|f(b)-f(a)|.
\]

证明:{\bf(该题似乎应加上$f(x)$在$a,b$两点的一阶导数连续的条件)}

$f(x)=f(a)+f'(a)(x-a)+\frac12f''(\xi_1)(x-a)^2=f(a)+\frac12f''(\xi_1)(x-a)^2,\xi_1\in(a,x)$

$f(x)=f(b)+f'(b)(x-b)+\frac12f''(\xi_2)(x-b)^2=f(b)+\frac12f''(\xi_2)(x-b)^2,\xi_2\in(x,b)$

$f(\frac{a+b}2)=f(a)+\frac12f''(\xi_1)(\frac{a+b}2-a)^2=f(a)+\frac12f''(\xi_1)(\frac{a-b}2)^2$

$f(\frac{a+b}2)=f(b)+\frac12f''(\xi_2)(\frac{a+b}2-b)^2=f(b)+\frac12f''(\xi_2)(\frac{a-b}2)^2$

以上两式相减得$f(b)-f(a)=\frac12[f''(\xi_1)-f''(\xi_2)]\frac{(a-b)^2}4$

$\therefore|f(b)-f(a)|=\frac12|f''(\xi_1)-f''(\xi_2)|\frac{(a-b)^2}4\leq\frac12[|f''(\xi_1)|+|f''(\xi_2)|]\frac{(a-b)^2}4$

记$|f''(\xi)|=\max\{|f''(\xi_1)|,|f''(\xi_2)|\}$

则$|f(b)-f(a)|\leq |f''(\xi)|\frac{(a-b)^2}4$

即$|f''(x_0)|\geq\frac4{(b-a)^2}|f(b)-f(a)|$.

\item设$f(x)\in C^2[a,b],f(a)=f(b)=0$,试证:
\newline
(1)$\max\limits_{a\leq x\leq b}|f(x)|\leq\frac18(b-a)^2\max\limits_{a\leq x\leq b}|f''(x)|$;
\newline
(2)$\max\limits_{a\leq x\leq b}|f'(x)|\leq\frac12(b-a)\max\limits_{a\leq x\leq b}|f''(x)|$.

证明:(1)设$|f(x_0)=\max\limits_{a\leq x\leq b}|f(x)|$

$\because f(a)=f(b)=0$

$\therefore x_0\in(a,b),|f(x_0)|\geq0$且$x_0$是$f(x)$的极值点

$\therefore f'(x_0)=0$

$f(x)$在$x_0$处的一阶泰勒多项式为$f(x)=f(x_0)+f'(x_0)(x-x_0)+\frac12f''(\xi)(x-x_0)^2=f(x_0)+\frac12f''(\xi)(x-x_0)^2,\xi$介于$x_0$和$x$之间

$\therefore f(a)=f(x_0)+\frac12f''(\xi_1)(a-x_0)^2=0(*)\\
f(b)=f(x_0)+\frac12f''(\xi_2)(b-x_0)^2=0(**)$

i)当$x_0\in(a,\frac{a+b}2]$时,由$(*)$式知$\max\limits_{a\leq x\leq b}|f(x)|=|f(x_0)|=\frac12(a-x_0)^2|f''(\xi_1)|\leq\frac12(a-\frac{a+b}2)|f''(\xi_1)|=\frac18(a-b)^2|f''(\xi_1)|\leq\frac18(b-a)^2\max\limits_{a\leq x\leq b}|f''(x)|$

ii)当$x_0\in(\frac{a+b}2,b)$时,由$(**)$式知$\max\limits_{a\leq x\leq b}|f(x)|=|f(x_0)|=\frac12(b-x_0)^2|f''(\xi_1)|<\frac12(b-\frac{a+b}2)|f''(\xi_1)|=\frac18(a-b)^2|f''(\xi_1)|\leq\frac18(b-a)^2\max\limits_{a\leq x\leq b}|f''(x)|$

综上所述,$\max\limits_{a\leq x\leq b}|f(x)|\leq\frac18(b-a)^2\max\limits_{a\leq x\leq b}|f''(x)|$.

(2)设$|f'(x_0)|=\max\limits_{a\leq x\leq b}|f'(x)|$

$f(x)$在$x_0$处的一阶泰勒多项式为$f(x)=f(x_0)+f'(x_0)(x-x_0)+\frac12f''(\xi)(x-x_0)^2,\xi$介于$x_0$和$x$之间

$\because f(a)=f(b)=0$

$\therefore f(a)=f(x_0)+f'(x_0)(a-x_0)+\frac12f''(\xi_1)(a-x_0)^2=0\\
f(b)=f(x_0)+f'(x_0)(b-x_0)+\frac12f''(\xi_2)(b-x_0)^2=0$

以上两式相减得$f'(x_0)(a-b)+\frac12[f''(\xi_1)(a-x_0)^2-f''(\xi_2)(b-x_0)^2]=0$

所以
\[\begin{split}
|f'(x_0)|(b-a)&=\frac12|f''(\xi_1)(a-x_0)^2-f''(\xi_2)(b-x_0)^2|\\
&\leq\frac12[(a-x_0)^2+(b-x_0)^2]\max\limits_{a\leq x\leq b}|f''(x)|\\
&=\frac12(a^2-2ax_0+x_0^2+b^2-2bx_0+x_0^2)\max\limits_{a\leq x\leq b}|f''(x)|\\
&=\frac12[2(x_0-\frac{a+b}2)+\frac{(a-b)^2}2]\max\limits_{a\leq x\leq b}|f''(x)|\\
&\leq\frac12[2(a-\frac{a+b}2)^2+\frac{(a-b)^2}2]\max\limits_{a\leq x\leq b}|f''(x)|\\
&=\frac12(b-a)^2\max\limits_{a\leq x\leq b}|f''(x)|
\end{split}\]

$\therefore\max\limits_{a\leq x\leq b}|f'(x)|=|f'(x_0)|\leq\frac12(b-a)\max\limits_{a\leq x\leq b}|f''(x)|$

\item设$f$在$(-\infty,+\infty)$有定义,并且满足$f(x+y)=f(x)f(y)$,对所有实数$x,y$都成立,又设$f'(0)=a$. 试求$f'(x)$和$f(x)$的表达式.

解:若存在$x=x_0,s.t.f(x_0)=0$,则$f(x_0+y)=f(x_0)f(y)=0,y\in\mathbb R$,即$f(x)\equiv0$

若$f(x)\neq0$,则由$f(0+0)=[f(0)]^2$得$f(0)=1$

所以
\[\begin{split}
f'(x)&=\lim\limits_{h\rightarrow0}\frac{f(x+h)-f(x)}h=\lim\limits_{h\rightarrow0}\frac{f(x)f(h)-f(x)}h=f(x)\lim\limits_{h\rightarrow0}\frac{f(h)-1}h\\
&=f(x)\lim\limits_{h\rightarrow0}\frac{f(h)-f(0)}h=f'(0)f(x)=af(x)
\end{split}\]

$\therefore f(x)=\mathrm e^{ax}$

即$f(x)=0$或$f(x)=\mathrm e^{ax}$.

\item设$f(x)$在区间$[0,+\infty)$有界,处处可导. 求证存在一个单调增加趋向于$+\infty$的点列$\{x_n\}$,使得$\lim\limits_{n\rightarrow\infty}f'(x_n)=0$.

证明:$\because f(x)$在区间$[0,+\infty)$有界,处处可导

$\therefore\exists M>0,s.t.|f(x)|\leq M,x\in[0,+\infty)$

取$a_n=2^n$,则$|f(a_n)-f(a_{n-1})|\leq|f(a_n)|+|f(a_{n-1})|\leq2M$

又$\because f(a_n)-f(a_{n-1})=f'(\xi_n)(a_n-a_{n-1})=f'(\xi_n)2^{n-1},\xi_n\in(a_{n-1},a_n),n>1$

$\therefore|f'(\xi_n)|\leq\frac{2M}{2^{n-1}},n>1$

$\because\lim\limits_{n\rightarrow\infty}\frac{2M}{2^{n-1}}=0$

$\therefore\lim\limits_{n\rightarrow\infty}f'(\xi_n)=0$

可取$x_n=\xi_n$,满足$\{x_n\}$单调增加趋向于$+\infty$且使得$\lim\limits_{n\rightarrow\infty}f'(x_n)=0$.
\end{enumerate}
\end{document}