\documentclass[12pt,UTF8]{ctexart}
\usepackage{ctex,amsmath,amssymb,geometry,fancyhdr,bm,amsfonts
,mathtools,extarrows,graphicx,url,enumerate,color,float,multicol} 
% 加入中文支持
\newcommand\Set[2]{%
\left\{#1\ \middle\vert\ #2 \right\}}
\geometry{a4paper,scale=0.80}
\pagestyle{fancy}
\rhead{习题6.1\&6.2\&6.3}
\lhead{基础习题课讲义}
\chead{微积分B(1)}
\begin{document}
\setcounter{section}{8}
\section{原函数与不定积分}
\noindent
\subsection{知识结构}
\noindent第6章用原函数与不定积分
	\begin{enumerate}
		\item[6.1] 概念和性质
			\begin{enumerate}
				\item[6.1.1] 原函数
				\item[6.1.2] 不定积分
				\item[6.1.3] 基本积分公式
				\item[6.1.4] 原函数存在的条件
				\item[6.1.5] 不定积分的线性性质
			\end{enumerate}
		\item[6.2] 换元积分法
			\begin{enumerate}
				\item[6.2.1] 第一换元法
				\item[6.2.2] 第二换元法
			\end{enumerate}
		\item[6.3] 分部积分法
		\item[6.4] 有理函数的积分
			\begin{enumerate}
				\item[6.4.1] 分式函数的积分
				\item[6.4.2] 三角函数有理式的积分
			\end{enumerate}
		\item[6.5] 简单无理式的积分、不定积分小结
			\begin{enumerate}
				\item[6.5.1] 简单无理式的积分
				\item[6.5.2] 不定积分小结
			\end{enumerate}
	\end{enumerate}
\subsection{习题6.1解答}
\begin{enumerate}
\item 证明$f(x)=\frac12x^2\text{sgn}x$是$|x|$在$(-\infty,+\infty)$的一个原函数.

证明:$f(x)=\frac12x^2\text{sgn}x=\begin{cases}
\frac12x^2,&x>0\\
0,&x=0\\
-\frac12x^2,&x<0
\end{cases}$

$f'(x)=\begin{cases}
x,&x>0\\
\lim\limits_{x\rightarrow0}\frac{f(x)-0}{x-0}=\lim\limits_{x\rightarrow0}\frac{\frac12x^2\text{sgn}x-0}{x-0}=\lim\limits_{x\rightarrow0}\frac12x\text{sgn}x=0,&x=0\\
-x,&x<0
\end{cases}=|x|$

故$f(x)=\frac12x^2\text{sgn}x$是$|x|$在$(-\infty,+\infty)$的一个原函数.

\item求下列不定积分:
\newline
(1)$\int\cos^2x\mathrm dx$;
\newline
(2)$\int\tan^2x\mathrm dx$;
\newline
(3)$\int\frac{\mathrm dx}{\sin^2x\cos^2x}$;
\newline
(4)$\int\frac{x^2-2}{x+1}\mathrm dx$;
\newline
(5)$\int\frac{x-2}{\sqrt{1+x}}\mathrm dx$.

解:(1)$\int\cos^2x\mathrm dx=\int\frac12(1+\cos2x)\mathrm dx=\frac12(\int1\mathrm dx+\int\cos2x\mathrm dx)=\frac12(x+\frac12\sin2x)+C=\frac12x+\frac14\sin2x+C$.

(2)$\int\tan^2x\mathrm dx=\int\frac{\sin^2x}{\cos^2x}\mathrm dx=\int\frac{1-\cos^2x}{\cos^2x}\mathrm dx=\int(\sec^2x-1)\mathrm dx=\tan x-x+C$.

(3)$\int\frac{\mathrm dx}{\sin^2x\cos^2x}=\int\frac{\sin^2x+\cos^2x}{\sin^2x\cos^2x}\mathrm dx=\int(\sec^2x+\csc^2x)\mathrm dx=\tan x-\cot x+C$.

(4)$\int\frac{x^2-2}{x+1}\mathrm dx=\int\frac{x^2-1-1}{x+1}\mathrm dx=\int(\frac{x^2-1}{x+1}-\frac1{x+1})\mathrm dx=\int(x-1-\frac1{x+1})\mathrm dx=\frac12x^2-x-\ln|x+1|+C$

(5)$\int\frac{x-2}{\sqrt{1+x}}\mathrm dx=\int\frac{x+1-3}{\sqrt{1+x}}\mathrm dx=\int(\sqrt{1+x}-\frac3{\sqrt{1+x}})\mathrm d(1+x)=\frac32\sqrt{(1+x)^3}-3\cdot2\sqrt{1+x}+C=\frac32\sqrt{(1+x)^3}-6\sqrt{1+x}+C$.

\item设$f(x)=\begin{cases}
e^x,&x\geq0\\
x+1,&x<0
\end{cases}.$求$\int f(x)\mathrm dx$.

解:$f(x)=e^x,x\geq0$的原函数是$F(x)=e^x+C_1$,$f(x)=x+1,x<0$的原函数是$F(x)=\frac12x^2+C_2$,

由$f(0)=1$知原函数$F(x)$在$x=0$点可导,故连续,故$1+C_1=C_2$,可取$C_1=0,C_2=1$,得$f(x)$的一个原函数$F(x)=\begin{cases}
e^x,&x\geq0\\
\frac12x^2+1,&x<0
\end{cases}.$

故$\int f(x)\mathrm dx=\begin{cases}
e^x+C,&x\geq0\\
\frac12x^2+1+C,&x<0
\end{cases}.$

\item求$\int\text{max}\{x,x^2\}\mathrm dx$.

解:$\text{max}\{x,x^2\}=\begin{cases}
x^2,&x\leq0\\
x,&0<x\leq1\\
x^2,&x>1
\end{cases}$

$\int\text{max}\{x,x^2\}\mathrm dx=\begin{cases}
\frac13x^3+C_1,&x\leq0\\
\frac12x^2+C_2,&0<x\leq1\\
\frac13x^3+C_3,&x>1
\end{cases}$

$\because\text{max}\{x,x^2\}(0)=0,\text{max}\{x,x^2\}(1)=1$

故$\text{max}\{x,x^2\}$的原函数在$x=0$和$x=1$处可导,故连续,故$C_1=C_2,\frac12+C_2=\frac13+C_3$,取$C_1=\frac12+C,C_2=\frac12+C,C_3=\frac23+C$得$\int\text{max}\{x,x^2\}\mathrm dx=\begin{cases}
\frac13x^3+\frac12+C,&x\leq0\\
\frac12x^2+\frac12+C,&0<x\leq1\\
\frac13x^3+\frac23+C,&x>1
\end{cases}.$
\end{enumerate}
\subsection{习题6.2解答}
\begin{enumerate}
\item求下列不定积分:
\begin{multicols}{2}
(1)$\int(2x+3)^4\mathrm dx$;
\newline
(2)$\int x3^{x^2+1}\mathrm dx$;
\newline
(3)$\int\frac{\ln x}x\mathrm dx$;
\newline
(4)$\int\frac1{x(2+x)}\mathrm dx$;
\newline
(5)$\int\cos x\cos3x\mathrm dx$;
\newline
(6)$\int(\frac1{\sqrt{4-x^2}}+\frac1{1+2x^2})$;
\newline
(7)$\int\frac1{1-\sin x}\mathrm dx$;
\newline
(8)$\int\frac{3x}{1+x^2}\mathrm dx$;
\newline
(9)$\int\frac{e^x}{1+e^x}\mathrm dx$;
\newline
(10)$\int\frac1{\sqrt{1-x^2}\arccos x}\mathrm dx$;
\newline
(11)$\int\frac{\sin\sqrt x}{\sqrt x}\mathrm dx$;
\newline
(12)$\int\frac1{(1+x^2)\arctan x}\mathrm dx$;
\newline
(13)$\int\cos^5x\mathrm dx$;
\newline
(14)$\int\frac1{\cos^2x-\sin^2x}\mathrm dx$;
\newline
(15)$\int\frac1{3-2x^2}\mathrm dx$;
\newline
(16)$\int\frac1{x^2-4x-12}$;
\newline
(17)$\int\frac2{\mathrm{e}^x+\mathrm{e}^{-x}}\mathrm dx$;
\newline
(18)$\int\frac1{\sin^2x+4\cos^2x}\mathrm dx$;
\newline
(19)$\int\frac1{1+\cos x}\mathrm dx$;
\newline
(20)$\int\frac{\mathrm dx}{1+\cos x}$;
\newline
(21)$\int\frac{\sin2x}{1+\cos^4x}\mathrm dx$;
\newline
(22)$\int\frac{\sqrt x}{1-\sqrt[3]x}\mathrm dx$;
\newline
(23)$\int\frac x{\sqrt{1-x}}\mathrm dx$;
\newline
(24)$\int\frac1{(4-x^2)^{\frac32}}\mathrm dx$;
\newline
(25)$\int\frac{\sqrt{3-x^2}}x\mathrm dx$;
\newline
(26)$\int\frac1{1+\sqrt{3x}}\mathrm dx$;
\newline
(27)$\int\frac1{\sqrt{1+\sqrt x}}\mathrm dx$;
\newline
(28)$\int\frac{\mathrm{e}^{2x}}{\sqrt[3]{1+\mathrm{e}^x}}\mathrm dx$;
\newline
(29)$\int\frac1{\sqrt{1+\mathrm e^x}}\mathrm dx$;
\newline
(30)$\int\frac{x^5}{\sqrt{1+x^2}}\mathrm dx$.
\end{multicols}

解:(1)$\int(2x+3)^4\mathrm dx=\frac12\int(2x+3)^4\mathrm d(2x+3)=\frac12\frac15(2x+3)^5+C=\frac1{10}(2x+3)^5+C$.

(2)$\int x3^{x^2+1}\mathrm dx=\frac12\int3^{x^2+1}\mathrm d(x^2+1)=\frac1{2\ln3}3^{x^2+1}+C$.

(3)$\int\frac{\ln x}x\mathrm dx=\int\ln x\mathrm d\ln x=\frac12(\ln x)^2+C$.

(4)$\int\frac1{x(2+x)}\mathrm dx=\frac12\int(\frac1x-\frac1{x+2})\mathrm dx=\frac12(\ln|x|-\ln|x+2|)+C=\frac12\ln|\frac x{x+2}|+C$.

(5)$\int\cos x\cos3x\mathrm dx=\int\frac12[\cos(x+3x)+\cos(x-3x)]\mathrm dx=\int\frac12[\cos4x+\cos2x]\mathrm dx=\frac12(\frac14\sin4x+\frac12\sin2x)+C=\frac18\sin4x+\frac14\sin2x+C$.

(6)$\int(\frac1{\sqrt{4-x^2}}+\frac1{1+2x^2})=\int[\frac1{2\sqrt{1-(\frac x2)^2}}+\frac1{1+(\sqrt2x)^2}]\mathrm dx=\int\frac{\mathrm d\frac x2}{\sqrt{1-(\frac x2)^2}}+\frac1{\sqrt2}\int\frac{\mathrm d\sqrt2x}{1+(\sqrt2x)^2}=\arcsin\frac x2+\frac1{\sqrt2}\arctan\sqrt2x+C$.

(7)$\int\frac1{1-\sin x}\mathrm dx=\int\frac{1+\sin x}{\cos^2x}\mathrm dx=\int(\sec^2x+\sec x\tan x)\mathrm dx=\tan x+\sec x+C$.

(8)$\int\frac{3x}{1+x^2}\mathrm dx=\frac32\int\frac1{1+x^2}\mathrm d(x^2+1)=\frac32\ln(x^2+1)+C$.

(9)$\int\frac{\mathrm e^x}{1+\mathrm e^x}\mathrm dx=\int\frac{\mathrm d(\mathrm e^x+1)}{\mathrm e^x+1}=\ln(\mathrm e^x+1)+C$.

(10)$\int\frac1{\sqrt{1-x^2}\arccos x}\mathrm dx=-\int\frac{\mathrm d\arccos x}{\arccos x}=-\ln|\arccos x|+C$.

(11)$\int\frac{\sin\sqrt x}{\sqrt x}\mathrm dx=2\int\sin\sqrt x\mathrm d\sqrt x=-2\cos\sqrt x+C$.

(12)$\int\frac1{(1+x^2)\arctan x}\mathrm dx=\int\frac{\mathrm d\arctan x}{\arctan x}=\ln|\arctan x|+C$.

(13)$\int\cos^5x\mathrm dx=\int\cos^4x\mathrm d\sin x=\int(1-\sin^2x)^2\mathrm d\sin x\xlongequal{u=\sin x}\int(1-u^2)^2\mathrm du=\int(1-2u^2+u^4)\mathrm du=u-\frac23u^3+\frac15u^5+C=\sin x-\frac23\sin^3x+\frac15\sin^5x+C$.

(14)$\int\frac1{\cos^2x-\sin^2x}\mathrm dx=\int\frac1{\cos2x}\mathrm dx=\frac12\int\sec2x\mathrm d2x=\frac12\ln|\sec2x+\tan2x|+C=\frac12\ln|\frac{1+\sin2x}{\cos2x}|+C=\frac12\ln|\frac{(\sin x+\cos x)^2}{\cos^2x-\sin^2x}|+C=\frac12\ln|\frac{\sin x+\cos x}{\sin x-\cos x}|+C$.

(15)$\int\frac1{3-2x^2}\mathrm dx=\frac12\int\frac1{\frac32-x^2}\mathrm dx=\frac12\frac1{2\sqrt{\frac32}}\int[\frac1{\sqrt{\frac32}-x}+\frac1{\sqrt{\frac32}+x}]\mathrm dx=\frac1{4\sqrt{\frac32}}\ln|\frac{x+\sqrt{\frac32}}{x-\sqrt{\frac32}}|+C=\frac1{2\sqrt6}\ln|\frac{x+\sqrt{\frac32}}{x-\sqrt{\frac32}}|+C$.

(16)$\int\frac1{x^2-4x-12}=\int\frac1{(x-6)(x+2)}\mathrm dx=\frac18\int(\frac1{x-6}-\frac1{x+2})\mathrm dx=\frac18\ln|\frac{x-6}{x+2}|+C$.

(17)$\int\frac2{\mathrm{e}^x+\mathrm{e}^{-x}}\mathrm dx=\int\frac{2e^x}{e^{2x}+1}\mathrm dx=2\int\frac{\mathrm de^x}{e^{2x}+1}=2\arctan e^x+C$.

(18)方法1:$\int\frac1{\sin^2x+4\cos^2x}\mathrm dx=\int\frac{\sec^2x}{\tan^2x+4}\mathrm dx=\frac12\int\frac{\mathrm d\frac{\tan x}2}{(\frac{\tan x}2)^2+1}=\frac12\arctan\frac{\tan x}2+C$.

方法2:$\int\frac1{\sin^2x+4\cos^2x}\mathrm dx=\int\frac{\csc^2x}{1+(2\cot x)^2}\mathrm dx=-\frac12\int\frac{\mathrm d2\cot x}{1+(2\cot x)^2}=-\frac12\arctan(2\cot x)+C.$

(19)$\int\frac1{1+\cos x}\mathrm dx=\int\frac{1-\cos x}{\sin^2x}\mathrm dx=\int(\csc^2x-\cot x\csc x)\mathrm dx=-\cot x+\csc x+C=\frac{1-\cos x}{\sin x}+C=\frac{2\sin^2\frac x2}{2\sin\frac x2\cos\frac x2}+C=\tan\frac x2+C$.

(20)$\int\frac{\mathrm dx}{1+\cos x}=\int\frac{1-\cos x}{\sin^2x}\mathrm dx=\int(\csc^2x-\cot x\csc x)\mathrm dx=-\cot x+\csc x+C$.

(21)$\int\frac{\sin2x}{1+\cos^4x}\mathrm dx=-\frac12\int\frac{\mathrm d\cos2x}{1+(\frac{1+\cos2x}2)^2}=-\int\frac{\mathrm d\frac{\cos2x+1}2}{1+(\frac{1+\cos2x}2)^2}=-\arctan\frac{1+\cos2x}2+C=-\arctan(\cos^2x)+C$.

(22)$\int\frac{\sqrt x}{1-\sqrt[3]x}\mathrm dx\xlongequal{x=t^6}\int\frac{t^3}{1-t^2}6t^5\mathrm dt=6\int\frac{t^8}{1-t^2}\mathrm dt=-6\int\frac{t^8-1+1}{t^2-1}\mathrm dt=-6\int\frac{(t^2-1)(t^2+1)(t^4+1)+1}{t^2-1}\mathrm dt=-6\int[(t^2+1)(t^4+1)+\frac1{t^2-1}]\mathrm dt=-6\int[t^6+t^4+t^2+1+\frac12(\frac1{t-1}-\frac1{t+1})]\mathrm dt=-6(\frac17t^7+\frac15t^5+\frac13t^3+t+\frac12\ln|\frac{t-1}{t+1}|)+C=-\frac67t^7-\frac65t^5-2t^3-6t-3\ln|\frac{t-1}{t+1}|+C=-\frac67x^{\frac76}-\frac65x^{\frac56}-2x^{\frac12}-6x^{\frac16}-3\ln|\frac{x^{\frac16}-1}{x^{\frac16}+1}|+C$.

(23)$\int\frac x{\sqrt{1-x}}\mathrm dx=\int\frac{x-1+1}{\sqrt{1-x}}\mathrm dx=\int(-\sqrt{1-x}+\frac1{\sqrt{1-x}})\mathrm dx=\frac23(1-x)^{\frac32}-2\sqrt{1-x}+C$.

(24)$\int\frac1{(4-x^2)^{\frac32}}\mathrm dx\xlongequal{x=2\sin t}\int\frac{2\cos t\mathrm dt}{(4-4\sin^2t)^{\frac32}}=\int\frac{2\cos t\mathrm dt}{8\cos^3t}=\int\frac{\mathrm dt}{4\cos^2t}=\frac14\tan t+C=\frac14\frac{2\sin x}{\sqrt{4-4\sin^2x}}+C=\frac14\frac x{\sqrt{4-x^2}}+C$.

(25)$\int\frac{\sqrt{3-x^2}}x\mathrm dx\xlongequal{x=\sqrt3\sin t}\int\frac{\sqrt3\cos t}{\sqrt3\sin t}\sqrt3\cos t\mathrm dt=\sqrt3\int\frac{\cos^2t}{\sin t}\mathrm dt=\sqrt3\int\frac{1-\sin^2t}{\sin t}\mathrm dt=\sqrt3\int(\csc t-\sin t)\mathrm dt=\sqrt3(\ln|\csc t-\cot t|+\cos t)+C=\sqrt3(\ln|\frac{\sqrt3}x-\frac{\sqrt{1-\frac{x^2}3}}{\frac x{\sqrt3}}|+\sqrt{1-\frac{x^2}3})+C=\sqrt3\ln|\frac{\sqrt3-\sqrt{3-x^2}}x|+\sqrt{3-x^2}+C$.

(26)$\int\frac1{1+\sqrt{3x}}\mathrm dx\xlongequal{\sqrt{3x}=t}\int\frac1{1+t}\frac{2t}3\mathrm dt=\frac23\int\frac{t+1-1}{1+t}\mathrm dt=\frac23\int(1-\frac1{1+t})\mathrm dt=\frac23[t-\ln(1+t)]+C=\frac23[\sqrt{3x}-\ln(1+\sqrt{3x})]+C$.

(27)$\int\frac1{\sqrt{1+\sqrt x}}\mathrm dx\xlongequal{t=\sqrt x}\int\frac1{\sqrt{1+t}}2t\mathrm dt=2\int\frac{1+t-1}{\sqrt{1+t}}\mathrm dt=2\int(\sqrt{1+t}-\frac1{\sqrt{1+t}})\mathrm d(1+t)=2[\frac23(1+t)^{\frac32}-2\sqrt{1+t}]+C=2[\frac23(1+\sqrt x)^{\frac32}-2\sqrt{1+\sqrt x}]+C=4\sqrt{1+\sqrt x}(\frac{\sqrt x-2}3)+C$.

(28)$\int\frac{\mathrm{e}^{2x}}{\sqrt[3]{1+\mathrm{e}^x}}\mathrm dx\xlongequal{t=\sqrt[3]{1+\mathrm e^x}}=\int\frac{(t^3-1)^2}{t}\mathrm d\ln(t^3-1)=\int\frac{(t^3-1)^2}t\frac{3t^2}{t^3-1}\mathrm dt=\int3t(t^3-1)\mathrm dt=\int(3t^4-3t)\mathrm dt=\frac35t^5-\frac32t^2+C=\frac35(1+\mathrm e^x)^{\frac53}-\frac32(1+\mathrm e^x)^{\frac23}+C=(1+e^x)^{\frac23}(\frac{3e^x}5-\frac9{10})+C$.

(29)$\int\frac1{\sqrt{1+\mathrm e^x}}\mathrm dx\xlongequal{\sqrt{1+\mathrm e^x}=t}\int\frac1t\mathrm d\ln(t^2-1)=\int\frac1t\frac{2t}{t^2-1}\mathrm dt=\int\frac2{t^2-1}\mathrm dt=\int(\frac1{t-1}-\frac1{t+1})\mathrm dt=\ln|\frac{t-1}{t+1}|+C=\ln\frac{\sqrt{1+\mathrm e^x}-1}{\sqrt{1+\mathrm e^x}+1}+C$.

(30)$\int\frac{x^5}{\sqrt{1+x^2}}\mathrm dx\xlongequal{x=\tan t}\int\frac{\tan^5t}{\sec t}\sec^2t\mathrm dt=\int\tan^5t\sec t\mathrm dt=\int\frac{\sin^5t}{\cos^6t}\mathrm dt=-\int\frac{\sin^4t}{\cos^6t}\mathrm d\cos t=-\int\frac{(1-\cos^2t)^2}{\cos^6t}\mathrm d\cos t=-\int\frac{1-2\cos^2t+\cos^4t}{\cos^6t}\mathrm d\cos t=-\int(\frac1{\cos^6t}-\frac2{\cos^4t}+\frac1{\cos^2t})\mathrm d\cos t=-(\frac1{-5\cos^5t}-\frac2{-3\cos^3t}+\frac1{-\cos t})+C=\frac1{5\cos^5t}-\frac2{3\cos^3t}+\frac1{\cos t}+C=\frac15\sec^5t-\frac23\sec^3t+\sec t+C=\frac15(1+x^2)^2\sqrt{1+x^2}-\frac23(1+x^2)\sqrt{1+x^2}+\sqrt{1+x^2}+C$.

\item证明:
\[
\int\frac{a_1\sin x+b_1\cos x}{a\sin x+b\cos x}\mathrm dx=Ax+B\ln|a\sin x+b\cos x|+C,
\]
其中$A,B$为常数,$a^2+b^2>0$.

证明:$(Ax+B\ln|a\sin x+b\cos x|+C)'=A+B\frac{a\cos x-b\sin x}{a\sin x+b\cos x}=\frac{(Aa-Bb)\sin x+(Ab+Ba)\cos x}{a\sin x+b\cos x}=\frac{a_1\sin x+b_1\cos x}{a\sin x+b\cos x}$

其中$a_1=Aa-Bb,b_1=Ab+Ba$,证毕.
\end{enumerate}
\subsection{习题6.3解答}
\begin{enumerate}
\item[]求下列不定积分
\newline
\begin{tabular}{ll}
1.$\int x\cos x\mathrm dx$;&2.$\int\frac{\ln x}{x^2}\mathrm dx$;\\
3.$\int(\ln x)^2\mathrm dx$;&4.$\int(\ln(\ln x)+\frac1{\ln x})\mathrm dx$;\\
5.$\int\arctan\sqrt x\mathrm dx$;&6.$\int\frac{x\tan x}{\cos^4x}\mathrm dx$;\\
7.$\int x^2\mathrm e^{-2x}\mathrm dx$;&8.$\int\mathrm e^{2x}\sin x\mathrm dx$;\\
9.$\int\sin\sqrt x\mathrm dx$;&10.$\int(\arccos x)^2\mathrm dx$;\\
11.$\int x\sin x\cos2x\mathrm dx$;&12.$\int\frac x{\cos^2x}\mathrm dx$;\\
13.$\int\mathrm e^{\sqrt[3]x}\mathrm dx$;&14.$\int\frac{\cos^2x}{e^x}\mathrm dx$;\\
15.$\int\sin(\ln x)\mathrm dx$;&16.$\int\frac{\ln(1+x^2)}{x^3}\mathrm dx$;\\
17.$\int(\frac{\ln x}x)^2$;&18.$\int x\ln(1+x^2)\mathrm dx$;\\
19.$\int\mathrm e^{2x}(1+\tan x)^2\mathrm dx$;&20.$\int x\tan^22x\mathrm dx$;\\
21.$\int\ln(x+\sqrt{1+x^2})\mathrm dx$;&22.$\int\frac{\arctan x}{x^2\sqrt{1-x^2}}\mathrm dx$.
\end{tabular}

解:1.$\int x\cos x\mathrm dx=\int x\mathrm d\sin x=x\sin x-\int\sin x\mathrm dx=x\sin x+\cos x+C$.

2.$\int\frac{\ln x}{x^2}\mathrm dx=\int\ln x\mathrm d(-\frac1x)=-\frac1x\ln x+\int\frac1x\mathrm d\ln x=-\frac{\ln x}x+\int\frac1{x^2}\mathrm dx=-\frac{\ln x}x-\frac1x+C$.

3.$\int(\ln x)^2\mathrm dx=x(\ln x)^2-\int x\mathrm d(\ln x)^2=x(\ln x)^2-\int x(\frac{2\ln x}x)\mathrm dx=x(\ln x)^2-2\int\ln x\mathrm dx\\
=x(\ln x)^2-2(x\ln x-\int x\mathrm d\ln x)=x(\ln x)^2-2x\ln x+2\int\mathrm dx=x(\ln x)^2-2x\ln x+2x+C$.

4.$\int(\ln(\ln x)+\frac1{\ln x})\mathrm dx=\int\ln(\ln x)\mathrm dx+\int\frac1{\ln x}\mathrm dx=x\ln(\ln x)-\int x\frac1{x\ln x}+\int\frac1{\ln x}\mathrm dx\\
=x\ln(\ln x)+C$.

5.$\int\arctan\sqrt x\mathrm dx=x\arctan\sqrt x-\int x\frac1{1+x}\frac1{2\sqrt x}\mathrm dx=x\arctan\sqrt x-\frac12\int\frac{\sqrt x}{1+x}\mathrm dx\\
=x\arctan\sqrt x-\frac12\int\frac{\sqrt x}{1+x}\mathrm d(\sqrt x)^2=x\arctan\sqrt x-\frac12\int\frac{\sqrt x}{1+x}2\sqrt x\mathrm d\sqrt x\\
=x\arctan\sqrt x-\int\frac{(\sqrt x)^2}{1+(\sqrt x)^2}\mathrm d\sqrt x=x\arctan\sqrt x-\int1-\frac1{1+(\sqrt x)^2}\mathrm d\sqrt x\\
=x\arctan\sqrt x-\sqrt x+\arctan\sqrt x+C.$

6.$\int\frac{x\tan x}{\cos^4x}\mathrm dx=\int\frac{x\sin x}{\cos^5x}\mathrm dx=-\int\frac{x\mathrm d\cos x}{\cos^5x}=\int x\mathrm d(\frac1{4\cos^4x})=\frac x{4\cos^4x}-\int\frac1{4\cos^4x}\mathrm dx\\
=\frac x{4\cos^4x}-\frac14\int\sec^4x\mathrm dx=\frac x{4\cos^4x}-\frac14\int\sec^2x\mathrm d\tan x=\frac x{4\cos^4x}-\frac14(\sec^2x\tan x-\int\tan x\mathrm d\sec^2x)\\
=\frac x{4\cos^4x}-\frac14(\sec^2x\tan x-\int\tan x\mathrm d\sec^2x)=\frac x{4\cos^4x}-\frac14(\sec^2x\tan x-\int\tan x2\sec x\sec x\tan x\mathrm dx)\\
=\frac x{4\cos^4x}-\frac14(\sec^2x\tan x-2\int\tan^2x\sec^2x\mathrm dx)=\frac x{4\cos^4x}-\frac14(\sec^2x\tan x-2\int\tan^2x\mathrm d\tan x)\\
=\frac x{4\cos^4x}-\frac14(\sec^2x\tan x-\frac23\tan^3x)+C=\frac x{4\cos^4x}-\frac14\sec^2x\tan x+\frac16\tan^3x+C\\
=\frac x{4\cos^4x}-\frac14\tan x-\frac1{12}\tan^3x+C$.

7.$\int x^2\mathrm e^{-2x}\mathrm dx=\int x^2\mathrm d(\frac{-1}2\mathrm e^{-2x})=-\frac12(x^2\mathrm e^{-2x}-\int\mathrm e^{-2x}2x\mathrm dx)=-\frac12x^2e^{-2x}+\int\mathrm e^{-2x}x\mathrm dx\\
=-\frac12x^2\mathrm e^{-2x}-\frac12\int x\mathrm d\mathrm e^{-2x}=-\frac12x^2\mathrm e^{-2x}-\frac12(x\mathrm e^{-2x}-\int\mathrm e^{-2x}\mathrm dx)\\
=-\frac12x^2\mathrm e^{-2x}-\frac12x\mathrm e^{-2x}+\frac12\int\mathrm e^{-2x}\mathrm dx=-\frac12x^2\mathrm e^{-2x}-\frac12x\mathrm e^{-2x}-\frac14\mathrm e^{-2x}+C$.

8.$\int\mathrm e^{2x}\sin x\mathrm dx=-\int\mathrm e^{2x}\mathrm d\cos x=-\mathrm e^{2x}\cos x+\int\cos x2\mathrm e^{2x}\mathrm dx=-\mathrm e^{2x}\cos x+2\int\mathrm e^{2x}\mathrm d\sin x\\
=-\mathrm e^{2x}\cos x+2(\mathrm e^{2x}\sin x-\int\sin x2\mathrm e^{2x}\mathrm dx)=-\mathrm e^{2x}\cos x+2\mathrm e^{2x}\sin x-4\int\mathrm e^{2x}\sin x\mathrm dx\\
=\frac15\mathrm e^{2x}(2\sin x-\cos x)+C$.

9.$\int\sin\sqrt x\mathrm dx\xlongequal{t=\sqrt x}2\int t\sin t\mathrm dt=-2\int t\mathrm d\cos t=-2(t\cos t-\int\cos t\mathrm dt)\\
=-2t\cos t+2\sin t+C=-2\sqrt x\cos\sqrt x+2\sin\sqrt x+C$.

10.方法1:$\int(\arccos x)^2\mathrm dx\xlongequal{t=\arccos x}\int t^2\mathrm d\cos t=t^2\cos t-2\int t\cos t\mathrm dt\\
=t^2\cos t-2\int t\mathrm d\sin t=t^2\cos t-2(t\sin t-\int\sin t\mathrm dt)=t^2\cos t-2t\sin t+2\int\sin t\mathrm dt\\
=t^2\cos t-2t\sin t-2\cos t+C=(\arccos x)^2x-2(\arccos x)\sqrt{1-x^2}-2x+C$.

方法2:$\int(\arccos x)^2\mathrm dx=x(\arccos x)^2-\int x\mathrm d(\arccos x)^2=x(\arccos x)^2-\int x(2\arccos x)\frac{-1}{\sqrt{1-x^2}}\mathrm dx\\
=x(\arccos x)^2+\int\arccos x\frac{\mathrm dx^2}{\sqrt{1-x^2}}=x(\arccos x)^2-\int\arccos x\mathrm d(2\sqrt{1-x^2})\\
=x(\arccos x)^2-(2\sqrt{1-x^2}\arccos x-2\int\sqrt{1-x^2}\frac{-1}{\sqrt{1-x^2}}\mathrm dx)\\
=x(\arccos x)^2-2\sqrt{1-x^2}\arccos x-2x+C$.

11.$\int x\sin x\cos2x\mathrm dx=\int x\sin x(2\cos^2x-1)\mathrm dx=2\int x\sin x\cos^2x\mathrm dx-\int x\sin x\mathrm dx\\
=-2\int x\cos^2x\mathrm d\cos x+\int x\mathrm d\cos x=-\frac23\int x\mathrm d\cos^3x+(x\cos x-\int\cos x\mathrm dx)\\
=-\frac23(x\cos^3x-\int\cos^3x\mathrm dx)+x\cos x-\sin x\\
=-\frac23[x\cos^3x-\int(1-\sin^2x)\mathrm d\sin x]+x\cos x-\sin x\\
=-\frac23x\cos^3x+\frac23\int(1-\sin^2x)\mathrm d\sin x+x\cos x-\sin x\\
=-\frac23x\cos^3x+\frac23\sin x-\frac29\sin^3x+x\cos x-\sin x+C\\
=-\frac23x\cos^3x-\frac29\sin^3x+x\cos x-\frac13\sin x+C$.

12.$\int\frac x{\cos^2x}\mathrm dx=\int x\mathrm d\tan x=x\tan x-\int\tan x\mathrm dx=x\tan x+\ln|\cos x|+C$.

13.$\int\mathrm e^{\sqrt[3]x}\mathrm dx\xlongequal{\sqrt[3]x=t}\int\mathrm e^t3t^2\mathrm dt=3\int t^2\mathrm d\mathrm e^t=3(t^2e^x-\int\mathrm e^t2t\mathrm dt)=3t^2e^t-6\int t\mathrm de^t\\
=3t^2\mathrm e^t-6(t\mathrm e^t-\int\mathrm e^t\mathrm dt)=3t^2\mathrm e^t-6t\mathrm e^t+6\mathrm e^t+C=\mathrm 3e^{\sqrt[3]x}(\sqrt[3]{x^2}-2\sqrt[3]x+2)+C$.

14.$\int\frac{\cos^2x}{\mathrm e^x}\mathrm dx=\int\mathrm e^{-x}\frac{1+\cos2x}2\mathrm dx=\frac12\int\mathrm e^{-x}\mathrm dx+\frac12\int\mathrm e^{-x}\cos2x\mathrm dx=-\frac12\mathrm e^{-x}-\frac12\int\mathrm e^{-x}\cos2x\mathrm dx$

$\because\frac12\int\mathrm e^{-x}\cos2x\mathrm dx=-\frac12\int\cos2x\mathrm d\mathrm e^{-x}\\=-\frac12(\mathrm e^{-x}\cos2x+\int\mathrm e^{-x}2\sin2x\mathrm dx)=-\frac12\mathrm e^{-x}\cos2x-\int\mathrm e^{-x}\sin2x\mathrm dx\\=-\frac12\mathrm e^{-x}\cos2x+\int\mathrm \sin2x\mathrm d\mathrm e^{-x}=-\frac12\mathrm e^{-x}\cos2x+\mathrm e^{-x}\sin2x-2\int\mathrm e^{-x}\cos2x\mathrm dx$

$\therefore\int\mathrm e^{-x}\cos2x\mathrm dx=\frac25(-\frac12\mathrm e^{-x}\cos2x+\mathrm e^{-x}\sin2x)+C$

$\therefore\int\frac{\cos^2x}{\mathrm e^x}\mathrm dx=-\frac12\mathrm e^{-x}-\frac12\int\mathrm e^{-x}\cos2x\mathrm dx=-\frac12\mathrm e^{-x}-\frac15(-\frac12\mathrm e^{-x}\cos2x+\mathrm e^{-x}\sin2x)+C\\
=\mathrm e^{-x}(\frac1{10}\cos2x-\frac15\sin2x-\frac12)+C$.

15.$\int\sin(\ln x)\mathrm dx=x\sin(\ln x)-\int x\mathrm d\sin(\ln x)=x\sin(\ln x)-\int x\cos(\ln x)\frac1x\mathrm dx\\
=x\sin(\ln x)-\int\cos(\ln x)\mathrm dx=x\sin(\ln x)-[x\cos(\ln x)+\int x\sin(\ln x)\frac1x\mathrm dx]\\
=x\sin(\ln x)-x\cos(\ln x)-\int\sin(\ln x)\mathrm dx=\frac12x[\sin(\ln x)-\cos(\ln x)]+C$.

16.$\int\frac{\ln(1+x^2)}{x^3}\mathrm dx=\int\ln(1+x^2)\mathrm d(-\frac1{2x^2})=-\frac1{2x^2}\ln(1+x^2)+\int\frac1{2x^2}\frac{2x}{1+x^2}\mathrm dx\\
=-\frac{\ln(1+x^2)}{2x^2}+\int\frac{1}{x(1+x^2)}\mathrm dx=-\frac{\ln(1+x^2)}{2x^2}+\int(\frac1x-\frac x{1+x^2})\mathrm dx=-\frac{\ln(1+x^2)}{2x^2}+\int\frac1x\mathrm dx-\frac12\int\frac{\mathrm d(1+x^2)}{1+x^2}\mathrm dx\\
=-\frac{\ln(1+x^2)}{2x^2}+\ln|x|-\frac12\ln(1+x^2)+C$.

17.$\int(\frac{\ln x}x)^2\mathrm dx=\int(\ln x)^2\mathrm d(-\frac1x)=-\frac1x(\ln x)^2+\int\frac1x(2\ln x)\frac1x\mathrm dx=-\frac1x(\ln x)^2+2\int\frac{\ln x}{x^2}\mathrm dx\\
=-\frac1x(\ln x)^2+2\int\ln x\mathrm d(-\frac1x)=-\frac1x(\ln x)^2+2(-\frac1x\ln x+\int\frac1x\frac1x\mathrm dx)\\
=-\frac1x(\ln x)^2-\frac2x\ln x-\frac2x+C$.

18.$\int x\ln(1+x^2)\mathrm dx=\int\ln(1+x^2)\mathrm d(\frac12x^2)=\frac12x^2\ln(1+x^2)-\int\frac12x^2\frac{2x}{1+x^2}\mathrm dx\\
=\frac12x^2\ln(1+x^2)-\int\frac{x^3}{1+x^2}\mathrm dx=\frac12x^2\ln(1+x^2)-\int(x-\frac x{1+x^2})\mathrm dx\\
=\frac12x^2\ln(1+x^2)-\int x\mathrm dx+\int\frac x{1+x^2}\mathrm dx=\frac12x^2\ln(1+x^2)-\frac12x^2+\frac12\int\frac{\mathrm d(1+x^2)}{1+x^2}\\
=\frac12x^2\ln(1+x^2)-\frac12x^2+\frac12\ln(1+x^2)+C=\frac12(1+x^2)\ln(1+x^2)-\frac12x^2+C$.

19.$\int\mathrm e^{2x}(1+\tan x)^2\mathrm dx=\int\mathrm e^{2x}(\sec^2x+2\tan x)\mathrm dx=\int\mathrm e^{2x}\mathrm d\tan x+\int\tan x\mathrm d\mathrm e^{2x}=\int\mathrm d(\mathrm e^{2x}\tan x)\\
=\mathrm e^{2x}\tan x+C$.

20.$\int x\tan^22x\mathrm dx=\int x(\sec^22x-1)\mathrm dx=\frac12\int x\mathrm d\tan2x-\frac12x^2=\frac12(x\tan2x-\int\tan2x\mathrm dx)-\frac12x^2=\frac12x\tan2x+\frac14\ln|\cos2x|-\frac12x^2+C$.

21.$\int\ln(x+\sqrt{1+x^2})\mathrm dx=x\ln(x+\sqrt{1+x^2})-\int x\frac{1+\frac{2x}{2\sqrt{1+x^2}}}{x+\sqrt{1+x^2}}\mathrm dx\\
=x\ln(x+\sqrt{1+x^2})-\int\frac x{\sqrt{1+x^2}}\mathrm dx=x\ln(x+\sqrt{1+x^2})-\frac12\int\frac{\mathrm d(1+x^2)}{\sqrt{1+x^2}}\\
=x\ln(x+\sqrt{1+x^2})-\frac122\sqrt{1+x^2}+C=x\ln(x+\sqrt{1+x^2})-\sqrt{1+x^2}+C$.

22.$\int\frac{\arcsin x}{x^2\sqrt{1-x^2}}\mathrm dx\xlongequal{t=\arcsin x}\int\frac t{\sin^2t\cos t}\cos t\mathrm dt=\int t\csc^2t\mathrm dt=-\int t\mathrm d\cot t=-t\cot t+\int\cot t\mathrm dt=-t\cot t+\ln|\sin t|+C=-\frac{\sqrt{1-x^2}}x\arcsin x+\ln|x|+C$.
\end{enumerate}
\subsection{习题6.4解答}
\begin{enumerate}
\item[]求下列不定积分:
\newline
\begin{tabular}{ll}
1.$\int\frac{x^2}{(1+x^2)^2}\mathrm dx$;&2.$\int\frac{x-1}{3+x^2}\mathrm dx$;\\
3.$\int\frac{x^3+1}{x(x-1)^3}\mathrm dx$;&4.$\int\frac{x^5}{x^6-x^3-2}\mathrm dx$;\\
5.$\int\frac{x^4\mathrm dx}{x^2+1}$;&6.$\int\frac{x^3}{(x-1)^{100}}$;\\
7.$\int\frac{x^9}{(x^{10}+2x^5+2)^2}\mathrm dx$;&8.$\int\frac x{x^2+2x-8}\mathrm dx$;\\
9.$\int\frac{x-1}{1+2x^2}\mathrm dx$;&10.$\int\frac x{x^2+4x+13}\mathrm dx$;\\
11.$\int\frac{\sin2x}{1+\cos x}\mathrm dx$;&12.$\int\frac{2+\cos x}{1+\cos x}\mathrm dx$;\\
13.$\int\frac{\mathrm dx}{\sin x\cos^3x}\mathrm dx$;&14.$\int\frac{\tan x\mathrm dx}{3\sin^2x+2\cos^2x}\mathrm dx$;\\
15.$\int\cot^3x\mathrm dx$;&16.$\int\frac{\mathrm dx}{1+\sin x+\cos x}$;\\
17.$\int\frac{\mathrm dx}{2+\cos x}$;&18.$\int\cos^4x\mathrm dx$;\\
19.$\int\frac{\mathrm dx}{\cos^2x-\sin^2x}$;&20.$\int\frac{\mathrm dx}{\sin2x+1}$.
\end{tabular}

解:1.方法1:$\int\frac{x^2}{(1+x^2)^2}\mathrm dx=\int[\frac1{1+x^2}-\frac1{(1+x^2)^2}]\mathrm dx=\arctan x-\int\frac1{(1+x^2)^2}\mathrm dx=\arctan x-[\frac12\arctan x+\frac x{2(1+x^2)}]+C=\frac12\arctan x-\frac x{2(1+x^2)}+C$.

方法2:$\int\frac{x^2}{(1+x^2)^2}\mathrm dx\xlongequal{x=2\tan t}\int\frac{\tan^2t}{\sec^4x}\sec^2t\mathrm dt=\int\sin^2t\mathrm dt=\int\frac12(1-\cos2t)\mathrm dt=\frac12t-\frac14\sin2t+C=\frac12\arctan x-\frac x{2(1+x^2)}+C$.

2.$\int\frac{x-1}{3+x^2}\mathrm dx=\frac12\int\frac{\mathrm d(3+x^2)}{3+x^2}-\frac13\int\frac1{1+(\frac x{\sqrt3})^2}\mathrm dx=\frac12\ln(3+x^2)-\frac1{\sqrt3}\arctan(\frac x{\sqrt3})+C$.

3.记$\frac{x^3+1}{x(x-1)^3}=\frac Ax+\frac B{x-1}+\frac C{(x-1)^2}+\frac D{(x-1)^3}$

则$x^3+1=A(x-1)^3+Bx(x-1)^2+Cx(x-1)+Dx$

可得$A=-1,B=2,C=1,D=2$

$\therefore\int[\frac{-1}x+\frac2{x-1}+\frac1{(x-1)^2}+\frac2{(x-1)^3}]\mathrm dx=-\ln|x|+2\ln|x-1|-\frac1{x-1}-\frac1{(x-1)^2}+C$.

4.$\int\frac{x^5}{x^6-x^3-2}\mathrm dx=\frac16\int\frac{\mathrm d(x^6-x^3-2)}{x^6-x^3-2}+\frac12\int\frac{x^2}{x^6-x^3-2}\mathrm dx=\frac16\ln|x^6-x^3-2|+\frac16\int\frac{\mathrm dx^3}{(x^3+1)(x^3-2)}=\frac16\ln|x^6-x^3-2|-\frac1{18}\int(\frac1{x^3+1}-\frac1{x^3-2})\mathrm dx^3=\frac16\ln|x^6-x^3-2|-\frac1{18}(\ln|x^3+1|-\ln|x^3-2|)+C\\
=\frac19\ln|x^3+1|+\frac29\ln|x^3-2|+C$.

5.$\int\frac{x^4\mathrm dx}{x^2+1}=\int(x^2-1+\frac1{x^2+1})\mathrm dx=\frac13x^3-x+\arctan x+C$.

6.方法1:记$\frac{x^3}{(x-1)^{100}}=\frac{A_1}{x-1}+\frac{A_2}{(x-1)^2}+\cdots+\frac{A_{96}}{(x-1)^{96}}+\frac{A_{97}}{(x-1)^{97}}+\frac{A_{98}}{(x-1)^{98}}+\frac{A_{99}}{(x-1)^{99}}+\frac{A_{100}}{(x-1)^{100}}\\
=\frac{A_1(x-1)^{99}+A_2(x-1)^{98}+\cdots+A_{96}(x-1)^4+A_{97}(x-1)^3+A_{98}(x-1)^2+A_{99}(x-1)+A_{100}}{(x-1)^{100}}$

则$A_1=A_2=\cdots=A_{96}=0$

$A_{97}=1,-3A_{97}+A_{98}=0,3A_{97}-2A_{98}+A_{99}=0,-A_{97}+A_{98}-A_{99}+A_{100}=0$

$\therefore A_{97}=1,A_{98}=3,A_{99}=3,A_{100}=1$

$\therefore\int\frac{x^3}{(x-1)^{100}}\mathrm dx=\int[\frac1{(x-1)^{97}}+\frac3{(x-1)^{98}}+\frac3{(x-1)^{99}}+\frac1{(x-1)^{100}}]\mathrm dx\\
=-\frac1{96(x-1)^{96}}-\frac3{97(x-1)^{97}}-\frac3{98(x-1)^{98}}-\frac1{99(x-1)^{99}}+C\\
=-\frac1{(x-1)^{96}}[\frac1{96}+\frac3{97(x-1)}+\frac3{98(x-1)^2}+\frac1{99(x-1)^3}]+C$.

方法2:$\int\frac{x^3}{(x-1)^{100}}\mathrm dx\xlongequal{t=x-1}\int\frac{(t+1)^3}{t^{100}}\mathrm dt=\int\frac{t^3+3t^2+3t+1}{t^{100}}\mathrm dt=\int(\frac1{t^{97}}+\frac3{t^{98}}+\frac3{t^{99}}+\frac1{t^{100}})\mathrm dt\\
=-\frac1{96t^{96}}-\frac3{97t^{97}}-\frac3{98t^{98}}-\frac1{99t^{99}}+C=-\frac1{(x-1)^{96}}[\frac1{96}+\frac3{97(x-1)}+\frac3{98(x-1)^2}+\frac1{99(x-1)^3}]+C$

7.$\int\frac{x^9}{(x^{10}+2x^5+2)^2}\mathrm dx=\frac1{10}\int\frac{\mathrm d(x^{10}+2x^5+2)}{(x^{10}+2x^5+2)^2}-\int\frac{x^4\mathrm dx}{(x^{10}+2x^5+2)^2}=-\frac1{10}\frac1{x^{10}+2x^5+2}-\frac15\int\frac{\mathrm d(x^5+1)}{[(x^5+1)^2+1]^2}\\
=-\frac1{10}\frac1{x^{10}+2x^5+2}-\frac15[\frac12\arctan(x^5+1)+\frac{x^5+1}{2[(x^5+1)^2+1]}]+C\\
=-\frac{x^5+2}{10[(x^5+1)^2+1]}-\frac1{10}\arctan(x^5+1)+C$.

8.$\int\frac x{x^2+2x-8}\mathrm dx=\frac12\int\frac{\mathrm d(x^2+2x-8)}{x^2+2x-8}-\int\frac1{x^2+2x-8}\mathrm dx=\frac12\ln|x^2+2x-8|-\int\frac1{(x+4)(x-2)}\mathrm dx\\
=\frac12\ln|x^2+2x-8|+\frac16\int(\frac1{x+4}-\frac1{x-2})\mathrm dx=\frac12\ln|x^2+2x-8|+\frac16(\ln|x+4|-\ln|x-2|)+C\\
=\frac23\ln|x+4|+\frac13\ln|x-2|+C$.

9.$\int\frac{x-1}{1+2x^2}\mathrm dx=\frac14\int\frac{\mathrm d(1+2x^2)}{1+2x^2}-\frac1{\sqrt2}\int\frac1{1+(\sqrt2x)^2}\mathrm d(\sqrt2x)=\frac14\ln(1+2x^2)-\frac1{\sqrt2}\arctan(\sqrt2x)+C$.

10.$\int\frac x{x^2+4x+13}\mathrm dx=\frac12\int\frac{\mathrm d(x^2+4x+13)}{x^2+4x+13}-2\int\frac1{x^2+4x+13}\mathrm dx=\frac12\ln(x^2+4x+13)-\frac23\int\frac1{(\frac{x+2}3)^2+1}\mathrm d\frac{x+2}3\\
=\frac12\ln(x^2+4x+13)-\frac23\arctan\frac{x+2}3+C$.

11.$\int\frac{\sin2x}{1+\cos x}\mathrm dx=-\int\frac{2\cos x}{1+\cos x}\mathrm d\cos x=-\int(2-\frac2{1+\cos x})\mathrm d\cos x=-2\cos x+2\ln(1+\cos x)+C$.

12.$\int\frac{2+\cos x}{1+\cos x}\mathrm dx=\int(1+\frac1{1+\cos x})\mathrm dx=x+\int\frac1{2\cos^2\frac x2}\mathrm dx=x+\int\sec^2\frac x2\mathrm d\frac x2=x+\tan\frac x2+C\\
=x-\cot x+\csc x+C$.

13.$\int\frac{\mathrm dx}{\sin x\cos^3x}=\int\frac{\sec^2x\mathrm dx}{\sin x\cos x}=\int\frac{\mathrm d\tan x}{\frac{\tan x}{1+\tan^2x}}=\int(\frac1{\tan x}+\tan x)\mathrm d\tan x=\ln|\tan x|+\frac12\tan^2x+C$.

14.$\int\frac{\tan x\mathrm dx}{3\sin^2x+2\cos^2x}=\int\frac{\tan x\sec^2x}{3\tan^2x+2}\mathrm dx=\frac12\int\frac{\tan x}{1+(\frac{\sqrt3}{\sqrt2}\tan x)^2}\mathrm d\tan x=\frac13\int\frac{\frac{\sqrt3}{\sqrt2}\tan x}{1+(\frac{\sqrt3}{\sqrt2}\tan x)^2}\mathrm d\frac{\sqrt3}{\sqrt2}\tan x\\
=\frac16\int\frac{\mathrm d[1+(\frac{\sqrt3}{\sqrt2}\tan x)^2]}{1+(\frac{\sqrt3}{\sqrt2}\tan x)^2}=\frac16\ln(1+\frac32\tan^2x)+C$.

15.$\int\cot^3x\mathrm dx=\int(\csc^2x-1)\cot x\mathrm dx=\int\csc^2x\cot x\mathrm dx-\int\cot x\mathrm dx=-\int\cot\mathrm d\cot x-\ln|\sin x|=-\frac12\cot^2x-\ln|\sin x|+C$.

16.$\int\frac{\mathrm dx}{1+\sin x+\cos x}=\int\frac{\mathrm dx}{2\cos^2\frac x2+2\sin\frac x2\cos\frac x2}=\int\frac{\sec^2\frac x2\mathrm d\frac x2}{1+\tan\frac x2}=\int\frac{\mathrm d\tan\frac x2}{1+\tan\frac x2}=\ln|1+\tan\frac x2|+C$.

17.$\int\frac{\mathrm dx}{2+\cos x}=\int\frac{\mathrm dx}{\sin^2\frac x2+3\cos^2\frac x2}=2\int\frac{\sec^2\frac x2\mathrm d\frac x2}{\tan^2\frac x2+3}=\frac2{\sqrt3}\int\frac{\mathrm d\frac1{\sqrt3}\tan\frac x2}{(\frac1{\sqrt3}\tan\frac x2)^2+1}=\frac2{\sqrt3}\arctan\frac{\tan\frac x2}{\sqrt3}+C$.

18.$\int\cos^4x\mathrm dx=\int\frac{\cos^4x}{(\sin^2x+\cos^2x)^3}\mathrm dx=\int\frac{\sec^2x\mathrm dx}{(\tan^2x+1)^3}=\int\frac{\mathrm d\tan x}{(\tan^2x+1)^3}=\frac34\int\frac{\mathrm d\tan x}{(\tan^2x+1)^2}+\frac{\tan x}{4(\tan^2x+1)^2}\\
=\frac34[\frac12x+\frac{\tan x}{2(\tan^2x+1)}]+\frac{\tan x}{4(\tan^2x+1)^2}+C=\frac38x+\frac{3\tan x}{8(\tan^2x+1)}+\frac{\tan x}{4(\tan^2x+1)^2}+C\\
=\frac38x+\frac38\sin x\cos x+\frac14\sin x\cos^3x+C=\frac38x+\frac3{16}\sin2x+\frac1{16}\sin2x(1+\cos2x)+C\\
=\frac38x+\frac14\sin2x+\frac1{32}\sin4x+C$.

19.$\int\frac{\mathrm dx}{\cos^2x-\sin^2x}=\int\frac{\mathrm dx}{\cos2x}=\frac12\int\sec2x\mathrm d2x=\frac12\ln|\sec2x+\tan2x|+C\\
=\frac12\ln|\frac{\cos x+\sin x}{\cos x-\sin x}|+C$.

20.$\int\frac{\mathrm dx}{\sin2x+1}=\int\frac{\mathrm dx}{(\sin x+\cos x)^2}=\int\frac{\sec^2x\mathrm dx}{(\tan x+1)^2}=\int\frac{\mathrm d\tan x}{(\tan x+1)^2}=-\frac1{1+\tan x}+C=-\frac{\sin x}{\sin x+\cos x}+C$.
\end{enumerate}
\subsection{习题6.5解答}
\begin{enumerate}
\item[]求下列不定积分:
\newline
\begin{tabular}{ll}
(1)$\int\frac{\mathrm dx}{1+\sqrt x}$;&(2)$\int\frac{\mathrm dx}{\sqrt x+\sqrt[3]x}$;\\
(3)$\int\frac{\sqrt{1-x}}x\mathrm dx$;&(4)$\int\frac{\mathrm dx}{x\sqrt{2x+1}}$;\\
(5)$\int\frac{\mathrm dx}{\sqrt[3]{1-3x}}$;&(6)$\int\frac{\sqrt x}{\sqrt{1-x}}\mathrm dx$;\\
(7)$\int\frac{\mathrm dx}{\sqrt{x^2-x}}$;&(8)$\int\frac1{x^2+2x+3}\mathrm dx$;\\
(9)$\int\frac{\mathrm dx}{\sqrt{x^2-4x}}$;&(10)$\int\frac{x^3}{\sqrt{x^8}}\mathrm dx$;\\
(11)$\int\frac{\mathrm dx}{\sqrt{1+\mathrm e^x}}$;&(12)$\int\frac{\sqrt{1+x^2}}x\mathrm dx$;\\
(13)$\int2\mathrm e^x\sqrt{1-\mathrm e^{2x}}\mathrm dx$;&(14)$\int\frac{\mathrm dx}{x^2\sqrt{x^2+9}}$;\\
(15)$\int\mathrm e^{\sqrt{2x-1}}\mathrm dx$.
\end{tabular}

解:(1)$\int\frac{\mathrm dx}{1+\sqrt x}\xlongequal{t=\sqrt x}\int\frac{2t\mathrm dt}{1+t}=2\int(1-\frac1{1+t})\mathrm dt=2(t-\ln|1+t|)+C=2t-2\ln|1+t|+C\\
=2\sqrt x-2\ln(1+\sqrt x)|+C$.

(2)$\int\frac{\mathrm dx}{\sqrt x+\sqrt[3]x}\xlongequal{x=t^6}\int\frac{6t^5\mathrm dt}{t^3+t^2}=6\int\frac{t^3}{t+1}\mathrm dt=6\int(t^2-t+1-\frac1{t+1})\mathrm dt\\
=6(\frac13t^3-\frac12t^2+t-\ln|1+t|)+C=2t^3-3t^2+6t-6\ln|1+t|+C\\
=2\sqrt x-3\sqrt[3]x+6\sqrt[6]x-6\ln(1+\sqrt[6]x)+C$.

(3)$\int\frac{\sqrt{1-x}}x\mathrm dx\xlongequal{\sqrt{1-x}=t}\int\frac{-2t^2\mathrm dt}{1-t^2}=2\int\frac{t^2}{t^2-1}\mathrm dt=2\int(1+\frac1{t^2-1})\mathrm dt=2t+\int(\frac1{t-1}-\frac1{t+1})\mathrm dt\\
=2t+\ln|\frac{t-1}{t+1}|+C=2\sqrt{1-x}+\ln|\frac{\sqrt{1-x}-1}{\sqrt{1-x}+1}|+C$.

(4)$\int\frac{\mathrm dx}{x\sqrt{2x+1}}\xlongequal{\sqrt{2x+1}=t}\int\frac{t\mathrm dt}{\frac12(t^2-1)t}=\int\frac{2\mathrm dt}{t^2-1}=\int(\frac1{t-1}-\frac1{t+1})\mathrm dt=\ln|\frac{t-1}{t+1}|+C\\
=\ln|\frac{\sqrt{2x+1}-1}{\sqrt{2x+1}+1}|+C$.

(5)$\int\frac{x\mathrm dx}{\sqrt[3]{1-3x}}\xlongequal{t=\sqrt[3]{1-3x}}\int\frac{\frac13(1-t^3)(-t^2)\mathrm dt}t=\frac13\int(t^4-t)\mathrm dt=\frac13(\frac15t^5-\frac12t^2)+C\\
=\frac13\sqrt[3]{(1-3x)^2}[\frac15(1-3x)-\frac12]+C=-\frac1{10}\sqrt[3]{1-3x}(1+2x)+C$.

(6)$\int\frac{\sqrt x}{1-x}\mathrm dx\xlongequal{t=\sqrt x}\int\frac{2t^2}{1-t^2}\mathrm dt=2\int(-1-\frac1{t^2-1})\mathrm dt=-2t-\int(\frac1{t-1}-\frac1{t+1})\mathrm dt\\
=-2t-\ln|\frac{t-1}{t+1}|+C=-2\sqrt x-\ln|\frac{\sqrt x-1}{\sqrt x+1}|+C$.

(7)方法1:$\int\frac{\mathrm dx}{\sqrt{x^2-x}}=\int\frac{\mathrm dx}{\sqrt{(x-\frac12)^2-\frac14}}\xlongequal{x-\frac12=\frac12\sec t}\int\frac{\frac12\sec t\tan t\mathrm dt}{\frac12\tan t}=\int\sec t\mathrm dt\\
=\ln|\sec t+\tan t|+C=\ln|2(x-\frac12)+\sqrt{4(x-\frac12)^2-1}|+C=\ln|x-\frac12+\sqrt{x^2-x}|+C$.

方法2:$\int\frac{\mathrm dx}{\sqrt{x^2-x}}=\int\sqrt{\frac{x-1}x}\frac1{x-1}\mathrm dx\xlongequal[x=\frac1{1-t^2}]{t=\sqrt{\frac{x-1}x}}\int t\frac1{\frac{1-1+t^2}{1-t^2}}\frac{2t}{(1-t^2)^2}\mathrm dt=\int\frac2{1-t^2}\mathrm dt\\
=\int(\frac1{1+t}+\frac1{1-t})\mathrm dt=\ln|\frac{1+t}{1-t}|+C=\ln|\frac{1+\sqrt{\frac{x-1}x}}{1-\sqrt{\frac{x-1}x}}|+C=\ln|\frac{\sqrt x+\sqrt{x-1}}{\sqrt x-\sqrt{x-1}}|+C\\
=2\ln(\sqrt x+\sqrt{x-1})+C$.
\footnotemark[1]\footnotetext[1]{对方法2而言这里可以只考虑$x>1$的情况。如果需要考虑符号,则可这样做:

$\int\frac{\mathrm dx}{\sqrt{x^2-x}}=\mathrm{sgn}(x-1)\int\sqrt{\frac{x-1}x}\frac1{x-1}\mathrm dx\xlongequal[x=\frac1{1-t^2}]{t=\sqrt{\frac{x-1}x}}\mathrm{sgn}(1-t)\int t\frac1{\frac{1-1+t^2}{1-t^2}}\frac{2t}{(1-t^2)^2}\mathrm dt=\mathrm{sgn}(1-t)\int\frac2{1-t^2}\mathrm dt\\
=\mathrm{sgn}(1-t)\int(\frac1{1+t}+\frac1{1-t})\mathrm dt=\mathrm{sgn}(1-t)\ln|\frac{1+t}{1-t}|+C=\mathrm{sgn}(x-1)\ln|\frac{1+\sqrt{\frac{x-1}x}}{1-\sqrt{\frac{x-1}x}}|+C\\
=\begin{cases}
\ln|\frac{1+\sqrt{\frac{x-1}x}}{1-\sqrt{\frac{x-1}x}}|+C,&x>1\\
-\ln|\frac{1+\sqrt{\frac{x-1}x}}{1-\sqrt{\frac{x-1}x}}|+C,&x<0
\end{cases}=\begin{cases}
\ln|\frac{(1+\sqrt{\frac{x-1}x})^2}{\frac1x}|+C,&x>1\\
-\ln|\frac{(1+\sqrt{\frac{x-1}x})^2}{\frac1x}|+C,&x<0
\end{cases}=\begin{cases}
\ln|x(1+\sqrt{\frac{x-1}x})^2|+C,&x>1\\
-\ln|x(1+\sqrt{\frac{x-1}x})^2|+C,&x<0
\end{cases}\\
=\begin{cases}
\ln|x+x-1+2x\sqrt{\frac{x-1}x}|+C,&x>1\\
-\ln|x+x-1+2x\sqrt{\frac{x-1}x}|+C,&x<0
\end{cases}=\begin{cases}
\ln|2x-1+2\sqrt{x^2-x}|+C,&x>1\\
-\ln|2x-1-2\sqrt{x^2-x}|+C,&x<0
\end{cases}\\
=\begin{cases}
\ln|2x-1+2\sqrt{x^2-x}|+C,&x>1\\
\ln|2x-1+2\sqrt{x^2-x}|+C,&x<0
\end{cases}=\ln|2x-1+2\sqrt{x^2-x}|+C$.

其中由于$t=\sqrt{\frac{x-1}x}=\begin{cases}
>1,&x<0\\
<1,&x>1
\end{cases}$,故$\mathrm{sgn}(x-1)=\mathrm{sgn}(1-t)$.}

(8)$\int\frac1{\sqrt{x^2+2x+3}}\mathrm dx=\int\frac1{\sqrt{(x+1)^2+2}}\mathrm dx\xlongequal{x+1=\sqrt2\tan t}\int\frac{\sqrt2\sec^t\mathrm dt}{\sqrt2\sec t}=\int\sec t\mathrm dt\\
=\ln|\sec t+\tan t|+C=\ln|\frac{x+1}{\sqrt2}+\sqrt{\frac{(x+1)^2}2+1}|+C=\ln(x+1+\sqrt{x^2+2x+3})+C\\
=\mathrm{arcsinh}\frac{x+1}{\sqrt2}+C$.

(9)$\int\frac{\mathrm dx}{\sqrt{x^2-4x}}=\int\frac{\mathrm dx}{\sqrt{(x-2)^2-4}}\xlongequal{x-2=2\sec t}\int\frac{2\sec t\tan t\mathrm dt}{2\tan t}=\int\sec t\mathrm dt=\ln|\sec t+\tan t|+C\\
=\ln|\frac12(x-2)+\sqrt{\frac14(x-2)^2-1}|+C=\ln|x-2+\sqrt{x^2-4x}|+C$.

(10)$\int\frac{x^3}{\sqrt{x^8+1}}\mathrm dx=\frac14\int\frac{\mathrm d(x^4)}{\sqrt{(x^4)^2+1}}\xlongequal{x^4=\tan t}\frac14\int\frac{\sec^2t\mathrm dt}{\sec t}=\frac14\int\sec t\mathrm dt=\frac14\ln|\sec t+\tan t|+C\\
=\frac14\ln|t^4+\sqrt{t^8+1}|+C=\frac14\mathrm{arcsinh}(x^4)+C$.

(11)$\int\frac{\mathrm dx}{\sqrt{1+\mathrm e^x}}\xlongequal{t=\sqrt{1+\mathrm e^x}}\int\frac1t\frac{2t}{t^2-1}\mathrm dt=\int(\frac1{t-1}-\frac1{t+1})\mathrm dt=\ln|\frac{t-1}{t+1}|+C=\ln(\frac{\sqrt{1+\mathrm e^x}-1}{\sqrt{1+\mathrm e^x}+1})+C$.

(12)$\int\frac{\sqrt{1+x^2}}x\mathrm dx\xlongequal{x=\tan t}\int\frac{\sec t\sec^2t\mathrm dt}{\tan t}=\int\frac{\mathrm dt}{\sin t\cos^2t}=\int\frac{\sin t\mathrm dt}{\sin^2t\cos^2t}=-\int\frac{\mathrm d\cos t}{(1-\cos^2t)\cos^2t}\\
=\int\frac{\mathrm d\cos t}{(\cos^2t-1)\cos^2t}\xlongequal{u=\cos t}\int\frac{\mathrm du}{(u^2-1)u^2}=\int(-\frac1{u^2}+\frac1{2(u-1)}-\frac1{2(u+1)})\mathrm du=\frac1u+\frac12\ln|\frac{u-1}{u+1}|+C\\
=\sec t+\frac12\ln|\frac{1-\sec t}{1+\sec t}|+C=\sqrt{x^2+1}+\frac12\ln(\frac{\sqrt{1+x^2}-1}{\sqrt{1+x^2}+1})+C$.

(13)$\int2\mathrm e^x\sqrt{1-\mathrm e^{2x}}\mathrm dx=\int2\sqrt{1-\mathrm e^{2x}}\mathrm d\mathrm e^x\xlongequal{\mathrm e^x=\sin t}\int2\cos t\mathrm d\sin t=\int2\cos^2t\mathrm dt\\
=\int(1+\cos2t)\mathrm dt=t+\frac12\sin2t+C=t+\sin t\cos t+C=\arcsin\mathrm e^x+\mathrm e^x\sqrt{1-\mathrm e^{2x}}+C$.

(14)$\int\frac{\mathrm dx}{x^2\sqrt{x^2+9}}\xlongequal{x=3\tan t}\int\frac{3\sec^2t\mathrm dt}{9\tan^2t3\sec t}=\frac19\int\frac{\cos t}{\sin^2t}\mathrm dt=\frac19\int{\mathrm d\sin t}{\sin^2t}=-\frac1{9\sin t}+C\\
=-\frac19\sqrt{\frac{\sin^2t+\cos^2t}{\sin^2t}}\mathrm{sgn}(\sin t)+C=-\frac19\sqrt{\frac{\tan^2t+1}{\tan^2t}}\mathrm{sgn}(\sin t)+C=-\frac19\sqrt{\frac{x^2+9}{x^2}}\mathrm{sgn}(x)+C\\
=-\frac19\frac{\sqrt{x^2+9}}x+C$(这里用到了$\mathrm{sgn}(x)=\mathrm{sgn}(\tan t)=\mathrm{sgn}(\sin t),t\in(-\frac\pi2,0)\cup(0,\frac\pi2)$).

(15)$\int\mathrm e^{\sqrt{2x-1}}\mathrm dx\xlongequal{t=\sqrt{2x-1}}\int\mathrm e^tt\mathrm dt=\int t\mathrm d\mathrm e^t=t\mathrm e^t-\int\mathrm e^t\mathrm dt=t\mathrm e^t-\mathrm e^t+C\\
=e^{\sqrt{2x-1}}(\sqrt{2x-1}-1)+C$.
\end{enumerate}
\end{document}